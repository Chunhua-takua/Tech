\section{应用题}

\title[第5讲\quad 应用题]{第5讲\quad 应用题} 
\author{}
\date{}

\begin{frame}
    \titlepage
\end{frame}

\setcounter{framecounter}{0}

\begin{frame}
    \stepcounter{framecounter}
    \frametitle{习题\theframecounter}
    \vspace*{-3cm}
    \textit{一筐水果中,恰好有一半数量是苹果,如果吃掉苹果数量的一半,筐中只剩下 60 个水果,那么,这时筐子中还有\underline{\hbox to 20mm{}}个苹果.} 
    % 2021;迎春杯四年级真题.pdf; 20
\end{frame}

\begin{frame}
    \stepcounter{framecounter}
    \frametitle{习题\theframecounter}
    \vspace*{-3cm}
    \textit{$2000\sim 2099$的某一年,小高和爷爷正在聊天,小高说:``爷爷,我发现今年我的年龄正好是我出生年份的末两位。''爷爷想了想说:``巧了,我也是.''如果爷爷的年龄正好是小高的3倍,那么,这一年是  \underline{\hbox to 20mm{}}年.(默认:年龄 = 这一年年份 - 出生年份)} 
    % 迎春杯四年级2022-试卷.pdf ; 2050
\end{frame}

\begin{frame}
    \stepcounter{framecounter}
    \frametitle{习题\theframecounter}
    \vspace*{-3cm}
    \textit{小迎、小春、小杯三人分享 141颗巧克力:已知小杯分到的巧克力比小迎和小春分到的巧克力之和的两倍多6颗,小迎分到的巧克力比小春分到的巧克力的三倍少3颗.那颗巧克力.么小迎分到  \underline{\hbox to 20mm{}}颗巧克力.} 
    % 2025;33
\end{frame}

\begin{frame}
    \stepcounter{framecounter}
    \frametitle{习题\theframecounter}
    \vspace*{-3cm}
    \textit{工人运瓷器 1000件,规定完整运到目的地一件给运费5元,损坏一件倒赔120 元;运完这批瓷器后,工人共得 4500元,那么损坏了 \underline{\hbox to 20mm{}}件.} 
    % 2023;2023YCB原题小中组(1).pdf;4
\end{frame}

\begin{frame}
    \stepcounter{framecounter}
    \frametitle{习题\theframecounter}
    \vspace*{-3cm}
    \textit{赵老师出的新书有普通版和签名版两种版本,普通版售价9元一本,签名版售价 10 元一本。有一位顾客花 37 元买了若干本赵老师的新书,那么其中有\underline{\hbox to 20mm{}}本签名版.} 
    % 2022;
\end{frame}

\begin{frame}
    \stepcounter{framecounter}
    \frametitle{习题\theframecounter}
    \vspace*{-3cm}
    \textit{某旅行团由6位男士和9位女士组成。已知6位男士的平均年龄是57岁9位女士的平均年龄是 52 岁。则这个旅行团所有成员的平均年龄是 \underline{\hbox to 20mm{}} 岁.} 
    % 2022华;54
\end{frame}

\begin{frame}
    \stepcounter{framecounter}
    \frametitle{习题\theframecounter}
    \vspace*{-3cm}
    \textit{红星小学围棋兴趣小组有4位小朋友和2位教练,4位小朋友依次相差2岁,2位教练相台差2岁,6个人年龄的平方和是2796岁,6个人年龄之和是 \underline{\hbox to 20mm{}} 岁.} 
    % 2022华;106
\end{frame}

\begin{frame}
    \stepcounter{framecounter}
    \frametitle{习题\theframecounter}
    \vspace*{-3cm}
    \textit{甲乙两人合作加工一批零件,如果甲先做10天,乙再做8天就可以完成全部工作;如果甲先做6天,乙再做16天也可以完成全部工作,那么如果甲单独加工这批零件,\underline{\hbox to 20mm{}}
    天能完成全部工作.} 
    % 2020华数之星初赛-三四年级真题.pdf
\end{frame}

\begin{frame}
    \stepcounter{framecounter}
    \frametitle{习题\theframecounter}
    \vspace*{-3cm}
    \textit{鸡兔关在同一个笼子里,鸡比兔子少20只. 兔脚的数量比鸡脚的数量的3倍多10只.那么鸡有\underline{\hbox to 20mm{}}只.} 
    % 2020华;
\end{frame}

\begin{frame}
    \stepcounter{framecounter}
    \frametitle{习题\theframecounter}
    \vspace*{-3cm}
    \textit{暑假期间小明在家写字.第一天写了比总字数的一半还少 50个的字,第二天写了余下字数的一半还少 20 个字,第三天写了再余下字数的一半多 10个字,第四天写了 60 个字,还剩 40 个字就全部写完了:小明假期一共要写\underline{\hbox to 20mm{}}个字.} 
    % 2020华数之星初赛-三四年级真题.pdf ; 700
\end{frame}

\begin{frame}
    \stepcounter{framecounter}
    \frametitle{习题\theframecounter}
    \vspace*{-3cm}
    \textit{某学校阶梯教室现有座位总数超过400个,但不超过440个,而且每一排都比前一排多两个座位,若第一排只有12个座位,这个阶梯教室一共有\underline{\hbox to 20mm{}}排座位.} 
    % 2020华;16
\end{frame}

\begin{frame}
    \stepcounter{framecounter}
    \frametitle{习题\theframecounter}
    \vspace*{-3cm}
    \textit{小花和小白一起去钓鱼,若小花把自己钓的鱼给小白2条后,小花剩下的鱼就是小白现在鱼数的4倍;若小花把自己钓的鱼给小白6条后,小花剩下的鱼则是小白现在鱼数的2倍,请问小花、小白各钓了几条鱼?} 
    % 2020华;26,4
\end{frame}

\begin{frame}
    \stepcounter{framecounter}
    \frametitle{习题\theframecounter}
    \vspace*{-3cm}
    \textit{六一儿童节,老师给幼儿园大班的小朋友分水果,已知苹果的总数是香蕉总数的 2 倍.如果给每个小朋友分3个香蕉,就多出5个香蕉;每个小朋友分7个苹果,就差两个苹果不够分,那么大班共有\underline{\hbox to 20mm{}}个小朋友,有\underline{\hbox to 20mm{}}个苹果.} 
    % 2022; 华数真题2021-2023(小中组).pdf ; 
\end{frame}

\begin{frame}
    \stepcounter{framecounter}
    \frametitle{习题\theframecounter}
    \vspace*{-3cm}
    \textit{某地7月的31天除了晴天就是雨天.有一颗新种的小树苗,在每个晴天会长高3毫米,在每个雨天会长高2毫米,但连续经过6个晴天,小树苗就会因为干旱而停止生长.那么整个7月小树苗最多长高\underline{\hbox to 20mm{}}毫米.} 
    % 2024数学花园探秘笔试小中年级夏季决赛C卷(B5试卷版).pdf ; 88
\end{frame}
