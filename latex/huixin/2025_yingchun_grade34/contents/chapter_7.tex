\section{计数问题}

\title[第7讲\quad 综合]{第7讲\quad 综合} 
\author{}
\date{}

\begin{frame}
    \titlepage
\end{frame}

\setcounter{framecounter}{0}

\begin{frame}
    \stepcounter{framecounter}
    \frametitle{习题\theframecounter}
    \textit{下列竖式中,相同汉字表示相同数字,不同汉字表示不同数字,且所有汉字对应的数字都不是 0,2,5。“空歌风度清”表示的五位数是\underline{\hbox to 20mm{}}.}
    \begin{figure}[H] 
        \centering
        \includegraphics[width=0.4\textwidth]{./pics/Chapter_7/1.png}
    \end{figure}
    % 2025;16934
\end{frame}

\begin{frame}
    \stepcounter{framecounter}
    \frametitle{习题\theframecounter}
    \textit{下面的算式中,相同的汉字代表相同的数字,不同的汉字代表不同的数字,那么\myoverline{龙行天下}  表示的四位数是\underline{\hbox to 20mm{}}.}
    \begin{figure}[H] 
        \centering
        \includegraphics[width=0.4\textwidth]{./pics/Chapter_7/2.png}
    \end{figure}
    % 2024
\end{frame}

\begin{frame}
    \stepcounter{framecounter}
    \frametitle{习题\theframecounter}
    \textit{将 $1\sim 9$分别填入到右图的方框中,每个数字用一次,使得竖式成立;现在数字 6、7、8已经被填入,那么竖式的和是\underline{\hbox to 20mm{}}.}
    \begin{figure}[H] 
        \centering
        \includegraphics[width=0.4\textwidth]{./pics/Chapter_7/3.png}
    \end{figure}
    % 2023; 819
\end{frame}

\begin{frame}
    \stepcounter{framecounter}
    \frametitle{习题\theframecounter}
    \textit{老虎、狐狸、猴子各3只分别入住右图的9个房间中,每个房间一只,结果每只动物都说“有老虎与我相邻"(有公共边的两个房间相邻);如果老虎都说真话,狐狸都说假话,猴子说的话不知真假,那么说真话的猴子所在房间编号的和是\underline{\hbox to 20mm{}}.}
    \begin{figure}[H] 
        \centering
        \includegraphics[width=0.4\textwidth]{./pics/Chapter_7/4.png}
    \end{figure}
    % 2023; 15
\end{frame}

\begin{frame}
    \stepcounter{framecounter}
    \frametitle{习题\theframecounter}
    \textit{季老师将分别写有1~9的九张卡片平均分给甲、乙、丙三人,每人只能看到自己的三张卡片(6 翻转后可以看成 9,9 翻转后可以看成 6).甲说:你们的数都无法组成形如“AXB=C(A,B,C互不相同)”的乘法等式乙说:听了甲的话,我就知道甲手中的三个数分别是多少了如果他们都足够聪明且诚实,那么丙手中三个数的乘积是\underline{\hbox to 20mm{}}.}
    % 2024; 35
\end{frame}

\begin{frame}
    \stepcounter{framecounter}
    \frametitle{习题\theframecounter}
    \textit{有一些自然数,如 121和 2552,从左到右和从右到左的数字顺序相同,我们把这样的自然数叫做“回文数”. 已知两个回文数的和是 2022,则这两个回文数的差是\underline{\hbox to 20mm{}}.}
    % 2022; 
\end{frame}


\begin{frame}
    \stepcounter{framecounter}
    \frametitle{习题\theframecounter}
    \textit{丽丽想用大小为 1x1、2x2、3x3 的三种正方形拼成下图所示的领奖台(图中每个小正方形的边长为1),所用正方形的面积总和为 15,且拼接过程中不可重叠,每种正方形数量不限(可以不用).共有\underline{\hbox to 20mm{}}种不同的拼接方法.(正方形的摆放位置或数量不同都算不同的拼法).}
    \begin{figure}[H] 
        \centering
        \includegraphics[width=0.4\textwidth]{./pics/Chapter_7/7.png}
    \end{figure}
    % 2022; 
\end{frame}

\begin{frame}
    \stepcounter{framecounter}
    \frametitle{习题\theframecounter}
    \textit{甲、乙、丙、丁四位绝世高手相约于华山之年比武论剑,争夺“天下第一”的名号.四人抽签两两分组各自决出胜者,两位胜者再次比试决出-二名,两位败者决出三四名,比武结束后四人进行了交流,每人说了两句话:
用:我在与丙的大战中获得了胜利,但是成输给了丁.
乙:我作为天下第一实至名归,甲的名次匕丙靠后.中输给了丁丙:我被甲的六脉神剑击败了。乙在比试
丁:好在我的名次没有垫底,丙的名次比,靠前己知每个人在比试中胜了几场就说几句真话那么甲乙丙丁最终的名次按顺序组成的四位数为\underline{\hbox to 20mm{}}.}
    % 2022; 
\end{frame}


\begin{frame}
    \stepcounter{framecounter}
    \frametitle{习题\theframecounter}
    \textit{在算式 $\overline{A0AA}= A\times B\times \overline{BBA}$中,A、B 分别代表不同的数字.那么, $\overline{AB}$ 代表的两位数是\underline{\hbox to 20mm{}}.}
    % 2022; 73
\end{frame}


\begin{frame}
    \stepcounter{framecounter}
    \frametitle{习题\theframecounter}
    \textit{如图所示,正五边形的五个顶点位置分别标记为$1\sim 5$,甲乙丙戊分别站在了这五个不同的顶点上,发生如下对话:
甲对乙说:我所在位置上的数比你大
乙对丙说:我所在位置上的数比你小
丙对丁说:我所在位置上的数比你大
丁对戊说:我所在位置上的数比你小
戊对甲说:我所在位置上的数比你大
如果五个人说的都是真话,且任意发生对话的两个人都不相邻. 那么甲乙丙丁戊所站位置按顺序连成的五位数是\underline{\hbox to 20mm{}}.}
    % 2022; 31425
\end{frame}


\begin{frame}
    \stepcounter{framecounter}
    \frametitle{习题\theframecounter}
    \textit{甲、乙二人按如下顺序填写下图中的减法算式:①甲选择一个数字,然后乙选择一个位置将这个数字填入:②甲选择一个之前没有选择过的数字然后乙选择一个位置将这个数字填入③甲选择一个之前没有选择过的数字然后乙将它填入最后一个位置组这两个数的差尽量小,那么甲、乙都安照最佳策略填写算式如果甲希望两个数的差尽量大,乙希时,差是\underline{\hbox to 20mm{}}.}
    % 2022; 1854
\end{frame}


\begin{frame}
    \stepcounter{framecounter}
    \frametitle{习题\theframecounter}
    \textit{在右图的加法竖式中,6个汉字恰好代表6个连续的数字,那么,花园探秘 所代表的四位数是\underline{\hbox to 20mm{}}.}
    \begin{figure}[H] 
        \centering
        \includegraphics[width=0.4\textwidth]{./pics/Chapter_7/12.png}
    \end{figure}
    % 2021; 8354
\end{frame}


\begin{frame}
    \stepcounter{framecounter}
    \frametitle{习题\theframecounter}
    \textit{甲乙丙丁四个人各有一些糖果,他们之间的对话如下:\\
    甲:如果把我的糖果数量变成和丙一样多,我们4人的平均数会减少2;\\
    乙:如果我的糖果数量变成和丁一样多,我们4人的平均数会减半;\\
    丙:如果我的糖果数量变为原来2倍,而甲的数量减半,我们4人的平均数会增加 2;\\
    丁:如果我的糖果数量变为原来2倍,而乙的数量减半,我们4人的平均数恰好会是一个整十数.
事实证明,他们4人中只有糖果数量最少的人说了假话,并且糖果最多人的糖果数恰好是糖果最少人糖果数的3倍.那么,他们4人一共有\underline{\hbox to 20mm{}}颗糖果.}
    % 2021; 120
\end{frame}

\begin{frame}
    \stepcounter{framecounter}
    \frametitle{习题\theframecounter}
    \textit{在右面的乘法竖式中,相同的汉字代表相同的数字,不同的汉字代表不同的数字;那么,\myoverline{迎接夏天} 代表的四位数是\underline{\hbox to 20mm{}}.}
    \begin{figure}[H] 
        \centering
        \includegraphics[width=0.4\textwidth]{./pics/Chapter_7/14.png}
    \end{figure}
    % 2021; 1024
\end{frame}

\begin{frame}
    \stepcounter{framecounter}
    \frametitle{习题\theframecounter}
    \textit{甲、乙、丙、丁共有糖果 17颗,且每人的糖果数都不超过9颗,他们有如下的对话:\\
    甲对乙说:``如果我给你1颗糖,我们的糖果数就相同了.''\\
    乙对甲说:``如果你给我2颗糖,我的糖果数就是你的3倍了.''\\
    丙对甲说:``如果我给你3颗糖,你的糖果数就是我的3倍了.''\\
    丁对甲说:``如果你给我4颗糖,我的糖果数就是你的4倍了.''\\
    结果发现:糖果数是奇数的人说的都是对的,而糖果数是偶数的人说的都是错的.设甲、乙、丙、丁依次拥有 A、B、C、D颗,那么,四位数$\overline{ABCD}$是\underline{\hbox to 20mm{}}.}
    % 2021; 3158
\end{frame}