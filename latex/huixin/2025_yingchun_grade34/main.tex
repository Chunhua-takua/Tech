\documentclass{beamer}

\mode<presentation> {
	\usetheme{Madrid}
	% \usecolortheme{spruce}   % 浅绿色
	% \setbeamertemplate{navigation symbols}{}  % 隐藏导航符号

	% \setbeamertemplate{footline}
	%若要删除所有幻灯片中的页脚,请取消注释此行
	
	%\setbeamertemplate{footline}[页码]
	%若要用简单的幻灯片计数替换所有幻灯片中的页脚,请取消注释此行
	
	%\setbeamertemplate{导航符号}{}
	%要删除所有幻灯片底部的导航符号,请取消注释此行
}

\usepackage{graphicx} % 允许包含图像
\usepackage{booktabs} % 允许在表中使用\toprule、\ midrule和\ bottomrule
\usepackage[UTF8,noindent]{ctexcap}  % 使用中文输入及显示
\usepackage{gensymb}
\usepackage[bookmarks=true]{hyperref}
\usepackage{tikz}  % 用于图片处理和定位
\usepackage{graphicx}  % 用于插入图片

%-----------------------------------
%	以下为正文
%-----------------------------------

\title[]{2025年迎春杯(小中)数学讲义} 

\author{}
\institute[] % 您的机构将出现在每张幻灯片的底部,可能是节省空间的简写
{
	% 会心升学 \\
	% \medskip
	% \textit{@163.com} % Your email address
}
\date{\today}

\renewcommand{\baselinestretch}{1.3}

\begin{document}
	\begin{frame}
		\vspace{-1cm}
		\begin{flushleft}
			\includegraphics[width=1cm]{./pics/huixin_logo.png}
		\end{flushleft}
		\titlepage
	\end{frame}

	\begin{frame}
		\frametitle{目录}
		\tableofcontents % 在整个演示过程中,如果您选择使用\ section{}和\ submission{}命令,这些命令将自动打印在此幻灯片上,作为演示的概述
	\end{frame}
	
	%-----------------------------------
	%	开始创建PPT
	%-----------------------------------
	
	% frame counter
	\newcounter{framecounter}
	\section{第1讲\quad 计算}


\item {
    【乘法分配律】
    $(200+2)\times 5 + 5020$ 
    \ifshowSolution
        \fangsong\zihao{4}
        \\
        思路:改。2025

        正解: 
    \else
        \\ \\ \\
    \fi
}

% \item {
%     $(20+2)\times 5 + 2025$ 
%     \ifshowSolution
%         \fangsong\zihao{4}
%         \\
%         思路:2025

%         正解: 
%     \else
%         \\ \\ \\
%     \fi
% }


\item {
    【乘法分配律】
    $7\times 19 + 3\times 13\times 41 + 13\times 19$
    \ifshowSolution
        \fangsong\zihao{4}
        \\
        思路:改。2021;迎春杯四年级真题.pdf

        正解: 
    \else
        \\ \\ \\
    \fi
}


\item {
    【乘法分配律】
    $(18\times 23 - 24\times 17)\div 3 + 5$
    \ifshowSolution
        \fangsong\zihao{4}
        \\
        思路:

        正解: 
    \else
        \\ \\ \\
    \fi
}

\item {
    【乘法分配律】
    $(11\times 24 - 23\times 9)\div 3 + 3$
    \ifshowSolution
        \fangsong\zihao{4}
        \\
        思路:

        正解: 
    \else
        \\ \\ \\
    \fi
}

% \item {
%     【乘法分配律】
%     $7\times 17 + 3\times 13\times 43 + 13\times 17$
%     \ifshowSolution
%         \fangsong\zihao{4}
%         \\
%         思路:2021;迎春杯四年级真题.pdf

%         正解: 2017
%     \else
%         \\ \\ \\
%     \fi
% }

% \item {
%     $(9\times 8\times 7 + 6 - 5)\times 4 + 3 -2 +1$
%     \ifshowSolution
%         \fangsong\zihao{4}
%         \\
%         思路: 迎春杯四年级2022-试卷.pdf

%         正解: 2022
%     \else
%         \\ \\ \\
%     \fi
% }

\item {
    【乘法凑10】
    $12\times 25 + 16\times 15$
    \ifshowSolution
        \fangsong\zihao{4}
        \\
        思路:

        正解: 
    \else
        \\ \\ \\
    \fi
}

\item {
    【乘法凑10】
    $5\times 432\times 1 - 98 - 7\times 6$
    \ifshowSolution
        \fangsong\zihao{4}
        \\
        思路: 2020数学花园探秘笔试小中决赛D卷.doc

        正解: 2020
    \else
        \\ \\ \\
    \fi
}

\item {
    【加法凑10】
    $1+3+4+6+7+9+10 + 12$
    \ifshowSolution
        \fangsong\zihao{4}
        \\
        思路:

        正解: 
    \else
        \\ \\ \\
    \fi
}

\item {
    【乘法凑10】
    $210\times 6 - 52\times 5$
    \ifshowSolution
        \fangsong\zihao{4}
        \\
        思路:

        正解: 
    \else
        \\ \\ \\
    \fi
}

\item {
    【尾同头合十】
    $5000- 22\times 82$  
    \ifshowSolution
        \fangsong\zihao{4}
        \\
        思路:

        正解: 
    \else
        \\ \\ \\
    \fi
}


\item {
    【头同尾合十】
    $67\times 62 - 34 + 67 + 34$
    \ifshowSolution
        \fangsong\zihao{4}
        \\
        思路:

        正解: 
    \else
        \\ \\ \\
    \fi
}

\item {
    【等差数列求和公式】
    $(1+3+5+\cdots + 89) - (1+2+3+\cdots + 63)$  
    \ifshowSolution
        \fangsong\zihao{4}
        \\
        思路:

        正解: 
    \else
        \\ \\ \\
    \fi
}

\item {
    【立方和公式】
    $3^3 + 4^3 + 5^3 + 6^3 + 7^3 + 8^3 + 9^3$
    \ifshowSolution
        \fangsong\zihao{4}
        \\
        思路:

        正解: 
    \else
        \\ \\ \\
    \fi
}

\item {
    【数值计算;数字谜】有一些自然数,如 121 和 2552,从左到右和从右到左的数字顺序相同,我们把这样的自然数叫做``回文数''. 已知两个回文数的和是 2022,则这两个回文数的差是\underline{\hbox to 20mm{}}.
    \ifshowSolution
        \fangsong\zihao{4}
        \\
        思路: 综合-数字谜.

        正解:  迎春杯三年级2022-试卷.pdf; 1740
    \else
        \\ \\ \\
    \fi
}


\item {
    【数值计算】$99\times 10101\times 111\times 1001001$的末5位数字是多少?
    \ifshowSolution
        \fangsong\zihao{4}
        \\
        思路:

        正解: 88889
    \else
        \\ \\ \\
    \fi
}
	\section{字典排列法}

\title[第2讲\quad 字典排列法]{第2讲\quad 字典排列法} 
\author{}
\date{}
\begin{frame}
    \titlepage
\end{frame}

\begin{frame}
    \frametitle{课前测}
    \vspace*{-2cm}
    \textit{用一块长8分米,宽4分米的长方形纸板与两块边长4分米的正方形纸板拼成一个正方形.拼成的正方形的周长是多少分米?\\}
    \textit{A.12\\ B.24\\ C.32\\ D.40}
\end{frame}

\begin{frame}
    \frametitle{课前测}
    \vspace*{-2cm}
    \begin{figure}[H] 
        \centering
        \includegraphics[width=1\textwidth]{./pics/Chapter_2/keqian2.png}
    \end{figure}
\end{frame}

\begin{frame}
    \frametitle{课前测}
    \vspace*{-2cm}
    \begin{figure}[H] 
        \centering
        \includegraphics[width=1\textwidth]{./pics/Chapter_2/keqian3.png}
    \end{figure}
\end{frame}

\begin{frame}
    \frametitle{知识梳理}
\end{frame}

\begin{frame}
    \frametitle{MISSION 1}
    \vspace*{-2cm}
    \textit{同学们,在进行字典排列法时,我们应当如何做到不重不漏呢?}
\end{frame}

\begin{frame}
    \frametitle{探索1}
    \vspace*{-2cm}
    \textit{(1)用数字1、2、3可以组成多少个不同的无重复数字的两位数?}
\end{frame}

\begin{frame}
    \frametitle{探索1}
    \vspace*{-2cm}
    \textit{(2)用数字1、3、6可以组成多少个不同的无重复数字的三位数?}
\end{frame}

\begin{frame}
    \frametitle{探索2}
    \vspace*{-2cm}
    \textit{(1)用数字1、2、3可以组成多少个不同的两位数?}
\end{frame}

\begin{frame}
    \frametitle{探索2}
    \vspace*{-2cm}
    \textit{(2)用数字1、3、6可以组成多少个不同的三位数?}
\end{frame}

\begin{frame}
    \frametitle{探索3}
    \vspace*{-2cm}
    \textit{用数字1、2、3可以组成多少个不同的无重复数字的自然数?}
\end{frame}

\begin{frame}
    \frametitle{课堂互动1}
    \begin{figure}[H] 
        \centering
        \includegraphics[width=1\textwidth]{./pics/Chapter_2/ketanghudong1.png}
    \end{figure}
\end{frame}

\begin{frame}
    \frametitle{课堂互动2}
    \begin{figure}[H] 
        \centering
        \includegraphics[width=1\textwidth]{./pics/Chapter_2/ketanghudong2.png}
    \end{figure}
\end{frame}

\begin{frame}
    \frametitle{课堂互动3}
    \begin{figure}[H] 
        \centering
        \includegraphics[width=1\textwidth]{./pics/Chapter_2/ketanghudong3.png}
    \end{figure}
\end{frame}

\begin{frame}
    \frametitle{捉虫时刻}
    \vspace*{-2cm}
    \textit{艾迪这道题的做法对么?如果不对,请你写一下正确的解法。\\
    用数字1、2能组成多少个不同的三位数 ?\\
    由于要组成三位数,因此有一个数字要重复.\\
    一一列举:112、121、211、221、212、122,共有6个.}
\end{frame}

\begin{frame}
    \frametitle{MISSION 2}
    \vspace*{-2cm}
    \textit{在无序问题中,我们应当如何进行枚举?}
\end{frame}

\begin{frame}
    \frametitle{探索4}
    \vspace*{-2cm}
    \textit{艾迪与薇儿做游戏,在分别标有1到10的10个完全一样的小球中,艾迪任意取出2个小球(不计取出顺序),由薇儿计算两球所标的数之和,若和大于10,则薇儿获胜.那么能使薇儿获胜的小球取法共有多少种?}
\end{frame}

\begin{frame}
    \frametitle{探索5}
    \vspace*{-2cm}
    \textit{艾迪去儿童餐厅买15元特惠套餐,他有若干张1元、2元、5元的纸币,但是购买特惠套餐的条件是必须找出一共有多少种不同的付钱方法(要求每种纸币都有),眼看优惠时间就要截止了,同学们你能帮助艾迪顺利买到优惠套餐吗 ?}
\end{frame}

\begin{frame}
    \frametitle{探索6}
    \vspace*{-2cm}
    \textit{在某地有四种不同面值的硬币,如图所示,假若你恰有这四种硬币各1枚。问:共能组成多少种不同的钱数?}
    \begin{figure}[H] 
        \centering
        \includegraphics[width=0.5\textwidth]{./pics/Chapter_2/tansuo6.png}
    \end{figure}
\end{frame}

\begin{frame}
    \frametitle{补充1}
    \textit{薇儿收集到四种不同的面值的硬币各1枚,如图所示,一共可以组成多少种不同的钱数?}
    \begin{figure}[H] 
        \centering
        \includegraphics[width=0.3\textwidth]{./pics/Chapter_2/buchong1_1.png}
    \end{figure}
\end{frame}

\begin{frame}
    \frametitle{补充1}
    \textit{艾迪收到三种不同面值的硬币,如图所示,假若你恰好有以下四枚硬币.问共能组成多少种不同的钱数?}
    \begin{figure}[H] 
        \centering
        \includegraphics[width=0.3\textwidth]{./pics/Chapter_2/buchong1_2.png}
    \end{figure}
\end{frame}

\begin{frame}
    \frametitle{补充1}
    \textit{用四种不同的硬币各1枚,如图所示,两两一组,一共可以组成多少种不同的钱数?}
    \begin{figure}[H] 
        \centering
        \includegraphics[width=0.3\textwidth]{./pics/Chapter_2/buchong1_3.png}
    \end{figure}
\end{frame}

\begin{frame}
    \frametitle{探索7}
    \vspace*{-2cm}
    \textit{博士给艾迪与薇儿上课,课上介绍了``拐弯''的概念.\\
    博士:``对于一行数,如果有三个数abc依次排一起,且$a > b, c > b$或者$a <b,c<b$,我们就称它发生了一次拐弯''\\
    艾迪:``我懂了,比如4321没有拐弯,像1243就发生了一次拐弯''\\
    薇儿:``没错,再比如1324就发生了两次拐弯.''\\
    博士:``非常棒!看来你们都掌握得非常扎实了,现在我要考考你们了,如果我们将1,2,3,4排成一行,则能使这行数刚好发生两次拐弯的排列方法共有多少种?''}
\end{frame}

\begin{frame}
    \frametitle{探索8}
    \vspace*{-2cm}
    \textit{一次,齐王与田忌赛马,每人各有等级不同的4匹马,这8匹马按照从快到慢的排序分别是齐王的一等马,田忌的一等马,齐王的二等马,田忌的二等马,齐王的三等马,田忌的三等马,齐王的四等马,田忌的四等马.田忌已经提前知道齐王本次赛马的出场顺序是一等、二等、三等、四等,他可以安排种不同的出场顺序,才能保证自己至少战平齐王呢?同学们,你们能帮田忌找出所有可能的决策吗 ?}
\end{frame}

\begin{frame}
    \frametitle{补充2}
    \textit{加加与减减做游戏,两人轮流在一张白纸上写出一个数字,组成一个多位数的前2位,而这个多位数从第三个数字开始,每个数字都恰好是它前面两个数字之和,直至不能再写为止.例如加加减减写了1和4,那么这个多位数就是1459,则这类多位数共有多少个?}
\end{frame}

\begin{frame}
    \frametitle{补充2}
    \textit{如果一个数的各位数字从左到右构成等差数串,我们就称这个数为“跳跃数”,例如:1358642均是“跳跃数”,153就不是“跳跃数”,那么一共有多少个三位“跳跃数”?}
\end{frame}

\begin{frame}
    \frametitle{思维导图}
    \begin{figure}[H] 
        \centering
        \includegraphics[width=1\textwidth]{./pics/Chapter_2/siweidaotu.png}
    \end{figure}
\end{frame}
	\section{第3讲\quad 行程问题与应用题}

\item {
    【追及】
    猎豹跑一步长为2米,狐狸跑一步长为1米. 猎豹跑2步的时间狐狸跑3步. 猎豹距离狐狸30米,则猎豹跑动\underline{\hbox to 20mm{}}米可追上狐狸. 
    \ifshowSolution 
        \fangsong\zihao{5}\textcolor{blue}{
            \\正解: 120m.\\
                不妨假设猎豹1秒跑2步,那么狐狸1秒跑3步;\\
                那么,猎豹的速度是 \[2\times 2 = 4m/s,\] 
                狐狸的速度是 
                \[1\times 3 = 3 m/s.\]
                \[距离\div 速度差 = 追及所需时间\]
                \[30\div (4-3) = 30(s). \]
                \[30 \times 4 = 120 (m).\]
        }
    \else
        \vspace{2cm}
    \fi
    % 2017年第二十二届“华罗庚金杯”少年数学邀请赛初赛试卷(小中组).doc, 120
}

\item {
    【相遇】
    一辆公共汽车和一辆小轿车同时从相距450千米的两地相向而行,公共汽车每小时行40千米,小轿车每小时行50千米,\underline{\hbox to 20mm{}}小时后两车第二次相距90千米. 
    \ifshowSolution 
        \fangsong\zihao{5}\textcolor{blue}{
            \\正解: \\
                第2次相遇时,两车一共走 \[450+90=540 km;\]
                \[540\div (40+50) = 6h.\]
        }
    \else
        \vspace{2cm}
    \fi
    % 2020数学花园探秘笔试小中决赛D卷.doc; 6
}

\item {
    【相遇】
    里山镇到省城的高速路全长189千米,途径县城. 县城离里山镇54千米. 早上8: 30一辆客车从里山镇开往县城,9: 15到达. 停留15分钟后开往省城,午前11: 00能够到达. 另有一辆客车于当日早上9: 00从省城径直开往里山镇. 每小时行驶60千米. 两车相遇时,省城开往里山镇的客车行驶了\underline{\hbox to 20mm{}}分钟.
    \ifshowSolution 
        \fangsong\zihao{5}\textcolor{blue}{
            \\正解: \\
            \begin{figure}[H] 
                \centering
                \includegraphics[width=0.6\textwidth]{./pics/Chapter_3/seikai_1.png}
            \end{figure}
                从里山开往省城的客车,9:30从县城出发,县城到省城段速度:
                    \[11:00 - 9:30 = 1.5 h\]
                    \[(189-54)\div 1.5=90 km/h\]
                9:00到9:30, 从省城出发的客车行驶了0.5h,行驶距离:
                    \[60\times 0.5 = 30km\]
                所以,在9:30两车相距
                    \[135 - 30 = 105 km\]
                两车相遇还需要
                \[105\div (90+60) = 0.7h,\]
                此时,从省城出发的客车行驶了
                \[0.5 + 0.7 = 1.2(h) = 72(min).\]
        }
    \else
        \vspace{2cm}
    \fi
    % 2012年第十七届“华罗庚金杯”少年数学邀请赛网上初赛试卷(小学中低年级组).doc, 72
}

\item {
    【多次追及】
    哥哥和弟弟两人同时从家出发去 2000米外的学校上学,哥哥每分钟走 60米,弟弟每分钟走 50 米,走了 10 分钟后,哥哥发现忘记带数学错题本,就以每分钟 100 米的速度跑回家,回到家后,哥哥用了2分钟找到了错题本,然后以每分钟 150 米的速度往学校跑.从哥哥第二次从家出发开始计算,经过\underline{\hbox to 20mm{}}分钟后,哥哥能追上弟弟.
    \ifshowSolution 
        \fangsong\zihao{5}\textcolor{blue}{
            \\正解: \\
            \begin{figure}[H] 
                \centering
                \includegraphics[width=0.7\textwidth]{./pics/Chapter_3/seikai_2.png}
            \end{figure}
                从哥哥第一次从家里出发,到再次出发,经过的时间如下.\\
                $t_1=10 min$: 第一次出发到发现忘带本子;此时哥哥走了
                \[60\times 10 = 600 (m).\]
                $t_2=600\div 100 = 6(min)$: 从发现忘带本子到跑回家.\\
                $t_3=2(min)$: 到家找到本子第二次从家里出发.\\
                在以上时间内,弟弟都以50m/min 的速度向学校走去,共走了
                \[ 50\times (10+6+2) = 900 m.\]
                哥哥第二次从家里出发,需要
                \[900\div (150 - 50) = 9 min \]
                追上弟弟.
        }
    \else
        \vspace{2cm}
    \fi
    % 2020华数之星初赛-三四年级真题.pdf; 9
}

\item {
    【多次相遇】
    甲、乙两人在一条长120米的直路上来回跑,甲的速度是5米/秒,乙的速度是3米/秒,若他们同时从同一端出发跑了15分钟,则他们在这段时间内共迎面相遇\underline{\hbox to 20mm{}}次(端点除外). 
    \ifshowSolution 
        \fangsong\zihao{5}\textcolor{blue}{
            \\正解: \\
                甲乙两人迎面相遇时,2人一共的行程是2个单程:
                $$120\times 2=240(米)$$
                用时为
                $$240\div (3+5)=30(秒)$$
                即每30秒就相遇一次(包括端点相遇).
                端点相遇用时为:
                甲单程用时:
                 $$120\div 3 = 40 (s),$$
                乙单程用时:
                $$120\div 5=24 (s)$$
                两人迎面相遇的时间为40和24的公倍数,最小公倍数是120.
                $$120\div 30=4$$
                可知,他们4次相遇中就有1次为端点相遇,
                即15分钟内相遇的总次数为:
                 $$15\times 60\div 30 = 30(次),$$
                其中在端点相遇的次数为 $30\div 4$ 的整数部分即7.
                所以他们在这段时间内共迎面相遇(端点除外)的次数为:
                $$30-7=23(次).$$
        }
    \else
        \vspace{2cm}
    \fi
    % 2015华, 23
}

\item {
    【多次相遇】
    甲、乙两车分别从A,B两地同时出发,相向匀速行进,在距A地 60 千米处相遇. 相遇后,两车继续行进,分别到达B,A后, 立即原路返回, 在距B地50 千米处再次相遇. 则A,B两地的路程是\underline{\hbox to 20mm{}}千米. 
    \ifshowSolution 
        \fangsong\zihao{5}\textcolor{blue}{
            \\正解: \\
            相遇时距离之比等于速度之比.\\
            设AB两地间的距离为S km.\\
            第一次相遇时,甲走了 60千米,而乙走了S-60 千米; \\
            第二次相遇,甲又走了S-60+50千米,乙又走了60+S-50千米.\\
            \begin{align*}
                \frac{60}{S-60} &= \frac{S-60+50}{60+S-50} \\
                S&=130.
            \end{align*}
        }
    \else
        \vspace{2cm}
    \fi
    % 2016华, 130
}

\item {
    【多次相遇】
    甲、乙两车分别从A,B两地同时出发,相向而行,3小时后相遇,甲掉头返回A地,乙继续前行. 甲到达A地后掉头往B行驶,半小时后和乙相遇. 那么乙从A到B共需\underline{\hbox to 20mm{}}小时.
    \ifshowSolution 
        \fangsong\zihao{5}\textcolor{blue}{
            \\正解:\\ 
                相遇后,甲还需要3小时返回甲地,\\
                第二次相遇时,甲距离第一次相遇点的距离等于甲 2.5 小时的路程,乙用了3.5小时走这些路程,所以甲乙速度的比是7:5,\\
                甲乙相遇需要3小时,那么乙单独到需要
                \[ 3\times \frac{7+5}{5} = 7.2小时.\]
        }
    \else
        \vspace{2cm}
    \fi
    % 2011年第十六届“华罗庚金杯”少年数学邀请赛初赛试卷(小学组).doc, 7.2
}

\item {
    【行程与时间计算】
    小文今天和朋友约定一起看 12:00 开场的电影,出门时,发现挂钟电池没电已经停止了,她把挂钟换好电池,但没来得及调整时间,出门前挂钟显示的时间是 9:25,小文赶到电影院时,电影刚好开场.电影结束后,小文立刻返回家中,发现挂钟显示的时间是 13:55,小文赶紧把它调成正确的时间15:45.如果小文从家到电影院和从电影院返回家中花的时间是一样的,那么,电影的时长是\underline{\hbox to 20mm{}}分钟.
    \ifshowSolution{}
        \fangsong\zihao{5}\textcolor{blue}{
            \\正解:
            \begin{figure}[H] 
                \centering
                \includegraphics[width=0.7\textwidth]{./pics/Chapter_3/seikai_8.png}
            \end{figure}
            钟从 9:25到13:55(真实时间为15:45),用时 
            \[ 13:55 - 9:25 = 4h 30 min. \]
            真实时间从12:00 到 15:45, 用时
            \[ 15:45 - 12:00 = 3h 45 min. \]
            钟9:25到真实12:00, 用时
            \[
                4h 30min - 3h 45 min = 45min
            \]
            所以,电影时长:
            \[3h 45min - 45min = 3h = 180min. \]
        }
    \else
        \vspace{2cm}
    \fi
    % 迎春杯四年级2022-试卷.pdf; 180
}

\item {
    【流水行船·追及·相遇】
    一条河上有A,B两个码头,A在上游,B在下游. 甲、乙两人分别从A,B同时出发,划船相向而行,4小时后相遇. 如果甲、乙两人分别从A,B同时出发,划船同向而行,乙16小时后追上甲. 已知甲在静水中划船的速度为每小时6千米,则乙在静水中划船每小时行驶\underline{\hbox to 20mm{}}千米.
    \ifshowSolution{}
        \fangsong\zihao{5}\textcolor{blue}{
            \\正解:\\
            两船相遇的速度即两船的速度和, 两船追及速度即两船的速度差.\\
            相向而行两船所行的路程是 A、B两个码头之间的距离; \\
            同向而行两船的距离差也为 A、B两个码头之间的距离. \\
            设乙船的速度是x千米/小时, 列出方程 
            \[(x+6)\times 4=(x-6)\times 16 \]
            \[ x=10. \]
        }
    \else
        \vspace{2cm}
    \fi
    % 2015华, 10
}


	\setbeamertemplate{background canvas}{
		\includegraphics[width=\paperwidth,height=\paperheight]{./pics/end.jpg}
	}
	\begin{frame}
		\Huge{\centerline{下次课见}}
	\end{frame}
	%------------------------------------------------
	% \subsection{Subsection Example 2}
	
	% \begin{frame}
	% 	\frametitle{Bullet Points}
	% 	\begin{itemize}
	% 		\item What we do may be small, but it has a certain character of permanence.
	% 		\item Euclid geometry was as dazzling as first love.
	% 		\item Talk is cheap, solve the PDE.
	% 	\end{itemize}
	% \end{frame}
	
	% %------------------------------------------------
	
	% \begin{frame}
	% 	\frametitle{Blocks of Highlighted Text}
	% 	\begin{block}{Block 1}
	% 		Certainly the best times were when I was alone with mathematics: free of ambition and pretense, and indifferent to the world.
	% 	\end{block}
		
	% 	% \begin{block}{Block 2}
	% 	% 	Wir müssen wissen, Wir werden wissen.
	% 	% \end{block}
		
	% 	\begin{block}{Block 3}
	% 		If people do not believe that mathematics is simple, it is only because they do not realize how complicated life is.	
	% 	\end{block}
	% \end{frame}
	
	% %------------------------------------------------
	
	% \begin{frame}
	% 	\frametitle{Multiple Columns}
	% 	\begin{columns}[c] % The "c" option specifies centered vertical alignment while the "t" option is used for top vertical alignment
			
	% 		\column{.45\textwidth} % Left column and width
	% 		\textbf{Heading}
	% 		\begin{enumerate}
	% 			\item Statement
	% 			\item Explanation
	% 			\item Example
	% 		\end{enumerate}
			
	% 		\column{.5\textwidth} % Right column and width
	% 		Lorem ipsum dolor sit amet, consectetur adipiscing elit. Integer lectus nisl, ultricies in feugiat rutrum, porttitor sit amet augue. Aliquam ut tortor mauris. Sed volutpat ante purus, quis accumsan dolor.
			
	% 	\end{columns}
	% \end{frame}
	
	% %------------------------------------------------
	% \section{Second Section}
	% %------------------------------------------------
	
	% \begin{frame}
	% 	\frametitle{Table}
	% 	\begin{table}
	% 		\begin{tabular}{l l l}
	% 			\toprule
	% 			\textbf{Treatments} & \textbf{Response 1} & \textbf{Response 2}\\
	% 			\midrule
	% 			Treatment 1 & 0.0003262 & 0.562 \\
	% 			Treatment 2 & 0.0015681 & 0.910 \\
	% 			Treatment 3 & 0.0009271 & 0.296 \\
	% 			\bottomrule
	% 		\end{tabular}
	% 		\caption{Table caption}
	% 	\end{table}
	% \end{frame}
	
	% %------------------------------------------------
	
	% \begin{frame}
	% 	\frametitle{Theorem}
	% 	\begin{theorem}[Mass--energy equivalence]
	% 	\centerline{$E = mc^2$}
	% 	\end{theorem}
	% \end{frame}
	
	%------------------------------------------------
	
	% \begin{frame}[fragile] % Need to use the fragile option when verbatim is used in the slide
	% 	\frametitle{Verbatim}
	% 	\begin{example}[Theorem Slide Code]
	% 		\begin{verbatim}
	% 			\begin{frame}
	% 			\frametitle{Theorem}
	% 			\begin{theorem}[Mass--energy equivalence]
	% 			$E = mc^2$
	% 			\end{theorem}
	% 			\end{frame}
	% 		\end{verbatim}
	% 	\end{example}
	% \end{frame}

\end{document}
