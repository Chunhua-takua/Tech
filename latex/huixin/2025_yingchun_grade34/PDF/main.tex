\documentclass[a4paper]{ctexart}

\usepackage{amsmath}
\usepackage{amssymb}
\usepackage[UTF8, scheme=plain]{ctex}
\usepackage{enumitem}
\usepackage{fancyhdr}
\usepackage{gensymb}
\usepackage[margin=1in]{geometry}
\usepackage{graphicx}  % 用于插入图片
\usepackage{hyperref}
\usepackage{indentfirst} % 首行缩进
\usepackage{lastpage}
\usepackage{ragged2e}
\usepackage{tikz}
\usepackage{upgreek}
\usepackage{verbatim}
\usepackage{xassoccnt}
\usepackage{wasysym} % 允许包含图像
\usepackage{chngcntr}
\usepackage{float}

\counterwithin*{enumi}{section} 

\usetikzlibrary{calc}

\date{} % not display time here

\fancyhf{}
\pagestyle{fancy} %fancyhdr宏包新增的页面风格

% foot decorative lines
\renewcommand{\footrulewidth}{1pt}
\fancyhead[R]{\leftmark}

\newcommand{\myoverline}[1]{%
  \leavevmode % 确保在段落中正常工作
  \rlap{\raisebox{2ex}{\rule{1.5cm}{0.4pt}}}% 上划线(位置0.5ex,粗细0.4pt)
  #1% 文本内容
}

\setmainfont{SimHei}

\begin{document}
    % whether show solutions 
    \newif\ifshowSolution
    % \showSolutiontrue
    \showSolutionfalse

    % 不显示自动编号 且 能自动生成目录
    \setcounter{secnumdepth}{0}
    % 行距
    \renewcommand{\baselinestretch}{1.25}

    \title{\heiti\zihao{2} 2025年迎春杯(小中)数学讲义}
    \maketitle
    \thispagestyle{empty} % 标题页不显示页码

    \vfill
    \centerline{\date{\today}}

    \setlength{\parindent}{2em}
    \newpage

    % 目录 罗马数字页码
    \setcounter{page}{1}
    \pagenumbering{Roman}
    \cfoot{\thepage}

    \hypersetup{colorlinks=true, linktocpage=true}
    \tableofcontents

    % 正文 阿拉伯数字页码
    \newpage
    \setcounter{page}{1}
    \pagenumbering{arabic}
    % 当前页 of 总页数
    \cfoot{\thepage\ / \pageref{LastPage}}

    \zihao{5}

    \begin{sloppy}
        \begin{enumerate}
            \section{第1讲\quad 计算}


\item {
    【乘法分配律】
    $(200+2)\times 5 + 5020$ 
    \ifshowSolution
        \fangsong\zihao{4}
        \\
        思路:改。2025

        正解: 
    \else
        \\ \\ \\
    \fi
}

% \item {
%     $(20+2)\times 5 + 2025$ 
%     \ifshowSolution
%         \fangsong\zihao{4}
%         \\
%         思路:2025

%         正解: 
%     \else
%         \\ \\ \\
%     \fi
% }


\item {
    【乘法分配律】
    $7\times 19 + 3\times 13\times 41 + 13\times 19$
    \ifshowSolution
        \fangsong\zihao{4}
        \\
        思路:改。2021;迎春杯四年级真题.pdf

        正解: 
    \else
        \\ \\ \\
    \fi
}


\item {
    【乘法分配律】
    $(18\times 23 - 24\times 17)\div 3 + 5$
    \ifshowSolution
        \fangsong\zihao{4}
        \\
        思路:

        正解: 
    \else
        \\ \\ \\
    \fi
}

\item {
    【乘法分配律】
    $(11\times 24 - 23\times 9)\div 3 + 3$
    \ifshowSolution
        \fangsong\zihao{4}
        \\
        思路:

        正解: 
    \else
        \\ \\ \\
    \fi
}

% \item {
%     【乘法分配律】
%     $7\times 17 + 3\times 13\times 43 + 13\times 17$
%     \ifshowSolution
%         \fangsong\zihao{4}
%         \\
%         思路:2021;迎春杯四年级真题.pdf

%         正解: 2017
%     \else
%         \\ \\ \\
%     \fi
% }

% \item {
%     $(9\times 8\times 7 + 6 - 5)\times 4 + 3 -2 +1$
%     \ifshowSolution
%         \fangsong\zihao{4}
%         \\
%         思路: 迎春杯四年级2022-试卷.pdf

%         正解: 2022
%     \else
%         \\ \\ \\
%     \fi
% }

\item {
    【乘法凑10】
    $12\times 25 + 16\times 15$
    \ifshowSolution
        \fangsong\zihao{4}
        \\
        思路:

        正解: 
    \else
        \\ \\ \\
    \fi
}

\item {
    【乘法凑10】
    $5\times 432\times 1 - 98 - 7\times 6$
    \ifshowSolution
        \fangsong\zihao{4}
        \\
        思路: 2020数学花园探秘笔试小中决赛D卷.doc

        正解: 2020
    \else
        \\ \\ \\
    \fi
}

\item {
    【加法凑10】
    $1+3+4+6+7+9+10 + 12$
    \ifshowSolution
        \fangsong\zihao{4}
        \\
        思路:

        正解: 
    \else
        \\ \\ \\
    \fi
}

\item {
    【乘法凑10】
    $210\times 6 - 52\times 5$
    \ifshowSolution
        \fangsong\zihao{4}
        \\
        思路:

        正解: 
    \else
        \\ \\ \\
    \fi
}

\item {
    【尾同头合十】
    $5000- 22\times 82$  
    \ifshowSolution
        \fangsong\zihao{4}
        \\
        思路:

        正解: 
    \else
        \\ \\ \\
    \fi
}


\item {
    【头同尾合十】
    $67\times 62 - 34 + 67 + 34$
    \ifshowSolution
        \fangsong\zihao{4}
        \\
        思路:

        正解: 
    \else
        \\ \\ \\
    \fi
}

\item {
    【等差数列求和公式】
    $(1+3+5+\cdots + 89) - (1+2+3+\cdots + 63)$  
    \ifshowSolution
        \fangsong\zihao{4}
        \\
        思路:

        正解: 
    \else
        \\ \\ \\
    \fi
}

\item {
    【立方和公式】
    $3^3 + 4^3 + 5^3 + 6^3 + 7^3 + 8^3 + 9^3$
    \ifshowSolution
        \fangsong\zihao{4}
        \\
        思路:

        正解: 
    \else
        \\ \\ \\
    \fi
}

\item {
    【数值计算;数字谜】有一些自然数,如 121 和 2552,从左到右和从右到左的数字顺序相同,我们把这样的自然数叫做``回文数''. 已知两个回文数的和是 2022,则这两个回文数的差是\underline{\hbox to 20mm{}}.
    \ifshowSolution
        \fangsong\zihao{4}
        \\
        思路: 综合-数字谜.

        正解:  迎春杯三年级2022-试卷.pdf; 1740
    \else
        \\ \\ \\
    \fi
}


\item {
    【数值计算】$99\times 10101\times 111\times 1001001$的末5位数字是多少?
    \ifshowSolution
        \fangsong\zihao{4}
        \\
        思路:

        正解: 88889
    \else
        \\ \\ \\
    \fi
}
            % \section{第2讲\quad 数论}

\item {
    【整除】
    再过12天就到2016年了, 昊昊感慨地说: 我到目前只经过2个闰年, 并且我出生的年份是9的倍数, 那么2016年昊昊是\underline{\hbox to 20mm{}}岁. 
    \ifshowSolution 
        \fangsong\zihao{5}\textcolor{blue}{
            \\正解: \\
            昊昊目前只经过2个闰年,从2015年向前推算两个闰年是2012年和2008年,\\
            所以昊昊出生的年份在 2004 年和 2015 年之间,其中9的倍数的年份是 2007,\\
            所以昊昊在 2007年出生.\\
            2016年,昊昊$2016-2007=9$(岁).
        }
    \else
        \vspace{1cm}
    \fi
}

\item {
    【整除】
    数列 $121, 1221, 12221, 122221,\cdots$ 的前2025项中, 有多少项能被3整除? 
    \ifshowSolution
        \\\fangsong\zihao{5}\textcolor{blue}{
            正解: 675.
        }
    \else
        \vspace{1cm}
    \fi
}

\item {
    【整除】
    各位数字都是 7, 并能被 63 整除的最小自然数是\underline{\hbox to 20mm{}}.
    \ifshowSolution
        \\\fangsong\zihao{5}\textcolor{blue}{
            正解: 777777777.
        }
    \else
        \vspace{1cm}
    \fi
}

\item {
    【余数】
    王老师在一个特殊的学校上课, 他每上3天课可以休息一天, 已知本学期他第一次休息在星期二, 那么他第五次休息是星期\underline{\hbox to 20mm{}} (填数字1--7).
    \ifshowSolution
        \\\fangsong\zihao{5}\textcolor{blue}{
            正解: \\
            工作3天休息1天共4天.\\
            第一次休息是星期2, 第五次休息共经过4个周期.\\
            $4\times 4=16$(天).\\
            16天是经过2个星期再过2天,\\
            所以,第五次休息是星期4.
        }
    \else
        \vspace{1cm}
    \fi
}

\item {
    【数列·余数】今天是1月30日, 我们先写下130;后面写数的规则是: 如果刚写下的数是偶数就把它除以2再加上2写在后面, 如果刚写下的数是奇数就把它乘以2再减去2写在后面, 于是得到: 130、67、132、68..., 那么这列数中第2016个数是\underline{\hbox to 20mm{}}
    \ifshowSolution
        \\\fangsong\zihao{5}\textcolor{blue}{
            正解: 6.\\
            数字规律是 $130,67,132,68,36,20,12,8,6,5,8,6,5,8,6,5\cdots$\\
            去掉前7项是循环周期数列.\\
            $2016-7=2009$.\\
            每3个数字一个循环,\\
            $ 2009\div 3 = 667\cdots 2$\\
            循环数列的第二个数字就是6.
        }
    \else
        \vspace{1cm}
    \fi
}

\item {
    【余数】
    已知 $S = 2^{2020} + 3^{2021} + 4^{2022} + 5^{2023} +6^{2024} + 7^{2025}$, 则 $S$ 的末位数字是多少? 
    \ifshowSolution
        \\\fangsong\zihao{5}\textcolor{blue}{
            正解: 3.
        }
    \else
        \vspace{1cm}
    \fi
}

\item {
    【带余除法】
    一个整数减去 77, 然后乘以 8, 再除以 7, 所得的商是 37, 而且有余数.这个数是多少? 
    \ifshowSolution
        \\\fangsong\zihao{5}\textcolor{blue}{
            正解: 110.
        }
    \else
        \vspace{1cm}
    \fi
    % 2025培训题3年级-答案版.pdf;
}

\item {
    【带余除法】已知被除数比除数大 80, 并且商是 8, 余数是 3, 则被除数与除数之积是 \underline{\hbox to 20mm{}}.
    \ifshowSolution
        \\\fangsong\zihao{5}\textcolor{blue}{
            正解: 1001.
        }
    \else
        \vspace{1cm}
    \fi
}

\item {
    【带余除法】已知 $A$ 是一个两位数, $A^2$ 除以15的余数为1, 则满足条件的 $A$ 的个数为 \underline{\hbox to 20mm{}}.
    \ifshowSolution
        \\\fangsong\zihao{5}\textcolor{blue}{
            正解: 24.
        }
    \else
        \vspace{1cm}
    \fi
    % 华数真题2021-2023(小中组).pdf
}

\item {
    【同余】
    有一个数除以3余 2, 除以4余 1. 此数除以 12 余\underline{\hbox to 20mm{}}.
    \ifshowSolution
        \\\fangsong\zihao{5}\textcolor{blue}{
            正解: 5.
        }
    \else
        \vspace{1cm}
    \fi
}

\item {
    【同余】
    一个三位数被 3 除余 1, 被 5 除余 3, 被 7 除余 5, 这个数最大是\underline{\hbox to 20mm{}}.
    \ifshowSolution
        \\\fangsong\zihao{5}\textcolor{blue}{
            正解: 943.
        }
    \else
        \vspace{1cm}
    \fi
}

\item {
    【位值原理;等差数列】
    一组有两位数组成的偶数项等差数列, 所有奇数项的和为100, 若从第1项开始, 将每个奇数项与它后面相邻的偶数项不改变次序地合并成一个四位数, 形成一个新的数列, 那么新数列的和与原数列的和相差\underline{\hbox to 20mm{}}.
    \ifshowSolution
        \\\fangsong\zihao{5}\textcolor{blue}{
            正解: 9900.\\
            设这个等差数列的奇数项分别为$a_1,a_3,a_5,\cdots$,公差为 $d$,\\
            那么将每个奇数项与后面相邻的偶数项合并,由于每一项都是两位数,所以合并后的四位数列可以表示为\\
            $a_1\times 100 + a_1+d, a_2\times 100 + a_2 + d,\cdots$\\
            所以, 新数列的和与原数列的和相差\\
            $99\times (a_1+a_3+a_5+\cdots)$,\\
            因为奇数项的和为 100, \\
            所以 $99\times (a_1+a_3+a_5+\cdots) = 99\times 100 = 9900$.
        }
    \else
        \vspace{1cm}
    \fi
}

        \end{enumerate}
        \begin{enumerate}
            \section{字典排列法}

\title[第2讲\quad 字典排列法]{第2讲\quad 字典排列法} 
\author{}
\date{}
\begin{frame}
    \titlepage
\end{frame}

\begin{frame}
    \frametitle{课前测}
    \vspace*{-2cm}
    \textit{用一块长8分米,宽4分米的长方形纸板与两块边长4分米的正方形纸板拼成一个正方形.拼成的正方形的周长是多少分米?\\}
    \textit{A.12\\ B.24\\ C.32\\ D.40}
\end{frame}

\begin{frame}
    \frametitle{课前测}
    \vspace*{-2cm}
    \begin{figure}[H] 
        \centering
        \includegraphics[width=1\textwidth]{./pics/Chapter_2/keqian2.png}
    \end{figure}
\end{frame}

\begin{frame}
    \frametitle{课前测}
    \vspace*{-2cm}
    \begin{figure}[H] 
        \centering
        \includegraphics[width=1\textwidth]{./pics/Chapter_2/keqian3.png}
    \end{figure}
\end{frame}

\begin{frame}
    \frametitle{知识梳理}
\end{frame}

\begin{frame}
    \frametitle{MISSION 1}
    \vspace*{-2cm}
    \textit{同学们,在进行字典排列法时,我们应当如何做到不重不漏呢?}
\end{frame}

\begin{frame}
    \frametitle{探索1}
    \vspace*{-2cm}
    \textit{(1)用数字1、2、3可以组成多少个不同的无重复数字的两位数?}
\end{frame}

\begin{frame}
    \frametitle{探索1}
    \vspace*{-2cm}
    \textit{(2)用数字1、3、6可以组成多少个不同的无重复数字的三位数?}
\end{frame}

\begin{frame}
    \frametitle{探索2}
    \vspace*{-2cm}
    \textit{(1)用数字1、2、3可以组成多少个不同的两位数?}
\end{frame}

\begin{frame}
    \frametitle{探索2}
    \vspace*{-2cm}
    \textit{(2)用数字1、3、6可以组成多少个不同的三位数?}
\end{frame}

\begin{frame}
    \frametitle{探索3}
    \vspace*{-2cm}
    \textit{用数字1、2、3可以组成多少个不同的无重复数字的自然数?}
\end{frame}

\begin{frame}
    \frametitle{课堂互动1}
    \begin{figure}[H] 
        \centering
        \includegraphics[width=1\textwidth]{./pics/Chapter_2/ketanghudong1.png}
    \end{figure}
\end{frame}

\begin{frame}
    \frametitle{课堂互动2}
    \begin{figure}[H] 
        \centering
        \includegraphics[width=1\textwidth]{./pics/Chapter_2/ketanghudong2.png}
    \end{figure}
\end{frame}

\begin{frame}
    \frametitle{课堂互动3}
    \begin{figure}[H] 
        \centering
        \includegraphics[width=1\textwidth]{./pics/Chapter_2/ketanghudong3.png}
    \end{figure}
\end{frame}

\begin{frame}
    \frametitle{捉虫时刻}
    \vspace*{-2cm}
    \textit{艾迪这道题的做法对么?如果不对,请你写一下正确的解法。\\
    用数字1、2能组成多少个不同的三位数 ?\\
    由于要组成三位数,因此有一个数字要重复.\\
    一一列举:112、121、211、221、212、122,共有6个.}
\end{frame}

\begin{frame}
    \frametitle{MISSION 2}
    \vspace*{-2cm}
    \textit{在无序问题中,我们应当如何进行枚举?}
\end{frame}

\begin{frame}
    \frametitle{探索4}
    \vspace*{-2cm}
    \textit{艾迪与薇儿做游戏,在分别标有1到10的10个完全一样的小球中,艾迪任意取出2个小球(不计取出顺序),由薇儿计算两球所标的数之和,若和大于10,则薇儿获胜.那么能使薇儿获胜的小球取法共有多少种?}
\end{frame}

\begin{frame}
    \frametitle{探索5}
    \vspace*{-2cm}
    \textit{艾迪去儿童餐厅买15元特惠套餐,他有若干张1元、2元、5元的纸币,但是购买特惠套餐的条件是必须找出一共有多少种不同的付钱方法(要求每种纸币都有),眼看优惠时间就要截止了,同学们你能帮助艾迪顺利买到优惠套餐吗 ?}
\end{frame}

\begin{frame}
    \frametitle{探索6}
    \vspace*{-2cm}
    \textit{在某地有四种不同面值的硬币,如图所示,假若你恰有这四种硬币各1枚。问:共能组成多少种不同的钱数?}
    \begin{figure}[H] 
        \centering
        \includegraphics[width=0.5\textwidth]{./pics/Chapter_2/tansuo6.png}
    \end{figure}
\end{frame}

\begin{frame}
    \frametitle{补充1}
    \textit{薇儿收集到四种不同的面值的硬币各1枚,如图所示,一共可以组成多少种不同的钱数?}
    \begin{figure}[H] 
        \centering
        \includegraphics[width=0.3\textwidth]{./pics/Chapter_2/buchong1_1.png}
    \end{figure}
\end{frame}

\begin{frame}
    \frametitle{补充1}
    \textit{艾迪收到三种不同面值的硬币,如图所示,假若你恰好有以下四枚硬币.问共能组成多少种不同的钱数?}
    \begin{figure}[H] 
        \centering
        \includegraphics[width=0.3\textwidth]{./pics/Chapter_2/buchong1_2.png}
    \end{figure}
\end{frame}

\begin{frame}
    \frametitle{补充1}
    \textit{用四种不同的硬币各1枚,如图所示,两两一组,一共可以组成多少种不同的钱数?}
    \begin{figure}[H] 
        \centering
        \includegraphics[width=0.3\textwidth]{./pics/Chapter_2/buchong1_3.png}
    \end{figure}
\end{frame}

\begin{frame}
    \frametitle{探索7}
    \vspace*{-2cm}
    \textit{博士给艾迪与薇儿上课,课上介绍了``拐弯''的概念.\\
    博士:``对于一行数,如果有三个数abc依次排一起,且$a > b, c > b$或者$a <b,c<b$,我们就称它发生了一次拐弯''\\
    艾迪:``我懂了,比如4321没有拐弯,像1243就发生了一次拐弯''\\
    薇儿:``没错,再比如1324就发生了两次拐弯.''\\
    博士:``非常棒!看来你们都掌握得非常扎实了,现在我要考考你们了,如果我们将1,2,3,4排成一行,则能使这行数刚好发生两次拐弯的排列方法共有多少种?''}
\end{frame}

\begin{frame}
    \frametitle{探索8}
    \vspace*{-2cm}
    \textit{一次,齐王与田忌赛马,每人各有等级不同的4匹马,这8匹马按照从快到慢的排序分别是齐王的一等马,田忌的一等马,齐王的二等马,田忌的二等马,齐王的三等马,田忌的三等马,齐王的四等马,田忌的四等马.田忌已经提前知道齐王本次赛马的出场顺序是一等、二等、三等、四等,他可以安排种不同的出场顺序,才能保证自己至少战平齐王呢?同学们,你们能帮田忌找出所有可能的决策吗 ?}
\end{frame}

\begin{frame}
    \frametitle{补充2}
    \textit{加加与减减做游戏,两人轮流在一张白纸上写出一个数字,组成一个多位数的前2位,而这个多位数从第三个数字开始,每个数字都恰好是它前面两个数字之和,直至不能再写为止.例如加加减减写了1和4,那么这个多位数就是1459,则这类多位数共有多少个?}
\end{frame}

\begin{frame}
    \frametitle{补充2}
    \textit{如果一个数的各位数字从左到右构成等差数串,我们就称这个数为“跳跃数”,例如:1358642均是“跳跃数”,153就不是“跳跃数”,那么一共有多少个三位“跳跃数”?}
\end{frame}

\begin{frame}
    \frametitle{思维导图}
    \begin{figure}[H] 
        \centering
        \includegraphics[width=1\textwidth]{./pics/Chapter_2/siweidaotu.png}
    \end{figure}
\end{frame}
        \end{enumerate}
        \begin{enumerate}
            \section{第3讲\quad 行程问题与应用题}

\item {
    【追及】
    猎豹跑一步长为2米,狐狸跑一步长为1米. 猎豹跑2步的时间狐狸跑3步. 猎豹距离狐狸30米,则猎豹跑动\underline{\hbox to 20mm{}}米可追上狐狸. 
    \ifshowSolution 
        \fangsong\zihao{5}\textcolor{blue}{
            \\正解: 120m.\\
                不妨假设猎豹1秒跑2步,那么狐狸1秒跑3步;\\
                那么,猎豹的速度是 \[2\times 2 = 4m/s,\] 
                狐狸的速度是 
                \[1\times 3 = 3 m/s.\]
                \[距离\div 速度差 = 追及所需时间\]
                \[30\div (4-3) = 30(s). \]
                \[30 \times 4 = 120 (m).\]
        }
    \else
        \vspace{2cm}
    \fi
    % 2017年第二十二届“华罗庚金杯”少年数学邀请赛初赛试卷(小中组).doc, 120
}

\item {
    【相遇】
    一辆公共汽车和一辆小轿车同时从相距450千米的两地相向而行,公共汽车每小时行40千米,小轿车每小时行50千米,\underline{\hbox to 20mm{}}小时后两车第二次相距90千米. 
    \ifshowSolution 
        \fangsong\zihao{5}\textcolor{blue}{
            \\正解: \\
                第2次相遇时,两车一共走 \[450+90=540 km;\]
                \[540\div (40+50) = 6h.\]
        }
    \else
        \vspace{2cm}
    \fi
    % 2020数学花园探秘笔试小中决赛D卷.doc; 6
}

\item {
    【相遇】
    里山镇到省城的高速路全长189千米,途径县城. 县城离里山镇54千米. 早上8: 30一辆客车从里山镇开往县城,9: 15到达. 停留15分钟后开往省城,午前11: 00能够到达. 另有一辆客车于当日早上9: 00从省城径直开往里山镇. 每小时行驶60千米. 两车相遇时,省城开往里山镇的客车行驶了\underline{\hbox to 20mm{}}分钟.
    \ifshowSolution 
        \fangsong\zihao{5}\textcolor{blue}{
            \\正解: \\
            \begin{figure}[H] 
                \centering
                \includegraphics[width=0.6\textwidth]{./pics/Chapter_3/seikai_1.png}
            \end{figure}
                从里山开往省城的客车,9:30从县城出发,县城到省城段速度:
                    \[11:00 - 9:30 = 1.5 h\]
                    \[(189-54)\div 1.5=90 km/h\]
                9:00到9:30, 从省城出发的客车行驶了0.5h,行驶距离:
                    \[60\times 0.5 = 30km\]
                所以,在9:30两车相距
                    \[135 - 30 = 105 km\]
                两车相遇还需要
                \[105\div (90+60) = 0.7h,\]
                此时,从省城出发的客车行驶了
                \[0.5 + 0.7 = 1.2(h) = 72(min).\]
        }
    \else
        \vspace{2cm}
    \fi
    % 2012年第十七届“华罗庚金杯”少年数学邀请赛网上初赛试卷(小学中低年级组).doc, 72
}

\item {
    【多次追及】
    哥哥和弟弟两人同时从家出发去 2000米外的学校上学,哥哥每分钟走 60米,弟弟每分钟走 50 米,走了 10 分钟后,哥哥发现忘记带数学错题本,就以每分钟 100 米的速度跑回家,回到家后,哥哥用了2分钟找到了错题本,然后以每分钟 150 米的速度往学校跑.从哥哥第二次从家出发开始计算,经过\underline{\hbox to 20mm{}}分钟后,哥哥能追上弟弟.
    \ifshowSolution 
        \fangsong\zihao{5}\textcolor{blue}{
            \\正解: \\
            \begin{figure}[H] 
                \centering
                \includegraphics[width=0.7\textwidth]{./pics/Chapter_3/seikai_2.png}
            \end{figure}
                从哥哥第一次从家里出发,到再次出发,经过的时间如下.\\
                $t_1=10 min$: 第一次出发到发现忘带本子;此时哥哥走了
                \[60\times 10 = 600 (m).\]
                $t_2=600\div 100 = 6(min)$: 从发现忘带本子到跑回家.\\
                $t_3=2(min)$: 到家找到本子第二次从家里出发.\\
                在以上时间内,弟弟都以50m/min 的速度向学校走去,共走了
                \[ 50\times (10+6+2) = 900 m.\]
                哥哥第二次从家里出发,需要
                \[900\div (150 - 50) = 9 min \]
                追上弟弟.
        }
    \else
        \vspace{2cm}
    \fi
    % 2020华数之星初赛-三四年级真题.pdf; 9
}

\item {
    【多次相遇】
    甲、乙两人在一条长120米的直路上来回跑,甲的速度是5米/秒,乙的速度是3米/秒,若他们同时从同一端出发跑了15分钟,则他们在这段时间内共迎面相遇\underline{\hbox to 20mm{}}次(端点除外). 
    \ifshowSolution 
        \fangsong\zihao{5}\textcolor{blue}{
            \\正解: \\
                甲乙两人迎面相遇时,2人一共的行程是2个单程:
                $$120\times 2=240(米)$$
                用时为
                $$240\div (3+5)=30(秒)$$
                即每30秒就相遇一次(包括端点相遇).
                端点相遇用时为:
                甲单程用时:
                 $$120\div 3 = 40 (s),$$
                乙单程用时:
                $$120\div 5=24 (s)$$
                两人迎面相遇的时间为40和24的公倍数,最小公倍数是120.
                $$120\div 30=4$$
                可知,他们4次相遇中就有1次为端点相遇,
                即15分钟内相遇的总次数为:
                 $$15\times 60\div 30 = 30(次),$$
                其中在端点相遇的次数为 $30\div 4$ 的整数部分即7.
                所以他们在这段时间内共迎面相遇(端点除外)的次数为:
                $$30-7=23(次).$$
        }
    \else
        \vspace{2cm}
    \fi
    % 2015华, 23
}

\item {
    【多次相遇】
    甲、乙两车分别从A,B两地同时出发,相向匀速行进,在距A地 60 千米处相遇. 相遇后,两车继续行进,分别到达B,A后, 立即原路返回, 在距B地50 千米处再次相遇. 则A,B两地的路程是\underline{\hbox to 20mm{}}千米. 
    \ifshowSolution 
        \fangsong\zihao{5}\textcolor{blue}{
            \\正解: \\
            相遇时距离之比等于速度之比.\\
            设AB两地间的距离为S km.\\
            第一次相遇时,甲走了 60千米,而乙走了S-60 千米; \\
            第二次相遇,甲又走了S-60+50千米,乙又走了60+S-50千米.\\
            \begin{align*}
                \frac{60}{S-60} &= \frac{S-60+50}{60+S-50} \\
                S&=130.
            \end{align*}
        }
    \else
        \vspace{2cm}
    \fi
    % 2016华, 130
}

\item {
    【多次相遇】
    甲、乙两车分别从A,B两地同时出发,相向而行,3小时后相遇,甲掉头返回A地,乙继续前行. 甲到达A地后掉头往B行驶,半小时后和乙相遇. 那么乙从A到B共需\underline{\hbox to 20mm{}}小时.
    \ifshowSolution 
        \fangsong\zihao{5}\textcolor{blue}{
            \\正解:\\ 
                相遇后,甲还需要3小时返回甲地,\\
                第二次相遇时,甲距离第一次相遇点的距离等于甲 2.5 小时的路程,乙用了3.5小时走这些路程,所以甲乙速度的比是7:5,\\
                甲乙相遇需要3小时,那么乙单独到需要
                \[ 3\times \frac{7+5}{5} = 7.2小时.\]
        }
    \else
        \vspace{2cm}
    \fi
    % 2011年第十六届“华罗庚金杯”少年数学邀请赛初赛试卷(小学组).doc, 7.2
}

\item {
    【行程与时间计算】
    小文今天和朋友约定一起看 12:00 开场的电影,出门时,发现挂钟电池没电已经停止了,她把挂钟换好电池,但没来得及调整时间,出门前挂钟显示的时间是 9:25,小文赶到电影院时,电影刚好开场.电影结束后,小文立刻返回家中,发现挂钟显示的时间是 13:55,小文赶紧把它调成正确的时间15:45.如果小文从家到电影院和从电影院返回家中花的时间是一样的,那么,电影的时长是\underline{\hbox to 20mm{}}分钟.
    \ifshowSolution{}
        \fangsong\zihao{5}\textcolor{blue}{
            \\正解:
            \begin{figure}[H] 
                \centering
                \includegraphics[width=0.7\textwidth]{./pics/Chapter_3/seikai_8.png}
            \end{figure}
            钟从 9:25到13:55(真实时间为15:45),用时 
            \[ 13:55 - 9:25 = 4h 30 min. \]
            真实时间从12:00 到 15:45, 用时
            \[ 15:45 - 12:00 = 3h 45 min. \]
            钟9:25到真实12:00, 用时
            \[
                4h 30min - 3h 45 min = 45min
            \]
            所以,电影时长:
            \[3h 45min - 45min = 3h = 180min. \]
        }
    \else
        \vspace{2cm}
    \fi
    % 迎春杯四年级2022-试卷.pdf; 180
}

\item {
    【流水行船·追及·相遇】
    一条河上有A,B两个码头,A在上游,B在下游. 甲、乙两人分别从A,B同时出发,划船相向而行,4小时后相遇. 如果甲、乙两人分别从A,B同时出发,划船同向而行,乙16小时后追上甲. 已知甲在静水中划船的速度为每小时6千米,则乙在静水中划船每小时行驶\underline{\hbox to 20mm{}}千米.
    \ifshowSolution{}
        \fangsong\zihao{5}\textcolor{blue}{
            \\正解:\\
            两船相遇的速度即两船的速度和, 两船追及速度即两船的速度差.\\
            相向而行两船所行的路程是 A、B两个码头之间的距离; \\
            同向而行两船的距离差也为 A、B两个码头之间的距离. \\
            设乙船的速度是x千米/小时, 列出方程 
            \[(x+6)\times 4=(x-6)\times 16 \]
            \[ x=10. \]
        }
    \else
        \vspace{2cm}
    \fi
    % 2015华, 10
}

        \end{enumerate}
        \begin{enumerate}
            \section{乘法竖式}

\title[第4讲\quad 乘法竖式]{第4讲\quad 乘法竖式} 
\author{}
\date{}
\begin{frame}
    \titlepage
\end{frame}

\begin{frame}
    \frametitle{课前测}
    \textit{已知有如图竖式谜: $\ding{73} =$\underline{\hbox to 10mm{}},$\triangle =$\underline{\hbox to 10mm{}}.\\
    A. 3, 2\qquad 
    B. 6, 1\qquad 
    C. 2, 1\qquad
    D. 2, 4}
    \begin{figure}[H] 
        \centering
        \includegraphics[width=0.4\textwidth]{./pics/Chapter_4/keqian1.png}
    \end{figure}
    % B
\end{frame}

\begin{frame}
    \frametitle{课前测}
    \textit{请问下列竖式中第二个加数是多少?\\
    A. 68\qquad 
    B. 78\qquad 
    C. 88\qquad
    D. 98}
    \begin{figure}[H] 
        \centering
        \includegraphics[width=0.4\textwidth]{./pics/Chapter_4/keqian2.png}
    \end{figure}
    % B
\end{frame}

\begin{frame}
    \frametitle{课前测}
    \parbox{0.9\textwidth}{在下图的加法算式中,每个字母代表一个数字,不同的字母代表不同的数字,那么EFFC代表的四位数是\underline{\hbox to 10mm{}}.\\
    A. 1006\qquad 
    B. 1007\qquad 
    C. 1008\qquad
    D. 1009}
    \begin{figure}[H] 
        \centering
        \includegraphics[width=0.4\textwidth]{./pics/Chapter_4/keqian3.png}
    \end{figure}
    % D
\end{frame}

\begin{frame}
    \frametitle{知识梳理}
\end{frame}

\begin{frame}
    \frametitle{MISSION 1}
    \vspace*{-2cm}
    \textit{在列乘法竖式的时候,有哪些地方需要注意?\\}
    \textit{1. 数位对齐\\
        2. 个位算起\\
        3. 满十进一}
\end{frame}

\begin{frame}
    \frametitle{探索1}
    \vspace*{-2cm}
    \textit{(1) $56\times 9 = $\\
        (2) $314\times 2 = $\\
        (3) $732\times 4 = $\\
        (4) $284\times 6 = $\\
        (5) $1234\times 7 = $\\}
\end{frame}

\begin{frame}
    \frametitle{探索2}
    \vspace*{-2cm}
    \textit{(1) $506\times 3 = $\\
        (2) $4003\times 7 = $\\
        (3) $5036\times 8 = $\\
        (4) $4320\times 8 = $}
\end{frame}

\begin{frame}
    \frametitle{探索3}
    \vspace*{-2cm}
    \textit{(1) $55\times 29 = $\\
        (2) $18\times 63 = $\\
        (3) $123\times 45 = $\\
        (4) $721\times 28 = $}
\end{frame}

\begin{frame}
    \frametitle{捉虫时刻}
    \textit{请你给企鹅治病,并开出处方.}
    \begin{figure}[H] 
        \centering
        \includegraphics[width=0.3\textwidth]{./pics/Chapter_4/zhuochong.png}
    \end{figure}
\end{frame}

\begin{frame}
    \frametitle{探索4}
    \vspace*{-2cm}
    \textit{(1) $150\times 270 = $\\
        (2) $3600\times 250 = $\\
        (3) $2030\times 140 = $\\
        (4) $102\times 3040 = $}
\end{frame}

\begin{frame}
    \frametitle{探索5}
    \vspace*{-2cm}
    \textit{(1) $38\times 32 + 37\times 33 + 36\times 34$}
\end{frame}

\begin{frame}
    \frametitle{探索5}
    \vspace*{-2cm}
    \textit{(2) $12\times 92 + 22\times 82 + 32\times 72$}
\end{frame}

\begin{frame}
    \frametitle{MISSION 2}
    \vspace*{-2cm}
    \textit{同学们,你能说说加、减、乘、除这四则运算的优先级么 ?}
\end{frame}

\begin{frame}
    \frametitle{探索6}
    \vspace*{-2cm}
    \textit{双面龟的壳上有一副神秘地图,巨整指出这是标记了三片龙鳞位置的山海图.图中显示金鳞距离出发点130米,火鳞距离出发点320米,水鳞距离出发点250米,小鲤鱼和它的小伙伴们一共6人,2人一组出发寻找宝藏,请问各组到达目的地时全组人一共走的路程是多少米?}
    % 1200
\end{frame}

\begin{frame}
    \frametitle{探索7}
    \vspace*{-2cm}
    \textit{艾迪去文具店帮老师买文具,文具的单价如下:自动铅笔4元一支,文具盒26元一个,钢笔45元一支,书包128元一个,艾迪要买32支铅笔,24个文具盒,65支钢笔,9个书包,请问艾迪要花多少钱 ?}
    % 4829
\end{frame}

\begin{frame}
    \frametitle{探索8}
    \vspace*{-2cm}
    \textit{鲤鱼湖中突然出现了一个奇怪的植物一刺刺球,由于没有天敌,刺刺球开始大量扩散.于是全村的居民出动,分为4组开始清理刺刺球.已知第1组有47人,每人每天可以清理21个刺刺球,第2组有94人,每人每天可以清理18个刺刺球,第3组有57人,每人每天可以清理53个刺刺球,第4组不仅没有清理好刺刺球,反而由于操作失误,使得刺刺球每天增加100个,请问全村一天一共可以使刺刺球减少多少个?}
    % 5600
\end{frame}

\begin{frame}
    \frametitle{思维导图}
    \begin{figure}[H] 
        \centering
        \includegraphics[width=1\textwidth]{./pics/Chapter_4/siweidaotu.png}
    \end{figure}
\end{frame}
        \end{enumerate}
        \begin{enumerate}
            \section{第4讲\quad 计数问题}

\item {
    下图是由 ``开罗五边形'' 组成的拼接图,图中的每个五边形的形状大小完全相同,观察图形并确定下右图中的图形在下左图中共出现了\underline{\hbox to 10mm{}}次. (右图图形可以旋转观察)
    \begin{figure}[H] 
        \centering
        \includegraphics[width=0.8\textwidth]{./pics/Chapter_6/8.png}
    \end{figure}
    % 迎春杯三年级2022-试卷.pdf. ; 12
}

\item {丽丽想用大小为 $1\times 1, 2\times 2, 3\times 3$的三种正方形拼成下图所示的领奖台(图中每个小正方形的边长为1),所用正方形的面积总和为 15,且拼接过程中不可重叠,每种正方形数量不限(可以不用).共有\underline{\hbox to 10mm{}}种不同的拼接方法.(正方形的摆放位置或数量不同都算不同的拼法)
    \begin{figure}[H] 
        \centering
        \includegraphics[width=0.6\textwidth]{./pics/Chapter_6/9.png}
    \end{figure}
    % 迎春杯三年级2022-试卷.pdf ; 15
}

\item {老师手中有4张牌,按照甲、乙、甲、乙的顺序分发。如果这4张牌的点数分别是 1、2、3、4,并且在整个过程中(包括最终),甲手中牌的点数之和一直比乙大,那么,满足要求的分发顺序共有\underline{\hbox to 10mm{}}种.
    % 迎春杯四年级2022-试卷.pdf ; 7
}

\item {右图中,共有\underline{\hbox to 10mm{}} 个三角形.
    \begin{figure}[H] 
        \centering
        \includegraphics[width=0.5\textwidth]{./pics/Chapter_6/12.png}
    \end{figure}
    % 2021;迎春杯三年级真题.pdf ;16
}

\item {如右图,小鱼老师在为圣诞树准备装饰物,每个树顶需要放一颗幸运星每一层树的两侧需要各放1个许原球,一共3层,小鱼老师数了数,许愿球比幸运星多 40个,那么,小鱼老师装饰了 \underline{\hbox to 10mm{}} 棵圣诞树.
    \begin{figure}[H] 
        \centering
        \includegraphics[width=0.3\textwidth]{./pics/Chapter_6/13.png}
    \end{figure}
    % 2021;迎春杯三年级真题.pdf ; 8
}

\item {
    右图中,共有 \underline{\hbox to 10mm{}}个正六边形.
    \begin{figure}[H] 
        \centering
        \includegraphics[width=0.4\textwidth]{./pics/Chapter_6/14.png}
    \end{figure}
    % 2021 ; 迎春杯四年级真题.pdf ; 12
}

\item {
    如图,一个$2\times 2\times 2$的正方体六个面已经被染成了不同的六种颜色.现将其分成4个 的小长方体,共有\underline{\hbox to 10mm{}}种不同分法.
    \begin{figure}[H] 
        \centering
        \includegraphics[width=0.4\textwidth]{./pics/Chapter_6/15.png}
    \end{figure}
    % 2020数学花园探秘笔试小中决赛D卷.doc ;9
}

\item {
    如图在$5\times 5$的方格中放置了编号为$1\sim 5$的5个小球,没有任何两个小球在同一行或同一列;如果同时移动其中3个小球到相邻格子(有公共点的格子)里,移动完后依然没有任何两个小球在同一行或同一列,那么共有\underline{\hbox to 20mm{}}种移动的方法.
    \begin{figure}[H] 
        \centering
        \includegraphics[width=0.4\textwidth]{./pics/Chapter_6/23.png}
    \end{figure}
    % 2020数学花园探秘笔试小中决赛D卷.doc ;13
}


\item {
    如图中共有\underline{\hbox to 20mm{}}个平行四边形.
    \begin{figure}[H] 
        \centering
        \includegraphics[width=0.4\textwidth]{./pics/Chapter_6/2017_1.png}
    \end{figure}
    % 2017年“迎春杯”数学花园探秘科普活动试卷(小中组决赛a卷).doc ;13
}

\item {
    如图中共有\underline{\hbox to 20mm{}}个梯形.
    \begin{figure}[H] 
        \centering
        \includegraphics[width=0.4\textwidth]{./pics/Chapter_6/2016_1.png}
    \end{figure}
    % 2016年“迎春杯”数学花园探秘网试试卷(四年级).doc ;12
}

\item {
    图\textcircled{3}是由6个图\textcircled{1}这样的模块拼成的,如果最底层已经给定两块的位置(如图\textcircled{2}),那么剩下部分一共有\underline{\hbox to 20mm{}}种不同的拼法.
    \begin{figure}[H] 
        \centering
        \includegraphics[width=0.4\textwidth]{./pics/Chapter_6/2016_2.png}
    \end{figure}
    % 2016年“迎春杯”数学花园探秘决赛试卷(小中组c卷).doc ;2
}

\item {
    有一颗神奇的树上长了58个果子,第一天会有1个果子从树上掉落.从第二天起,每天掉落的果子数量比前一天多1个,但如果某天树上的果子数量少于这一天本应该掉落的数量时,那么这一天它又重新从掉落1颗果子开始.按原规律进行新的一轮.如此继续.那么第\underline{\hbox to 20mm{}}天树上的果子会掉光.
    % \begin{figure}[H] 
    %     \centering
    %     \includegraphics[width=0.4\textwidth]{./pics/Chapter_6/2016_2.png}
    % \end{figure}
    % 2016年“迎春杯”数学花园探秘初赛试卷(四年级d卷).doc ;12
}


\item {
    用4种不同颜色给圆圈涂色(4种颜色可以不全用).要求有线直接相连的两个圆圈的颜色不同.则共有\underline{\hbox to 20mm{}}种不同的涂色方法.
    \begin{figure}[H] 
        \centering
        \includegraphics[width=0.4\textwidth]{./pics/Chapter_6/2016_3.png}
    \end{figure}
    % 2016年“迎春杯”数学花园探秘初赛试卷(四年级d卷).doc ;756
}

\item {
    如图,甲、乙两人从A沿最短路线走到B,两人所走路线不出现交叉(除A、B两点外没有其它公共点)的走法共有\underline{\hbox to 20mm{}}种.
    \begin{figure}[H] 
        \centering
        \includegraphics[width=0.4\textwidth]{./pics/Chapter_6/2016_4.png}
    \end{figure}
    % 2016年“迎春杯”数学花园探秘初赛试卷(四年级a卷).doc ;38
}

\item {
    在平面上用长度为6厘米的牙签棒摆正方形,摆出一个长为6厘米的正方形需要4根牙签棒,摆出5个这样的正方形至少需要\underline{\hbox to 20mm{}}根牙签棒.
    % 2016年“迎春杯”数学花园探秘初赛试卷(三年级d卷).doc ;15
}

\item {
    植物射手有豌豆射手,双重射手、三重射手、寒冰射手、双向射手、豌豆荚6种.种植一株该种射手所需要的阳光依次为100、200、300、150、125、125.菲菲种了10株植物射手共花费2500阳光,她的种法有\underline{\hbox to 20mm{}}种不同的可能.(例如,7株三重射手+1株寒冰射手+1株豌豆荚,是符合要求的一种可能.)
    % 2015年“迎春杯”数学花园探秘网试试卷(四年级).doc ;8
}

\item {
    如图中共能数出\underline{\hbox to 20mm{}}个三角形.
    \begin{figure}[H] 
        \centering
        \includegraphics[width=0.4\textwidth]{./pics/Chapter_6/2015_1.png}
    \end{figure}
    % 2015年“迎春杯”数学花园探秘科普活动试卷(小中组决赛c卷).doc ;11
}


\item {
    数一数,如图中共有\underline{\hbox to 20mm{}}个三角形.
    \begin{figure}[H] 
        \centering
        \includegraphics[width=0.4\textwidth]{./pics/Chapter_6/2015_2.png}
    \end{figure}
    % 2015年“迎春杯”数学花园探秘科普活动试卷(四年级初赛b卷).doc ;8
}

\item {
    如图所示,从正三角形的边作一个正方形,再用与正三角形不相邻的正方形一边做一个正五边形,再从与正方形不相邻的正五边形一边作一个正六边形,继续以相同的方式再作一个正七边形,依序再作一个正八边形,这样形成了一个多边形,请问这个多边形有\underline{\hbox to 20mm{}}个边.
    \begin{figure}[H] 
        \centering
        \includegraphics[width=0.4\textwidth]{./pics/Chapter_6/2015_3.png}
    \end{figure}
    % 2015年“迎春杯”数学花园探秘科普活动试卷(三年级初赛b卷).doc ;23
}


\item {
    如图所示,一个圆形托盘上放着三个相同的盘子,笑笑只将7个相同的苹果放在这一个盘子中,每个盘子中至少要放一个.那么笑笑有\underline{\hbox to 20mm{}}种放苹果的方法.(托盘旋转后相同的算同一种情况)
    \begin{figure}[H] 
        \centering
        \includegraphics[width=0.4\textwidth]{./pics/Chapter_6/2015_4.png}
    \end{figure}
    % 2015年“迎春杯”数学花园探秘科普活动试卷(三年级初赛a卷).doc ;23
}

%%%%%%%%%%%%%%%%%%%%%%%%%%%%%
% \item {如图所示,墙上挂着6件礼物,分成两列,每列必须从下往上依次取走。现在有6位同学,身高分别为 100、120、140、160、180、200厘米.每人只能拿到最多比自己身高高 100 厘米处的礼物.现在6位同学排成一列拿取礼物.为了让每人都能拿到一件礼物,有\underline{\hbox to 20mm{}}种符合要求的排队方式.
%     \begin{figure}[H] 
%         \centering
%         \includegraphics[width=0.4\textwidth]{./pics/Chapter_6/3.png}
%     \end{figure}
%     % 2025数学花园探秘笔试小中年级决赛C卷(解析版).pdf;40
% }

% \item {如图,2x3的棋盘上由6个单位正方形构成,棋盘上共有12 个格点,甲、乙、丙、丁11.
% 四人站在其中四个格点上,若任意两人处于同一横线或竖线则认为可相互看见。
% 甲说:我可以看见你们所有人。
% 乙说:我一个人也看不到。
% 丙说:我只能看到一个人。
% 丁说:我们四人所在格点连成四边形面积为3,且任意3人所在格点组成的三角形面积都不为整数。
% 已知四人中看见人数最多(其他人看见的人数都比他看见的少)的那个人说了假话,其他人都说了真话,那么这4人有\underline{\hbox to 20mm{}}种不同的站法.
%     \begin{figure}[H] 
%         \centering
%         \includegraphics[width=0.4\textwidth]{./pics/Chapter_6/1.png}
%     \end{figure}
%     % 2025数学花园探秘笔试小中年级决赛C卷(解析版).pdf ;8
% }

% \item {将1、2、3、4、5、6、7、8这八个数字分别填到一个固定好的正方体的八个顶点上,要求同一条棱上的两个数之和小于12,那么共有\underline{\hbox to 20mm{}}种不同的填法.
%     % 2024数学花园探秘笔试小中年级夏季决赛C卷(B5试卷版).pdf ; 
% }

% \item {右图已固定,请将1、2、3、4各两个分别填入八个圆圈中,使得阴影圆圈中的数比它两边相邻的白色圆圈中的数都大;那么不同的填法共有\underline{\hbox to 20mm{}}种.
%     \begin{figure}[H] 
%         \centering
%         \includegraphics[width=0.4\textwidth]{./pics/Chapter_6/6.png}
%     \end{figure}
%     % 2023YCB初赛真题答案小中.pdf ;44
% }

% \item {右图中共能数出\underline{\hbox to 20mm{}}个 三角形.
%     \begin{figure}[H] 
%         \centering
%         \includegraphics[width=0.4\textwidth]{./pics/Chapter_6/7.png}
%     \end{figure}
%     % 2023;YCB第40届小中组试卷答案.pdf ;8
% }


% \item {图中,三角形共有\underline{\hbox to 20mm{}}个.
%     \begin{figure}[H] 
%         \centering
%         \includegraphics[width=0.4\textwidth]{./pics/Chapter_6/5.png}
%     \end{figure}
%     % 2023YCB初赛真题答案小中.pdf ; 6
% }


% \item {如图,在正方体的一些顶点处各有一只蚂蚁,它们的爬行速度相同且只能沿着正方体的棱爬行,在棱上爬行到达下一个顶点前不能回头,且从未有任两只蚂蚁相遇.\\
%     (1)如果在 A、B 处各有一只蚂蚁,每只蚂蚁各爬行1条棱,共有\underline{\hbox to 10mm{}}种不同的爬行情况.\\
%     (2)如果在 A、C、F 处各有一只蚂蚁,每只蚂蚁各爬行2条棱, 
%     一共有 \underline{\hbox to 10mm{}} 种不同的爬行情况.
%     \begin{figure}[H] 
%         \centering
%         \includegraphics[width=0.2\textwidth]{./pics/Chapter_6/22.png}
%     \end{figure}
%     % 2023.YCB第40届小中组试卷答案.pdf. 8;121
% }
        \end{enumerate}
        \begin{enumerate}
            \section{幂的运算}

\begin{comment}
\item {
    当 $ x=7, y=-\frac{1}{7}$ 时, 求 $x^{4n+1}\cdot y^{4n+2}$ ($n$为整数)的值.
    \ifshowSolution
        \fangsong\zihao{4}
        \\
        思路: 观察到$xy=-1$,让将$x$和$y$凑成一对,相乘.直接将$x, y$的值代入表达式中进行计算.

        解答: 
        \begin{align*}
            \mbox{原式} &= 7^{4n+1}\cdot \left(-\frac{1}{7}\right) ^{4n+2}\\
            &= [7\times(-\frac{1}{7})]^{4n+1} \cdot(-\frac{1}{7})\\
            &= (-1)^{4n+1} \cdot(-\frac{1}{7})\\
            &= \frac{1}{7}.
        \end{align*}
    \else
        \\ \\ \\
    \fi
}
\end{comment}

\begin{comment}
\item {
    已知$ m=8^9, n=9^8 $, 用含$m, n$的式子表示 $72^{72}$.
    \ifshowSolution
        \fangsong\zihao{4}
        \\
        思路: 观察到$72=8\times 9$,将原式中的72分解,凑出$m, n$.
        
        解答: 
        \begin{align*}
            \mbox{原式} &= (8\times 9)^{72}\\
            &= 8^{72}\times 9^{72}\\
            &= (8^9)^8\times (9^8)^9\\
            &= m^8 n^9.
        \end{align*}
    \else
        \\ \\ \\
    \fi
}
\end{comment}

\begin{comment}
\item {
    已知$x-y=k$, 求$(3x-3y)^3.$
    \ifshowSolution
        \fangsong\zihao{4}
        \\
        解答: 
        \begin{align*}
            \mbox{原式} &= [3(x-y)]^3\\
            &= 27(x-y)^3\\
            &= 27k^3.
        \end{align*}
    \else
        \\ \\ \\
    \fi
}
\end{comment}

\begin{comment}
\item {
    若$(a^nb^mb)^3 = a^9 b^{15}$, 求$2^{m+n}$.
    \ifshowSolution
        \fangsong\zihao{4}
        \\
        思路: 先将左边化简,再与右边比较,解出$m,n$.
        
        解答: 
        \begin{align*}
            (a^nb^{m+1})^3 &= a^9b^{15}\\
            a^{3n}b^{3m+3} &= a^9b^{15}
        \end{align*}
        $\therefore 3n=9, 3m+3=15$\\
        $\therefore n=3, m=4$
        \begin{align*}
            2^{m+n} &= 2^7\\
            &= 128.
        \end{align*}
    \else
        \\ \\ \\
    \fi
}
\end{comment}

\begin{comment}
    \item {
        化简: $(-a-b)^{2n}$ ($n$为整数).
    }
    \\ \\ \\
    \item {
        化简: $(-a-b)^{2n+1}$ ($n$为整数).
    }
    \\ \\ \\

    \item {
        (用科学计数法表示)已知 1 nm = 0.000000001 m, 则 15 nm 等于多少 m?
        \ifshowSolution
        \fangsong\zihao{4}
        \\
        解答: 

        \textcircled{1} 写出换算关系
        \begin{align*}
            1 \rm{nm} &= 10^{-9} \rm{m}
        \end{align*}
        \textcircled{2} 两边同时乘15
        \begin{align*}
            15 \rm{nm} &= 15 \times 10^{-9} \rm{m}\\
            &= 1.5\times 10^{-8} \rm{m}.
        \end{align*}
        \fi
    }
    \\ \\ \\

    \item {
        (用科学计数法表示)肥皂泡表面厚度大约是 0.0007 mm,换算成以米为单位是多少?
    \ifshowSolution
    \fangsong\zihao{4}
    \\
    解答: 

    \textcircled{1} 写出换算关系
    \begin{align*}
        1 \rm{mm} &= 10^{-3} \rm{m}
    \end{align*}
    \textcircled{2} 两边同时乘0.0007
    \begin{align*}
        0.0007 \rm{mm} &= 0.0007 \times 10^{-3} \rm{m}\\
        &= 7\times 10^{-7} \rm{m}.
    \end{align*}
    \fi
    }
    \\ \\ \\
\end{comment}

\begin{comment}
\item {
    (用科学计数法表示)已知 $0.25 \upmu$m $ = 2.5\times 10^{-7}$m,那么 1 m 等于多少$\upmu$m?
    \ifshowSolution
        \fangsong\zihao{4}
        \\
        思路: 将题中给出的换算关系两边同时除以 $2.5\times 10^{-7}$,右边就出现了 1m.

        解答: 
        \begin{align*}
            \frac{0.25}{2.5\times 10^{-7}} \rm{\upmu m} &= 1\rm{m}\\
            \frac{2.5\times 0.1}{2.5\times \frac{1}{10^7}} \rm{\upmu m} &= 1\rm{m}\\
            0.1\times 10^7 \rm{\upmu m} &= 1\rm{m}\\
            10^6 \rm{\upmu m} &= 1\rm{m}\\
            1\rm{m} &= 10^6 \rm{\upmu m}.
        \end{align*}
    \else
        \\ \\ \\
    \fi
}
\end{comment}

\begin{comment}
    \item {
        若多项式$ 9x^2 - mx+16$是一个完全平方式,则 $m$的值是多少?
    }
    \\ \\ \\
\end{comment}

\begin{comment}
\item {
    已知$a^2+b^2=8, a-b=3$,求$ab$的值.
    \ifshowSolution
        \fangsong\zihao{4}
        \\
        思路: 看到$a^2+b^2, a-b, ab$, 应该想到完全平方公式.
    \else
        \\ \\ \\
    \fi
}
\end{comment}

\begin{comment}
    \item {
        若$x^2+mx+9$是完全平方式,求常数$m$的值.
    }
    \\ \\ \\

    \item {
        若$x+y=2$,求代数式$x^2-y^2+4y$的值.
    }
    \\ \\ \\
\end{comment}



\begin{comment}
\item {
    (注意: 除号使用分数形式) 已知$10^{-m}=a, 10^{-n}=b$($m, n$是整数),求$10^{2m-3n}$的值(用含有$a, b$的代数式表示).
    \\ \\ \\
}
\end{comment}

\begin{comment}
\item {
    已知$2^x=3, 2^y=6, 2^z=12$,判断下列有关$x, y, z$的数量关系式的对错.\\
    (1) $x+z=2y$\\
    (2) $x+y+3=2z$\\
    (3) $4x=z$\\
    (4) $x+1=y$
    \\ \\
}
\end{comment}

\item {
    计算: $ (\frac{1}{2})^{-1} + \lvert 2-\pi \rvert $
    \ifshowSolution
    \fangsong\zihao{4}
    \\
    思路: 去绝对值符号,运算到底.

    解答: 
    \begin{align*}
        \mbox{原式} &= 2 + \pi - 2\\
        &= \pi.
    \end{align*}
    \else
        \\ \\ \\
    \fi
}

\begin{comment}
\item {
    已知$(x+2)^{x+5}=1$, 求$x$.
    \\ \\ \\
}
\end{comment}

\item {
    $\bigstar$
    (用科学记数法)一个正方体集装箱的棱长为 $0.8 \rm{m}$.\\
    % (1) 这个集装箱的体积是多少?(用科学记数法)\\
    (2) 若有一个小立方块的棱长为$2\times 10^{-3} $ m, 则需要多少个这样的小立方块才能将集装箱装满?
    \ifshowSolution
    \fangsong\zihao{4}
    \\
    思路: 问题(2)注意简便运算.

    解答:
    \begin{align*}
        \frac{0.8^3} {(2\times 10^{-3})^3} &= \frac{0.8^3} {8\times 10^{-9}}\\
        &= \frac{0.064} {\frac{1}{10^9}}\\
        &= 0.064\times 10^9\\
        &= 6.4\times 10^{-2}\times 10^9\\
        &= 6.4\times 10^7\\
    \end{align*}
    \else
        \\ \\ \\
    \fi
}

\begin{comment}
    \item {
        (把 $\frac{1}{27}$ 化为以3为底的幂) 若$3^{x-1}=\frac{1}{27}$, 求$x$.
        \\ \\ \\
    }
\end{comment}
    
\begin{comment}
\item {
    (注意: 除号使用分数形式) 已知$a^{2n}=3, a^{3m}=5$, 求$a^{6n-9m}$.
    \ifshowSolution
    \fangsong\zihao{4}
    \\
    思路: 将$a^{6n-9m}$凑出$a^{2n}, a^{3m}$, 直接代入计算. 结果使用分数形式即可.

    解答: 
    \begin{align*}
        a^{6n-9m} &= \frac{a^{6n}}{a^{9m}}\\
        &= \frac{(a^{2n})^3} {(a^{3m})^3}\\
        &= \frac{3^3} {5^3}\\
        &= \frac{27} {125}.
    \end{align*}
    \else
        \\ \\ \\
    \fi
}
\end{comment}

\begin{comment}
    \item {
        已知$3\cdot2^x + 2^{x+1}=40$, 求$x$.
        \ifshowSolution
        \fangsong\zihao{4}
        \\
        思路: 将左边的2个$2^{x}$整理到一起.
    
        解答: 
        \begin{align*}
            3\cdot2^x + 2^{x+1} &= 40\\
            3\cdot2^x + 2\cdot 2^{x} &= 40\\
            5\cdot2^x &= 40\\
            2^x &= 8\\
            \therefore x = 3.
        \end{align*}
        \else
            \\ \\ \\
        \fi
    }
\end{comment}
        \end{enumerate}
    \end{sloppy}
\end{document}
