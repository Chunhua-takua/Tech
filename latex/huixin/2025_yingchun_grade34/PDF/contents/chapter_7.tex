\section{第7讲\quad 综合}


\item {
    【逻辑推理】
    四百米比赛进入冲刺阶段,甲在乙前面30米,丙在丁后面60米,乙在丙前面20米. 这时,跑在前面的两位同学相差 \underline{\hbox to 20mm{}} 米.
    \vspace{2cm}
    % 2012华, 10
}

\item {
    【综合·推理】现在有一台奇怪的电脑,电脑上有个按键,如果电脑上原来的数是3的倍数,按下键后就会除以3;如果电脑上原来的数不是3的倍数,那么按下键后就会乘以6.小明在按键前没有看屏幕上的数,结果连按6次,最后电脑上显示的数是12,那么电脑上最开始的数最小可能是\underline{\hbox to 20mm{}}.
    \ifshowSolution
        \fangsong\zihao{4}
        \\
        思路:

        正解: 27
    \else
        \\ \\ \\
    \fi
}

\item {
    【综合·同余·周期】
    一条圆形跑道长 600 米,因铺设水管,其中跑道上 AB 一段被挖开,形成一个大坑. AB的跑道长度为 150 米.  有一机器人放在跑道上循环行走, 前进的步长(跑道弧长)为 d米,可调整步长 d的大小,但调后不再改变,并且 d小于 600 米.请设计出两种(d 的不同长度)方案,使得机器人不断循环,并且永远不会落入坑里(碰到 A或 B也算落入坑里)每种方案包括:\\
    (1)步长d的值(不同方案的d的值). \\
    (2)机器人的出发点.
    \vspace{2cm}
    % 方案一:作一弧长 300米,该弧包含 AB,(4,B不在弧的端点上).机器人从该段弧的端点出发,d=300
    % 方案二:作一弧长 200 米,该弧包含 AB,(A,B不在弧的端点上).机器人从该段弧的端点出发,d=200
    % 方案三:作一弧长 400米,该弧的一半部分包含 AB,(4,B不在弧的端点与中点上).机器人从该段弧的端点出发,d=200.
    % 2021“华数之星”复评(初级)参考答案及评阅标准.pdf
}
