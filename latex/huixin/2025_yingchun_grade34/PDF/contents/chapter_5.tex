\section{第4讲\quad 计数问题}

\item {
    下图是由 ``开罗五边形'' 组成的拼接图,图中的每个五边形的形状大小完全相同,观察图形并确定下右图中的图形在下左图中共出现了\underline{\hbox to 10mm{}}次. (右图图形可以旋转观察)
    \begin{figure}[H] 
        \centering
        \includegraphics[width=0.8\textwidth]{./pics/Chapter_6/8.png}
    \end{figure}
    % 迎春杯三年级2022-试卷.pdf. ; 12
}

\item {丽丽想用大小为 $1\times 1, 2\times 2, 3\times 3$的三种正方形拼成下图所示的领奖台(图中每个小正方形的边长为1),所用正方形的面积总和为 15,且拼接过程中不可重叠,每种正方形数量不限(可以不用).共有\underline{\hbox to 10mm{}}种不同的拼接方法.(正方形的摆放位置或数量不同都算不同的拼法)
    \begin{figure}[H] 
        \centering
        \includegraphics[width=0.6\textwidth]{./pics/Chapter_6/9.png}
    \end{figure}
    % 迎春杯三年级2022-试卷.pdf ; 15
}

\item {老师手中有4张牌,按照甲、乙、甲、乙的顺序分发。如果这4张牌的点数分别是 1、2、3、4,并且在整个过程中(包括最终),甲手中牌的点数之和一直比乙大,那么,满足要求的分发顺序共有\underline{\hbox to 10mm{}}种.
    % 迎春杯四年级2022-试卷.pdf ; 7
}

\item {右图中,共有\underline{\hbox to 10mm{}} 个三角形.
    \begin{figure}[H] 
        \centering
        \includegraphics[width=0.5\textwidth]{./pics/Chapter_6/12.png}
    \end{figure}
    % 2021;迎春杯三年级真题.pdf ;16
}

\item {如右图,小鱼老师在为圣诞树准备装饰物,每个树顶需要放一颗幸运星每一层树的两侧需要各放1个许原球,一共3层,小鱼老师数了数,许愿球比幸运星多 40个,那么,小鱼老师装饰了 \underline{\hbox to 10mm{}} 棵圣诞树.
    \begin{figure}[H] 
        \centering
        \includegraphics[width=0.3\textwidth]{./pics/Chapter_6/13.png}
    \end{figure}
    % 2021;迎春杯三年级真题.pdf ; 8
}

\item {
    右图中,共有 \underline{\hbox to 10mm{}}个正六边形.
    \begin{figure}[H] 
        \centering
        \includegraphics[width=0.4\textwidth]{./pics/Chapter_6/14.png}
    \end{figure}
    % 2021 ; 迎春杯四年级真题.pdf ; 12
}

\item {
    如图,一个$2\times 2\times 2$的正方体六个面已经被染成了不同的六种颜色.现将其分成4个 的小长方体,共有\underline{\hbox to 10mm{}}种不同分法.
    \begin{figure}[H] 
        \centering
        \includegraphics[width=0.4\textwidth]{./pics/Chapter_6/15.png}
    \end{figure}
    % 2020数学花园探秘笔试小中决赛D卷.doc ;9
}

\item {
    如图在$5\times 5$的方格中放置了编号为$1\sim 5$的5个小球,没有任何两个小球在同一行或同一列;如果同时移动其中3个小球到相邻格子(有公共点的格子)里,移动完后依然没有任何两个小球在同一行或同一列,那么共有\underline{\hbox to 20mm{}}种移动的方法.
    \begin{figure}[H] 
        \centering
        \includegraphics[width=0.4\textwidth]{./pics/Chapter_6/23.png}
    \end{figure}
    % 2020数学花园探秘笔试小中决赛D卷.doc ;13
}


\item {
    如图中共有\underline{\hbox to 20mm{}}个平行四边形.
    \begin{figure}[H] 
        \centering
        \includegraphics[width=0.4\textwidth]{./pics/Chapter_6/2017_1.png}
    \end{figure}
    % 2017年“迎春杯”数学花园探秘科普活动试卷(小中组决赛a卷).doc ;13
}

\item {
    如图中共有\underline{\hbox to 20mm{}}个梯形.
    \begin{figure}[H] 
        \centering
        \includegraphics[width=0.4\textwidth]{./pics/Chapter_6/2016_1.png}
    \end{figure}
    % 2016年“迎春杯”数学花园探秘网试试卷(四年级).doc ;12
}

\item {
    图\textcircled{3}是由6个图\textcircled{1}这样的模块拼成的,如果最底层已经给定两块的位置(如图\textcircled{2}),那么剩下部分一共有\underline{\hbox to 20mm{}}种不同的拼法.
    \begin{figure}[H] 
        \centering
        \includegraphics[width=0.4\textwidth]{./pics/Chapter_6/2016_2.png}
    \end{figure}
    % 2016年“迎春杯”数学花园探秘决赛试卷(小中组c卷).doc ;2
}

\item {
    有一颗神奇的树上长了58个果子,第一天会有1个果子从树上掉落.从第二天起,每天掉落的果子数量比前一天多1个,但如果某天树上的果子数量少于这一天本应该掉落的数量时,那么这一天它又重新从掉落1颗果子开始.按原规律进行新的一轮.如此继续.那么第\underline{\hbox to 20mm{}}天树上的果子会掉光.
    % \begin{figure}[H] 
    %     \centering
    %     \includegraphics[width=0.4\textwidth]{./pics/Chapter_6/2016_2.png}
    % \end{figure}
    % 2016年“迎春杯”数学花园探秘初赛试卷(四年级d卷).doc ;12
}


\item {
    用4种不同颜色给圆圈涂色(4种颜色可以不全用).要求有线直接相连的两个圆圈的颜色不同.则共有\underline{\hbox to 20mm{}}种不同的涂色方法.
    \begin{figure}[H] 
        \centering
        \includegraphics[width=0.4\textwidth]{./pics/Chapter_6/2016_3.png}
    \end{figure}
    % 2016年“迎春杯”数学花园探秘初赛试卷(四年级d卷).doc ;756
}

\item {
    如图,甲、乙两人从A沿最短路线走到B,两人所走路线不出现交叉(除A、B两点外没有其它公共点)的走法共有\underline{\hbox to 20mm{}}种.
    \begin{figure}[H] 
        \centering
        \includegraphics[width=0.4\textwidth]{./pics/Chapter_6/2016_4.png}
    \end{figure}
    % 2016年“迎春杯”数学花园探秘初赛试卷(四年级a卷).doc ;38
}

\item {
    在平面上用长度为6厘米的牙签棒摆正方形,摆出一个长为6厘米的正方形需要4根牙签棒,摆出5个这样的正方形至少需要\underline{\hbox to 20mm{}}根牙签棒.
    % 2016年“迎春杯”数学花园探秘初赛试卷(三年级d卷).doc ;15
}

\item {
    植物射手有豌豆射手,双重射手、三重射手、寒冰射手、双向射手、豌豆荚6种.种植一株该种射手所需要的阳光依次为100、200、300、150、125、125.菲菲种了10株植物射手共花费2500阳光,她的种法有\underline{\hbox to 20mm{}}种不同的可能.(例如,7株三重射手+1株寒冰射手+1株豌豆荚,是符合要求的一种可能.)
    % 2015年“迎春杯”数学花园探秘网试试卷(四年级).doc ;8
}

\item {
    如图中共能数出\underline{\hbox to 20mm{}}个三角形.
    \begin{figure}[H] 
        \centering
        \includegraphics[width=0.4\textwidth]{./pics/Chapter_6/2015_1.png}
    \end{figure}
    % 2015年“迎春杯”数学花园探秘科普活动试卷(小中组决赛c卷).doc ;11
}


\item {
    数一数,如图中共有\underline{\hbox to 20mm{}}个三角形.
    \begin{figure}[H] 
        \centering
        \includegraphics[width=0.4\textwidth]{./pics/Chapter_6/2015_2.png}
    \end{figure}
    % 2015年“迎春杯”数学花园探秘科普活动试卷(四年级初赛b卷).doc ;8
}

\item {
    如图所示,从正三角形的边作一个正方形,再用与正三角形不相邻的正方形一边做一个正五边形,再从与正方形不相邻的正五边形一边作一个正六边形,继续以相同的方式再作一个正七边形,依序再作一个正八边形,这样形成了一个多边形,请问这个多边形有\underline{\hbox to 20mm{}}个边.
    \begin{figure}[H] 
        \centering
        \includegraphics[width=0.4\textwidth]{./pics/Chapter_6/2015_3.png}
    \end{figure}
    % 2015年“迎春杯”数学花园探秘科普活动试卷(三年级初赛b卷).doc ;23
}


\item {
    如图所示,一个圆形托盘上放着三个相同的盘子,笑笑只将7个相同的苹果放在这一个盘子中,每个盘子中至少要放一个.那么笑笑有\underline{\hbox to 20mm{}}种放苹果的方法.(托盘旋转后相同的算同一种情况)
    \begin{figure}[H] 
        \centering
        \includegraphics[width=0.4\textwidth]{./pics/Chapter_6/2015_4.png}
    \end{figure}
    % 2015年“迎春杯”数学花园探秘科普活动试卷(三年级初赛a卷).doc ;23
}

%%%%%%%%%%%%%%%%%%%%%%%%%%%%%
% \item {如图所示,墙上挂着6件礼物,分成两列,每列必须从下往上依次取走。现在有6位同学,身高分别为 100、120、140、160、180、200厘米.每人只能拿到最多比自己身高高 100 厘米处的礼物.现在6位同学排成一列拿取礼物.为了让每人都能拿到一件礼物,有\underline{\hbox to 20mm{}}种符合要求的排队方式.
%     \begin{figure}[H] 
%         \centering
%         \includegraphics[width=0.4\textwidth]{./pics/Chapter_6/3.png}
%     \end{figure}
%     % 2025数学花园探秘笔试小中年级决赛C卷(解析版).pdf;40
% }

% \item {如图,2x3的棋盘上由6个单位正方形构成,棋盘上共有12 个格点,甲、乙、丙、丁11.
% 四人站在其中四个格点上,若任意两人处于同一横线或竖线则认为可相互看见。
% 甲说:我可以看见你们所有人。
% 乙说:我一个人也看不到。
% 丙说:我只能看到一个人。
% 丁说:我们四人所在格点连成四边形面积为3,且任意3人所在格点组成的三角形面积都不为整数。
% 已知四人中看见人数最多(其他人看见的人数都比他看见的少)的那个人说了假话,其他人都说了真话,那么这4人有\underline{\hbox to 20mm{}}种不同的站法.
%     \begin{figure}[H] 
%         \centering
%         \includegraphics[width=0.4\textwidth]{./pics/Chapter_6/1.png}
%     \end{figure}
%     % 2025数学花园探秘笔试小中年级决赛C卷(解析版).pdf ;8
% }

% \item {将1、2、3、4、5、6、7、8这八个数字分别填到一个固定好的正方体的八个顶点上,要求同一条棱上的两个数之和小于12,那么共有\underline{\hbox to 20mm{}}种不同的填法.
%     % 2024数学花园探秘笔试小中年级夏季决赛C卷(B5试卷版).pdf ; 
% }

% \item {右图已固定,请将1、2、3、4各两个分别填入八个圆圈中,使得阴影圆圈中的数比它两边相邻的白色圆圈中的数都大;那么不同的填法共有\underline{\hbox to 20mm{}}种.
%     \begin{figure}[H] 
%         \centering
%         \includegraphics[width=0.4\textwidth]{./pics/Chapter_6/6.png}
%     \end{figure}
%     % 2023YCB初赛真题答案小中.pdf ;44
% }

% \item {右图中共能数出\underline{\hbox to 20mm{}}个 三角形.
%     \begin{figure}[H] 
%         \centering
%         \includegraphics[width=0.4\textwidth]{./pics/Chapter_6/7.png}
%     \end{figure}
%     % 2023;YCB第40届小中组试卷答案.pdf ;8
% }


% \item {图中,三角形共有\underline{\hbox to 20mm{}}个.
%     \begin{figure}[H] 
%         \centering
%         \includegraphics[width=0.4\textwidth]{./pics/Chapter_6/5.png}
%     \end{figure}
%     % 2023YCB初赛真题答案小中.pdf ; 6
% }


% \item {如图,在正方体的一些顶点处各有一只蚂蚁,它们的爬行速度相同且只能沿着正方体的棱爬行,在棱上爬行到达下一个顶点前不能回头,且从未有任两只蚂蚁相遇.\\
%     (1)如果在 A、B 处各有一只蚂蚁,每只蚂蚁各爬行1条棱,共有\underline{\hbox to 10mm{}}种不同的爬行情况.\\
%     (2)如果在 A、C、F 处各有一只蚂蚁,每只蚂蚁各爬行2条棱, 
%     一共有 \underline{\hbox to 10mm{}} 种不同的爬行情况.
%     \begin{figure}[H] 
%         \centering
%         \includegraphics[width=0.2\textwidth]{./pics/Chapter_6/22.png}
%     \end{figure}
%     % 2023.YCB第40届小中组试卷答案.pdf. 8;121
% }