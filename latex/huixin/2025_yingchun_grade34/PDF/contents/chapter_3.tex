\section{第3讲\quad 行程问题与应用题}

\item {
    【追及】
    猎豹跑一步长为2米,狐狸跑一步长为1米. 猎豹跑2步的时间狐狸跑3步. 猎豹距离狐狸30米,则猎豹跑动\underline{\hbox to 20mm{}}米可追上狐狸. 
    \vspace{2cm}
    % 2017华, 120
}
\item {
    【相遇】
    一辆公共汽车和一辆小轿车同时从相距450千米的两地相向而行,公共汽车每小时行40千米,小轿车每小时行50千米,\underline{\hbox to 20mm{}}小时后两车第二次相距90千米. 
    \vspace{2cm}
    % 2020数学花园探秘笔试小中决赛D卷.doc; 6
}

\item {
    【相遇】
    里山镇到省城的高速路全长189千米,途径县城. 县城离里山镇54千米. 早上8: 30一辆客车从里山镇开往县城,9: 15到达. 停留15分钟后开往省城,午前11: 00能够到达. 另有一辆客车于当日早上9: 00从省城径直开往里山镇. 每小时行驶60千米. 两车相遇时,省城开往里山镇的客车行驶了\underline{\hbox to 20mm{}}分钟.
    \vspace{2cm}
    % 2012华, 72
}


\item {
    【多次追及】
    哥哥和弟弟两人同时从家出发去 2000米外的学校上学,哥哥每分钟走 60米,弟弟每分钟走 50 米,走了 10 分钟后,哥哥发现忘记带数学错题本,就以每分钟 100 米的速度跑回家,回到家后,哥哥用了2分钟找到了错题本,然后以每分钟 150 米的速度往学校跑.从哥哥第二次从家出发开始计算,经过\underline{\hbox to 20mm{}}分钟后,哥哥能追上弟弟.
    \vspace{2cm}
    % 2020华数之星初赛-三四年级真题.pdf; 9
}

\item {
    【多次相遇】
    甲、乙两人在一条长120米的直路上来回跑,甲的速度是5米/秒,乙的速度是3米/秒,若他们同时从同一端出发跑了15分钟,则他们在这段时间内共迎面相遇 \underline{\hbox to 20mm{}} 次(端点除外). 
    \vspace{2cm}
    % 2015华, 23
}

\item {
    【多次相遇】
    甲、乙两车分别从A,B两地同时出发,相向匀速行进,在距A地 60 千米处相遇. 相遇后,两车继续行进,分别到达B,A后,立即原路返回,在距B地50 千米处再次相遇. 则A,B两地的路程是\underline{\hbox to 20mm{}}千米. 
    \vspace{2cm}
    % 2016华, 130
}

\item {
    【多次相遇】
    甲、乙两车分别从A,B两地同时出发,相向而行,3小时后相遇,甲掉头返回A地,乙继续前行. 甲到达A地后掉头往B行驶,半小时后和乙相遇. 那么乙从A到B共需\underline{\hbox to 20mm{}}小时.
    \vspace{2cm}
    % 2011华, 7.2
}



\item {
    【行程与时间计算】
    小文今天和朋友约定一起看 12:00 开场的电影,出门时,发现挂钟电池没电已经停止了,她把挂钟换好电池,但没来得及调整时间,出门前挂钟显示的时间是 9:25,小文赶到电影院时,电影刚好开场.电影结束后,小文立刻返回家中,发现挂钟显示的时间是 13:55,小文赶紧把它调成正确的时间15:45.如果小文从家到电影院和从电影院返回家中花的时间是一样的,那么,电影的时长是\underline{\hbox to 20mm{}}分钟.
    \vspace{2cm}
    % 迎春杯四年级2022-试卷.pdf; 180
}

\item {
    【流水行船·追及·相遇】
    一条河上有A,B两个码头,A在上游,B在下游. 甲、乙两人分别从A,B同时出发,划船相向而行,4小时后相遇. 如果甲、乙两人分别从A,B同时出发,划船同向而行,乙16小时后追上甲. 已知甲在静水中划船的速度为每小时6千米,则乙在静水中划船每小时行驶\underline{\hbox to 20mm{}}千米.
    \vspace{2cm}
    % 2015华, 10
}
