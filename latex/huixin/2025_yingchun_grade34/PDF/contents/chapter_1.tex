\section{第1讲\quad 计算}


\item {
    【乘法分配律】
    $(200+2)\times 5 + 5020$ 
    \ifshowSolution
        \fangsong\zihao{4}
        \\
        思路:改。2025

        正解: 
    \else
        \\ \\ \\
    \fi
}

% \item {
%     $(20+2)\times 5 + 2025$ 
%     \ifshowSolution
%         \fangsong\zihao{4}
%         \\
%         思路:2025

%         正解: 
%     \else
%         \\ \\ \\
%     \fi
% }


\item {
    【乘法分配律】
    $7\times 19 + 3\times 13\times 41 + 13\times 19$
    \ifshowSolution
        \fangsong\zihao{4}
        \\
        思路:改。2021;迎春杯四年级真题.pdf

        正解: 
    \else
        \\ \\ \\
    \fi
}


\item {
    【乘法分配律】
    $(18\times 23 - 24\times 17)\div 3 + 5$
    \ifshowSolution
        \fangsong\zihao{4}
        \\
        思路:

        正解: 
    \else
        \\ \\ \\
    \fi
}

\item {
    【乘法分配律】
    $(11\times 24 - 23\times 9)\div 3 + 3$
    \ifshowSolution
        \fangsong\zihao{4}
        \\
        思路:

        正解: 
    \else
        \\ \\ \\
    \fi
}

% \item {
%     【乘法分配律】
%     $7\times 17 + 3\times 13\times 43 + 13\times 17$
%     \ifshowSolution
%         \fangsong\zihao{4}
%         \\
%         思路:2021;迎春杯四年级真题.pdf

%         正解: 2017
%     \else
%         \\ \\ \\
%     \fi
% }

% \item {
%     $(9\times 8\times 7 + 6 - 5)\times 4 + 3 -2 +1$
%     \ifshowSolution
%         \fangsong\zihao{4}
%         \\
%         思路: 迎春杯四年级2022-试卷.pdf

%         正解: 2022
%     \else
%         \\ \\ \\
%     \fi
% }

\item {
    【乘法凑10】
    $12\times 25 + 16\times 15$
    \ifshowSolution
        \fangsong\zihao{4}
        \\
        思路:

        正解: 
    \else
        \\ \\ \\
    \fi
}

\item {
    【乘法凑10】
    $5\times 432\times 1 - 98 - 7\times 6$
    \ifshowSolution
        \fangsong\zihao{4}
        \\
        思路: 2020数学花园探秘笔试小中决赛D卷.doc

        正解: 2020
    \else
        \\ \\ \\
    \fi
}

\item {
    【加法凑10】
    $1+3+4+6+7+9+10 + 12$
    \ifshowSolution
        \fangsong\zihao{4}
        \\
        思路:

        正解: 
    \else
        \\ \\ \\
    \fi
}

\item {
    【乘法凑10】
    $210\times 6 - 52\times 5$
    \ifshowSolution
        \fangsong\zihao{4}
        \\
        思路:

        正解: 
    \else
        \\ \\ \\
    \fi
}

\item {
    【尾同头合十】
    $5000- 22\times 82$  
    \ifshowSolution
        \fangsong\zihao{4}
        \\
        思路:

        正解: 
    \else
        \\ \\ \\
    \fi
}


\item {
    【头同尾合十】
    $67\times 62 - 34 + 67 + 34$
    \ifshowSolution
        \fangsong\zihao{4}
        \\
        思路:

        正解: 
    \else
        \\ \\ \\
    \fi
}

\item {
    【等差数列求和公式】
    $(1+3+5+\cdots + 89) - (1+2+3+\cdots + 63)$  
    \ifshowSolution
        \fangsong\zihao{4}
        \\
        思路:

        正解: 
    \else
        \\ \\ \\
    \fi
}

\item {
    【立方和公式】
    $3^3 + 4^3 + 5^3 + 6^3 + 7^3 + 8^3 + 9^3$
    \ifshowSolution
        \fangsong\zihao{4}
        \\
        思路:

        正解: 
    \else
        \\ \\ \\
    \fi
}

\item {
    【数值计算;数字谜】有一些自然数,如 121 和 2552,从左到右和从右到左的数字顺序相同,我们把这样的自然数叫做``回文数''. 已知两个回文数的和是 2022,则这两个回文数的差是\underline{\hbox to 20mm{}}.
    \ifshowSolution
        \fangsong\zihao{4}
        \\
        思路: 综合-数字谜.

        正解:  迎春杯三年级2022-试卷.pdf; 1740
    \else
        \\ \\ \\
    \fi
}


\item {
    【数值计算】$99\times 10101\times 111\times 1001001$的末5位数字是多少?
    \ifshowSolution
        \fangsong\zihao{4}
        \\
        思路:

        正解: 88889
    \else
        \\ \\ \\
    \fi
}