\section{数论}

\item {
    再过12天就到2016年了,昊昊感慨地说:我到目前只经过2个闰年,并且我出生的年份是9的倍数,那么2016年昊昊是 \underline{\hbox to 20mm{}} 岁. 
    \ifshowSolution
        \fangsong\zihao{4}
        \\
        思路:

        正解: 9 
    \else
        \\ \\ \\
    \fi
}

\item {
    一个整数减去 77,然后乘以 8,再除以 7,所得的商是 37,而且有余数。这个数是多少?
    \ifshowSolution
        \fangsong\zihao{4}
        \\
        思路:

        正解: 2025培训题3年级-答案版.pdf; 110
    \else
        \\ \\ \\
    \fi
}

\item {
    王老师在一个特殊的学校上课,他每上3天课可以休息一天,已知本学期他第一次休息在星期二,那么他第五次休息是星期\underline{\hbox to 20mm{}} (填数字1--7).
    \ifshowSolution
        \fangsong\zihao{4}
        \\
        思路:

        正解: 4
    \else
        \\ \\ \\
    \fi
}

\item {
    今天是1月30日,我们先写下130;后面写数的规则是:如果刚写下的数是偶数就把它除以2再加上2写在后面,如果刚写下的数是奇数就把它乘以2再减去2写在后面,于是得到:130、67、132、68…,那么这列数中第2016个数是\underline{\hbox to 20mm{}}
    \ifshowSolution
        \fangsong\zihao{4}
        \\
        思路:

        正解:  6
    \else
        \\ \\ \\
    \fi
}

\item {
    一组有两位数组成的偶数项等差数列,所有奇数项的和为100,若从第1项开始,将每个奇数项与它后面相邻的偶数项不改变次序地合并成一个四位数,形成一个新的数列,那么新数列的和与原数列的和相差\underline{\hbox to 20mm{}}.
    \ifshowSolution
        \fangsong\zihao{4}
        \\
        思路:

        正解: 9900
    \else
        \\ \\ \\
    \fi
}

\item {
    现在有一台奇怪的电脑,电脑上有个按键,如果电脑上原来的数是3的倍数,按下键后就会除以3;如果电脑上原来的数不是3的倍数,那么按下键后就会乘以6.小明在按键前没有看屏幕上的数,结果连按6次,最后电脑上显示的数是12,那么电脑上最开始的数最小可能是\underline{\hbox to 20mm{}}.
    \ifshowSolution
        \fangsong\zihao{4}
        \\
        思路:

        正解: 27
    \else
        \\ \\ \\
    \fi
}
\item {
    有一些自然数,如 121 和 2552,从左到右和从右到左的数字顺序相同,我们把这样的自然数叫做``回文数''. 已知两个回文数的和是 2022,则这两个回文数的差是\underline{\hbox to 20mm{}}.
    \ifshowSolution
        \fangsong\zihao{4}
        \\
        思路: 综合-数字谜.

        正解:  迎春杯三年级2022-试卷.pdf; 1740
    \else
        \\ \\ \\
    \fi
}

\item {
    $99\times 10101\times 111\times 1001001$的末5位数字是多少?
    \ifshowSolution
        \fangsong\zihao{4}
        \\
        思路:

        正解: 88889
    \else
        \\ \\ \\
    \fi
}

\item {
    已知 $S = 2^{2020} + 3^{2021} + 4^{2022} + 5^{2023} +6^{2024} + 7^{2025}$, 则 $S$ 的末位数字是多少?
    \ifshowSolution
        \fangsong\zihao{4}
        \\
        思路:

        正解: 3
    \else
        \\ \\ \\
    \fi
}

\item {
    数列 $121, 1221, 12221, 122221,\cdots$ 的前2025项中,有多少项能被3整除?
    \ifshowSolution
        \fangsong\zihao{4}
        \\
        思路:

        正解: 675
    \else
        \\ \\ \\
    \fi
}

\item {
    有一个数除以3余 2,除以4余 1. 此数除以 12 余\underline{\hbox to 20mm{}}.
    \ifshowSolution
        \fangsong\zihao{4}
        \\
        思路:

        正解: 5
    \else
        \\ \\ \\
    \fi
}

\item {
    各位数字都是 7,并能被 63 整除的最小自然数是\underline{\hbox to 20mm{}}.
    \ifshowSolution
        \fangsong\zihao{4}
        \\
        思路:

        正解: 777777777
    \else
        \\ \\ \\
    \fi
}

\item {
    一个三位数被 3 除余 1,被 5 除余 3,被 7 除余 5,这个数最大是\underline{\hbox to 20mm{}}.
    \ifshowSolution
        \fangsong\zihao{4}
        \\
        思路:

        正解: 943
    \else
        \\ \\ \\
    \fi
}

\item {
    一个整数减去 77,然后乘以 8,再除以 7,所得的商是 37,而且有余数。这个数是 \underline{\hbox to 20mm{}}.
    \ifshowSolution
        \fangsong\zihao{4}
        \\
        思路:

        正解: 110
    \else
        \\ \\ \\
    \fi
}

\item {
    已知被除数比除数大 80,并且商是 8,余数是 3,则被除数与除数之积是 \underline{\hbox to 20mm{}}.
    \ifshowSolution
        \fangsong\zihao{4}
        \\
        思路:

        正解: 1001
    \else
        \\ \\ \\
    \fi
}

\item {
    已知 $A$ 是一个两位数,$A^2$ 除以15的余数为1,则满足条件的 $A$ 的个数为 \underline{\hbox to 20mm{}}.
    \ifshowSolution
        \fangsong\zihao{4}
        \\
        思路:

        正解: 华数真题2021-2023(小中组).pdf;24
    \else
        \\ \\ \\
    \fi
}
