\section{幂的运算}

\item {
    $ (-s)^7\div (-s)^4 $ 
    \ifshowSolution
        \fangsong\zihao{4}
        \\
        思路:

        解答: 
    \else
        \\ \\ \\
    \fi
}

\item {
    $ (-\frac13)^4\div (-\frac13) $ 
    \ifshowSolution
        \fangsong\zihao{4}
        \\
        思路:

        解答: 
    \else
        \\ \\ \\
    \fi
}

\item {
    $ (\frac23)^{-3} $ 
    \ifshowSolution
        \fangsong\zihao{4}
        \\
        思路:

        解答: 
    \else
        \\ \\ \\
    \fi
}

\item {
    $ (-\frac12)\div (-2)^{3} \times (-2){-2} $ 
    \ifshowSolution
        \fangsong\zihao{4}
        \\
        思路:

        解答: 
    \else
        \\ \\ \\
    \fi
}

\item {
    $ -(-2x^3y)^4 $ 
    \ifshowSolution
        \fangsong\zihao{4}
        \\
        思路:

        解答: 
    \else
        \\ \\ \\
    \fi
}


\item {
    $ (\frac{1}{10})^5 \times (0.1)^7$
    \ifshowSolution
        \fangsong\zihao{4}
        \\
        思路:

        解答: 
    \else
        \\ \\ \\
    \fi
}

\item {
    计算机存储单位一般用B, KB, MB, GB, TB,...表示,它们之间的关系:$1 KB = 2^{10} B, 1 MB = 2^{10} KB, 1 GB = 2^{10} MB, 1 TB = 2^{10} GB.$ 1 TB 的硬盘容量等于多少B?
    \ifshowSolution
        \fangsong\zihao{4}
        \\
        思路:

        解答: 
    \else
        \\ \\ \\
    \fi
}

\item {
    $ (\frac{4}{3})^{6}\div (-\frac{4}{3})^{3} $
    \ifshowSolution
        \fangsong\zihao{4}
        \\
        思路:

        解答: 
    \else
        \\ \\ \\
    \fi
}

\item {
    $ (-a)^{4}\div (-a) $
    \ifshowSolution
        \fangsong\zihao{4}
        \\
        思路:

        解答: 
    \else
        \\ \\ \\
    \fi
}

\item {
    $ (-xy)^{5}\div (xy)^{2} $
    \ifshowSolution
        \fangsong\zihao{4}
        \\
        思路:

        解答: 
    \else
        \\ \\ \\
    \fi
}

\item {
    用科学计算法表示:$ \frac{17}{10000} $
    \ifshowSolution
        \fangsong\zihao{4}
        \\
        思路:

        解答: 
    \else
        \\ \\ \\
    \fi
}

\item {
    用科学计算法表示:$ \frac{57}{1000} $
    \ifshowSolution
        \fangsong\zihao{4}
        \\
        思路:

        解答: 
    \else
        \\ \\ \\
    \fi
}

\item {
    用科学计算法表示下列结果。
    
    (1) $ 0.0005 cm$ 换算成以米为单位。\\
    (2) $ 7 nm $ 换算成以米为单位。\\
    (3) 已知 $1 \mu m = 10^{-3}mm$. $ 2.5 \mu m$ 换算成以米为单位。
    \ifshowSolution
        \fangsong\zihao{4}
        \\
        思路:

        解答: 
    \else
        \\ \\ \\
    \fi
}

\item {
    鸵鸟是世界上现在体型最大的鸟,1枚鸵鸟蛋的质量约为 1.5kg;蜂鸟是世界上现在体型最小的鸟,1枚蜂鸟蛋的质量约为 $2\times 10^{-1}g$. 1枚鸵鸟蛋的质量相当于多少蜂鸟蛋的质量?
    \ifshowSolution
        \fangsong\zihao{4}
        \\
        思路:

        解答: 
    \else
        \\ \\ \\
    \fi
}

\item {
    有一块钟乳石每年平均增长约 0.0001m。用科学计数法表示这块钟乳石增长1m 需要的时间(单位:年)。
    \ifshowSolution
        \fangsong\zihao{4}
        \\
        思路:

        解答: 
    \else
        \\ \\ \\
    \fi
}

\item {
    已知 $a^m = 8$, $a^n = 32$ (m,n是整数),求 $a^{m-2n}$ 的值。
    \ifshowSolution
        \fangsong\zihao{4}
        \\
        思路:

        解答: 
    \else
        \\ \\ \\
    \fi
}

\item {
    $ 10^{-5} - 10^{-6} $
    \ifshowSolution
        \fangsong\zihao{4}
        \\
        思路:

        解答: 
    \else
        \\ \\ \\
    \fi
}

\item {
    $ (-3x)^{4}\div (-3x) $
    \ifshowSolution
        \fangsong\zihao{4}
        \\
        思路:

        解答: 
    \else
        \\ \\ \\
    \fi
}

\item {
    $ (-a^2)^{3}\cdot (-a^3)^2 $
    \ifshowSolution
        \fangsong\zihao{4}
        \\
        思路:

        解答: 
    \else
        \\ \\ \\
    \fi
}

\item {
    $ (\frac{10^2}{0.000001})^{2} $
    \ifshowSolution
        \fangsong\zihao{4}
        \\
        思路:

        解答: 
    \else
        \\ \\ \\
    \fi
}

\item {
    $ (-\frac{3}{4})^4\times (-\frac{3}{4})^3 $
    \ifshowSolution
        \fangsong\zihao{4}
        \\
        思路:

        解答: 
    \else
        \\ \\ \\
    \fi
}

\item {
    $ 0.2^4\times 0.4^4 \times 12.5^4 $
    \ifshowSolution
        \fangsong\zihao{4}
        \\
        思路:

        解答: 
    \else
        \\ \\ \\
    \fi
}

\item {
    比较大小:$ 2^{55}, 3^{44}, 4^{33} $
    \ifshowSolution
        \fangsong\zihao{4}
        \\
        思路:

        解答: 
    \else
        \\ \\ \\
    \fi
}

\item {
    比较大小:$a=1.001\times 10^{-9}, b=9.99\times 10^{-8}, c=1.002\times 10^{-8}, d=-9.9999\times 10^{-7} $.
    \ifshowSolution
        \fangsong\zihao{4}
        \\
        思路:

        解答: 
    \else
        \\ \\ \\
    \fi
}

\item {
    判断 $49^{8} - 14^{2}\times 7^{12} $ 能否被9整除.
    \ifshowSolution
        \fangsong\zihao{4}
        \\
        思路:

        解答: 
    \else
        \\ \\ \\
    \fi
}

\item {
    $1 cm^3$ 空气的质量约为  $1.293\times 10^{-3} g$, $1 m^3$ 空气的质量是多少(单位:kg)?
    \ifshowSolution
        \fangsong\zihao{4}
        \\
        思路:

        解答: 
    \else
        \\ \\ \\
    \fi
}

\item {
    一滴水约$0.05 cm^3$,有一个未拧紧的水龙头每分钟大约漏40滴水,一天大约漏水多少立方米?
    \ifshowSolution
        \fangsong\zihao{4}
        \\
        思路:

        解答: 
    \else
        \\ \\ \\
    \fi
}

\item {
    一个水分子包含2个氢原子和1个氧原子。1个氢原子的质量约为 $1.674\times 10^{-27} kg$,1个氧原子的质量约为 $2.657\times 10^{-26} kg$,1个水分子的质量大约是多少(单位:kg)?
    \ifshowSolution
        \fangsong\zihao{4}
        \\
        思路:

        解答: 
    \else
        \\ \\ \\
    \fi
}


\begin{comment}
    \item {
        $ (-a^3b^2)^2$ 
        \ifshowSolution
            \fangsong\zihao{4}
            \\
            思路:
    
            解答: 
        \else
            \\ \\ \\
        \fi
    }

    \item {
        $ 3^{14}\times (-\frac19)^7 $ 
        \ifshowSolution
            \fangsong\zihao{4}
            \\
            思路:

            解答: 
        \else
            \\ \\ \\
        \fis
    }

\item {
    $ (\frac12)^2\times (\frac12)^5\times (-\frac12)^3 $ 
    \ifshowSolution
        \fangsong\zihao{4}
        \\
        思路:

        解答: 
    \else
        \\ \\ \\
    \fi
}

   

\item {
    $ x\cdot x^{(\quad)}\cdot x^{(n+1)} = x^{n+6}. $ ($n$是正整数) 
    \ifshowSolution
        \fangsong\zihao{4}
        \\
        思路:

        解答: 
    \else
        \\ \\ \\
    \fi
} 


\item {
    $ a^{m+1}\cdot a^{m-1}$
    \ifshowSolution
        \fangsong\zihao{4}
        \\
        思路:

        解答: 
    \else
        \\ \\ \\
    \fi
}

\item {
    $ -[(2a-b)^{4}]^{2}$ 
    \ifshowSolution
        \fangsong\zihao{4}
        \\
        思路:

        解答: 
    \else
        \\ \\ \\
    \fi
}

\end{comment}