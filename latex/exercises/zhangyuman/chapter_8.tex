\section{整式乘法}

\item {
    $\bigstar$
    (用2种方法)若把代数式$x^2-4x-5$化成$(x-m)^2+k$的形式, 其中$m, k$为常数. 求$m+k$的值. 
    \ifshowSolution
        \fangsong\zihao{4}
        \\
        思路1: 把$x^2-4x-5$转化为$(x-m)^2+k$的形式, 求出m和k. 

        解答: 
        \begin{align*}
            x^2-4x-5 &= x^2 - 2\times 2x + 4 - 9 \\
            &= (x-2)^2 - 9 \\
        \end{align*}
        所以, 
        \[\left\{ 
            \begin{array}{lc}
                m = 2\\
                k =-9
            \end{array}
        \right. \]
        $\therefore m+k=-7. $
        \\
        \begin{tikzpicture}
            \draw[dashed] (0, 0) -- (15, 0);
        \end{tikzpicture}
        \\
        思路2: 对应项系数,常数项相等。
        \begin{align*}
            (x-m)^2+k &= x^2 - 2m\cdot x + m^2 + k \\
        \end{align*}
        所以, 
        \[\left\{ 
            \begin{array}{lc}
                4 = 2m\\
                -5 = m^2 + k
            \end{array}
        \right. \]
        所以,
        \[\left\{ 
            \begin{array}{lc}
                m = 2\\
                k =-9
            \end{array}
        \right. \]
        $\therefore m+k=-7. $
    \else
        \\ \\ \\
    \fi
}

\item {
    $\bigstar \bigstar$
    用公式计算: $(x+2y-3z)^2$. 
    \\ \\ \\
}

\begin{comment}
\item {
    $a=2^{44}, b=3^{33}, c=5^{22}$, 比较$a, b, c$的大小. 
    \ifshowSolution
    \fangsong\zihao{4}
    \\
    思路: 把指数化为一样, 比较底数大小. 

    解答: 
    \begin{align*}
        a &= (2^4)^{11} = 16^{11}\\
        b &= (3^3)^{11} = 27^{11}\\
        c &= (5^2)^{11} = 25^{11}\\
        & 16^{11} < 25^{11} < 27^{11}\\
        &\therefore a < c < b. 
    \end{align*}
    \else
        \\ \\ \\
    \fi
}
\end{comment}

\begin{comment}
\item {
    已知$100^a=20, 1000^b=50$, 则$a+\frac{3}{2}b-\frac{3}{2}$的值是多少? 
    \ifshowSolution
    \fangsong\zihao{4}
    \\
    思路: 观察$100^a, 1000^b$ 发现 $a, b$都出现在指数上, 要求$a+\frac{3}{2}b-\frac{3}{2}$的值, 应该想到尝试把$a+\frac{3}{2}b-\frac{3}{2}$放在指数上. 

    解答: 
    \begin{align*}
        100^{a+\frac{3}{2}b-\frac{3}{2}} &= \frac{100^a\cdot 100^{\frac{3}{2}b}}{100^\frac{3}{2}}\\
        &= \frac{100^a\cdot 10^{2\cdot \frac{3}{2}b}}{10^{2\cdot\frac{3}{2}}}\\
        &= \frac{100^a\cdot 10^{3b}} {10^{3}}\\
        &= \frac{100^a\cdot 1000^{b}} {1000}\\
        &= \frac{20\times 50} {1000}\\
        &= 1\\
        &\therefore a+\frac{3}{2}b-\frac{3}{2} = 0. 
    \end{align*}
    \else
        \\ \\ \\
    \fi
}
\end{comment}

\begin{comment}
\item {
    用多项式乘法法则推导完全平方公式. 
    \\ \\ \\
}
\end{comment}

\begin{comment}
\item {
    计算 $(-x-y)^2$. 
    \ifshowSolution
    \fangsong\zihao{4}
    \\
    思路: 先将括号里的负号处理掉, 再用公式. 

    解答: 
    \begin{align*}
        \mbox{原式} &= (x+y)^2\\
        &= x^2 +2xy + y^2
    \end{align*}
    \else
        \\ \\ \\
    \fi
}
\end{comment}

\begin{comment}
\item {
    判断下列各式的正误. 

    \textcircled{1}$(a-b)^2 = a^2 - b^2$\\
    \textcircled{2}$(-2a-3b)^2 = 4a^2 - 12ab +9b^2$\\
    \textcircled{3}$(\frac13 m + \frac12 n)^2 = \frac19 m^2 + \frac16 mn + \frac14 n^2$\\
    \textcircled{4}$(-y-3)^2 = y^2 + 6y + 9$
    \\ \\ \\
}
\end{comment}

\begin{comment}
\item {
    用公式计算. 

    $(1) (-\frac34 x + \frac43 y)^2$ \\ \\

    $(2) (-7m-2n)^2$ \\ \\
    
    $(3) (a+b+c)^2$ \\ \\
    \\ \\
}
\end{comment}

\item {
    $\bigstar$
    多项式 $a^2 + 4$ 加上一个单项式后, 可化为一个多项式的平方, 求这个单项式. (列出所有结果)
    \ifshowSolution
    \fangsong\zihao{4}
    \\
    思路: 想到完全平方公式 $(x + y)^2 = x^2 + 2xy + y^2$. 

    解答: 

    (1) $a^2$对应公式里的$x^2$, $4$对应公式里的$2xy$
    \[\left\{ 
        \begin{array}{lc}
            x^2 = a^2\\
            2xy = 4
        \end{array}
    \right. \]
    所以,
    \[\left\{ 
        \begin{array}{lc}
            x = a\\
            y = \frac2a
        \end{array}
    \right. \]
    或
    \[\left\{ 
        \begin{array}{lc}
            x = -a\\
            y = -\frac2a
        \end{array}
    \right. \]
    所以,$y^2 = \frac{4}{a^2} (a\neq 0). $

    (2) $a^2$对应公式里的$x^2$, $4$对应公式里的$y^2$, 所以
    \[\left\{ 
        \begin{array}{lc}
            x^2 = a^2 \\
            y^2 = 4
        \end{array}
    \right. \]
    所以,
    \[\left\{ 
        \begin{array}{lc}
            x = \pm a \\
            y = \pm 2
        \end{array}
    \right. \]
    所以,$2xy = \pm 4a. $

    
    (3) $4$对应公式里的$x^2$, $a^2$对应公式里的$2xy$, 所以
    \[\left\{ 
        \begin{array}{lc}
            x^2 = 4 \\
            2xy = a^2
        \end{array}
    \right. \]
    所以,
    \[\left\{ 
        \begin{array}{lc}
            x = 2 \\
            y = \frac{a^2}{4}
        \end{array}
    \right. \]
    或
    \[\left\{ 
        \begin{array}{lc}
            x = -2 \\
            y = -\frac{a^2}{4}
        \end{array}
    \right. \]
    所以,$y^2 = \frac{a^4}{16}. $
    \else
        \\ \\ \\ \\ \\ \\
    \fi
}

\begin{comment}
\item {
    计算: $(-xy^2)\cdot (x^2y - 6xy)$
    \\ \\ \\
}

\item {
    计算: $(a+3)(a-1) + a(a-2)$
    \\ \\ \\
}
\end{comment}

\begin{comment}
\item {
    计算(用公式): $(x+2y-1)\cdot (x+2y+1)$
    \\ \\ \\
}
\end{comment}

\item {
    $\bigstar \bigstar$
    观察下列各式, 写出第n个等式. 

    第\textcircled{1}个等式: $1\times 5 + 4 = 3^2;$ \\
    第\textcircled{2}个等式: $3\times 7 + 4 = 5^2;$ \\
    第\textcircled{3}个等式: $5\times 9 + 4 = 7^2; \\ \cdots $ \\
    \ifshowSolution
        \fangsong\zihao{4}
        \\
        思路: 先找到等式中第1列的数字的规律, 再找下一列数的规律, 最后写出公式. 第一列数是奇数1,3,5,可以用n表示为 $2n-1$. 第二列数比第一列大4,所以是$2n+3$. 依此类推. 

        解答: 

        $(2n-1)\times (2n+3) + 4 = (2n+1)^2$. 
    \else
        \\ \\ \\
    \fi
}

\item {
    $\bigstar \bigstar$
    已知多项式 $x-2a$ 与 $x^2+x-1$ 的乘积中不含 $x^2$ 项, 则常数$a$的值是多少? 
    \ifshowSolution
    \fangsong\zihao{4}
    \\
    思路: 不含 $x^2$ 项的意思即该项的“系数为0”. 展开后合并同类项. 

    解答:
    \begin{align*}
        (x-2a)x^2+x-1 &= x^3 + (1-2a)x^2-(1+2a)x + 2a\\
        1-2a &= 0 \\
        a &= \frac12. 
    \end{align*}

    \else
        \\ \\ \\
    \fi
}

\begin{comment}
\item {
    多项式 $4a^2+9$ 加上一个单项式后, 可化为一个多项式的平方, 求这个单项式. 
    \\ \\ \\
}
\end{comment}

\begin{comment}
\item {
    已知 $a^2+a-1=0$, 求 $a^3 + 2a^2 + 2023$ 的值. 
    \\ \\ \\
}
\end{comment}

\item {
    $\bigstar$
    用公式计算: $(a-2b+3c)\cdot (a+2b+3c)$. 
    \ifshowSolution
    \fangsong\zihao{4}
    \\
    思路: 先把原式整理为平方差公式的形式,再用公式. 

    解答:
    \begin{align*}
        \mbox{原式} &= (a+3c-2b)\cdot (a+3c+2b) \\
        &= (a+3c)^2 - 4b^2 \\
        &= a^2 + 6ac + 9c^2 - 4b^2. \\
    \end{align*}

    \else
        \\ \\ \\
    \fi
}

\item {
    已知公式: $(x-1)(x^n + x^{n-1} + \cdots + x + 1) = x^{n+1} - 1$ (n为正整数). 
    
    利用上述公式, 求 $2^{100} + 2^{99} +\cdots + 2^2 + 2$ 的值. 
    \ifshowSolution
        \fangsong\zihao{4}
        \\
        思路: 使用公式前, 必须把原式的形式整理得和公式完全一致. 

        解答:记原式为S(这句话必须写,让别人知道S是什么). 
        \begin{align*}
            S + 1 &= 2^{100} + 2^{99} + \cdots + 2^2 + 2 + 1 \\
            &= (2-1)\times (2^{100} + 2^{99} + \cdots + 2^2 + 2 + 1) \\
            &= 2^{101} - 1 \\
            \therefore
            S &= 2^{101} - 2. 
        \end{align*}
    \else
        \\ \\ \\
    \fi
}

\item {
    $\bigstar$
    (用2种方法)用若干A类正方形(边长a)、B类长方形(长a, 宽b)、C类正方形(边长b), 拼成一个边长为 $2a + 2b$ 的正方形, 需要B类正方形多少个? 
    \ifshowSolution
        \fangsong\zihao{4}
        \\
        思路: 画图. 
    \else
        \\ \\ \\ \\ \\ \\ \\ \\
    \fi
}

\begin{comment}
\item {
    观察``杨辉三角'', 计算 $(a+b)^5$ 的结果中, 项 $a^3b^2$的系数, 
    \[
    \begin{matrix}
        & & & & 1 & & & \\
        & & & 1 & & 1 & & \\
        & & 1 & & 2 & & 1 & \\
        & 1 & & 3 & & 3 & & 1 \\
        1 & & 4 & & 6 & & 4 & & 1
    \end{matrix}
    \]
    \begin{align*}
        (a+b)^1 &= a+b \\
        (a+b)^2 &= a^2 + 2ab + b^2 \\
        (a+b)^3 &= a^3 + 3a^2b + 3ab^2 + b^3 \\
        (a+b)^4 &= a^4 + 4a^3b + 6a^2b^2 + 4ab^3 + b^4 \\
    \end{align*}
    \ifshowSolution
        \fangsong\zihao{4}
        \\
        % 思路: 画图. 
    \else
        \\ \\ \\
    \fi
}
\end{comment}

\begin{comment}
\item {
    判断 $49^8 - 14^2\times 7^{12}$ 能否被9整除, 并说明理由. 
    \ifshowSolution
        \fangsong\zihao{4}
        \\
        % 思路: . 
    \else
        \\ \\ \\
    \fi
}
\end{comment}

\begin{comment}
\item {
    求代数式 $y^2 +10y + 27$ 的最小值. 
    \ifshowSolution
        \fangsong\zihao{4}
        \\
        思路: 将原式整理成 $(y + m)^2 + k$ 的形式; 再利用 $a \geq 0$ 的性质,计算出原式的最小值. 
    \else
        \\ \\ \\
    \fi
}
\end{comment}

\item {
    $\bigstar\bigstar$
    求代数式 $-m^2 + 4m + 8$ 的最值, 并判断是最大值还是最小值. 
    \ifshowSolution
        \fangsong\zihao{4}
        \\
        思路: 利用 $a^2 \geq 0$ 的性质. 
    \else
        \\ \\ \\
    \fi
}

\item {
    $\bigstar$
    写出 $(a+b)^2, (a-b)^2, ab$ 三者之间的等量关系. 
    \ifshowSolution
        \fangsong\zihao{4}
        \\
        思路: 直接相减. 

        解答:
        \begin{align*}
            (a+b)^2 - (a-b)^2 &= a^2 + 2ab + b^2 - (a^2 - 2ab + b^2) \\
            &= 4ab. 
        \end{align*}
    \else
        \\ \\ \\
    \fi
}

\item {
    $\bigstar$
    若多项式 $2x^2 + kx - 14$ 是由整式 $x-2$ 与另一个整式 $2x+m$相乘得到的, 求k的值. 
    \ifshowSolution
        \fangsong\zihao{4}
        \\
        思路: 对应系数相等,常数项相等. 

        解答:
        \begin{align*}
            (x-2)(2x+m) &= 2x^2 + (m-4)x - 2m
        \end{align*}
        对应系数相等,常数项相等,所以
        \[\left\{ 
            \begin{array}{lc}
                m-4 = k \\
                2m = 14
            \end{array}
        \right. \]
        所以
        \[\left\{ 
            \begin{array}{lc}
                m = 7 \\
                k = 3
            \end{array}
        \right. \]
    \else
        \\ \\ \\
    \fi
}

\begin{comment}
\item {
    已知 $a^2 + a - 3 = 0$, 计算$(a^2 - 3)(a+1)$. 
    \ifshowSolution
        \fangsong\zihao{4}
        \\
        % 思路:
    \else
        \\ \\ \\
    \fi
}
\end{comment}

\item {
    若 $\lvert x+y-4 \rvert + (xy-3)^2 = 0$, 计算$x^2 + y^2$. 
    \ifshowSolution
        \fangsong\zihao{4}
        \\
        思路: 找出$x+y, xy, x^2+y^2$三者之间的数量关系. 
    \else
        \\ \\ \\
    \fi
}

\begin{comment}
\item {
    在计算 $(ax+1)(2x+b)$时,小泉同学看错了$b$的值, 计算结果为$2x^2 + 6x + 4$; 小张同学看错了$a$ 的值,计算结果为 $4x^2+12x+5$. 

    (1) 求 $a, b$. \\
    (2) 计算 $(ax+1)(2x+b)$ 的正确结果. 
    \ifshowSolution
        \fangsong\zihao{4}
        \\
        思路: 小泉同学算出的a是正确的, 小张同学算出的b是正确的. 
    \else
        \\ \\ \\ \\ 
    \fi
}
\end{comment}

\item {
    若 $x=2y+2$, 则求 $x^2 - 4xy + 4y^2$ 的值. 
    \ifshowSolution
        \fangsong\zihao{4}
        \\
        思路: 直接把$x=2y+2$ 代入,可以消去字母. 
    \else
        \\ \\ \\ \\ 
    \fi
}

\item {
    若$(3z^3 + M)(2z^2 - 1)$是一个五次多项式, 则下列说法中正确的是(\quad). 

    A. M是一个三次单项式

    B. M是一个三次多项式

    C. M的次数不高于三

    D. M不可能是一个常数
    \ifshowSolution
        \fangsong\zihao{4}
        \\
        % 思路: 直接把$x=2y+2$ 代入,可以消去字母. 
    \else
        \\ \\ \\ \\ 
    \fi
}

\item {
    已知多项式$x-a$与$x^2 - 2x + 1$的乘积的结果中不含 $x^2$ 项. 求常数$a$的值. 
    \ifshowSolution
        \fangsong\zihao{4}
        \\
        % 思路: 直接把$x=2y+2$ 代入,可以消去字母. 
    \else
        \\ \\ \\ \\ 
    \fi
}

\item {
    已知代数式$a^2 + (m+2)ab+ 16b^2$ 是一个完全平方式, 则有理数$m$ 的值是多少. 
    \ifshowSolution
        \fangsong\zihao{4}
        \\
        % 思路: 直接把$x=2y+2$ 代入,可以消去字母. 
    \else
        \\ \\ \\ \\ 
    \fi
}

\item {
    $\bigstar$
    已知$y=\frac13 [(x - 1)^2 + (x - 3)^2 + (x - 2)^2]$, 当$x$等于多少时, $y$的值最小? $y$的最小值是多少? 
    \ifshowSolution
        \fangsong\zihao{4}
        \\
        % 思路: 直接把$x=2y+2$ 代入,可以消去字母. 
    \else
        \\ \\ \\ \\ 
    \fi
}

\item {
    $y=x^2 + 6x - 2$. 当$x$等于多少时, $y$的值最小? $y$的最小值是多少? 
    \ifshowSolution
        \fangsong\zihao{4}
        \\
        % 思路: 直接把$x=2y+2$ 代入,可以消去字母. 
    \else
        \\ \\ \\ \\ 
    \fi
}

\item {
    $y=x^2 + 16x + 3$. 当$x$等于多少时, $y$的值最小? $y$的最小值是多少? 
    \ifshowSolution
        \fangsong\zihao{4}
        \\
        % 思路: 直接把$x=2y+2$ 代入,可以消去字母. 
    \else
        \\ \\ \\ \\ 
    \fi
}

\item {
    $y=x^2 - 16x + 3$. 当$x$等于多少时, $y$的值最小? $y$的最小值是多少? 
    \ifshowSolution
        \fangsong\zihao{4}
        \\
        % 思路: 直接把$x=2y+2$ 代入,可以消去字母. 
    \else
        \\ \\ \\ \\ 
    \fi
}

\item {
    先化简,再求值:$(2x+y)(2x-y) + (x-y)^2 - (10x^2y - 2xy^2)\div (2y)$. 其中$x=-4, y=\frac12$.
    \ifshowSolution
        \fangsong\zihao{4}
        \\
        % 思路: 
    \else
        \\ \\ \\ \\ 
    \fi
}

\item {
    定义:$L(A)$ 是多项式A化简后的项数,例如多项式$A=x^2+2x-3$,则$L(A)=3$. 一个多项式$A$乘多项式$B$化简得到多项式C(即$C=A\times B$),如果$L(A)\leq L(C)\leq L(A)+1$, 则称$B$是$A$的``好多项式'',如果$L(A)=L(C)$, 则称$B$是$A$的``极好多项式''。若$A=x-3, B=x^2-ax+9$均是关项式$x$的多项式,且$B$是$A$的“极好多项式”,则$a$等于多少?
    \ifshowSolution
        \fangsong\zihao{4}
        \\
        % 思路: 
    \else
        \\ \\ \\ \\ 
    \fi
}

\item {
    已知$\frac1m + m = 3$,则$\frac{1}{m^2} + m^2$的值是多少?
    \ifshowSolution
        \fangsong\zihao{4}
        \\
        % 思路: 7
    \else
        \\ \\ \\ \\ 
    \fi
}