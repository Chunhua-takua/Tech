\begin{comment}
\section{幂的运算}

\item {
    若$a,b$是正整数,且满足 $2^a + 2^a + 2^a + 2^a = 2^b\cdot 2^b\cdot 2^b\cdot 2^b\cdot$,则 $a$与$b$ 的关系是?\\
    \ifshowSolution
    \fangsong\zihao{6}
    \\
    正解: $a+2=4b$
    \else
        \\ \\ \\
    \fi
}

\item {
    $\bigstar\bigstar\bigstar\bigstar$
    (用科学记数法)一个正方体集装箱的棱长为 $0. 8 \rm{m}$. \\
    % (1) 这个集装箱的体积是多少?(用科学记数法)\\
    (2) 若有一个小立方块的棱长为$2\times 10^{-3} $ m, 则需要多少个这样的小立方块才能将集装箱装满?
    \ifshowSolution
    \fangsong\zihao{6}
    \\
    思路: 问题(2)注意简便运算. 

    正解:
    \begin{align*}
        \frac{0. 8^3} {(2\times 10^{-3})^3} &= \frac{0. 8^3} {8\times 10^{-9}}\\
        &= \frac{0. 064} {\frac{1}{10^9}}\\
        &= 0. 064\times 10^9\\
        &= 6. 4\times 10^{-2}\times 10^9\\
        &= 6. 4\times 10^7\\
    \end{align*}
    \else
        \\ \\ \\
    \fi
}

\item {
    当 $ x=7, y=-\frac{1}{7}$ 时, 求 $x^{4n+1}\cdot y^{4n+2}$ ($n$为整数)的值. 
    \ifshowSolution
        \fangsong\zihao{6}
        \\
        思路: 观察到$xy=-1$, 让将$x$和$y$凑成一对, 相乘. 直接将$x, y$的值代入表达式中进行计算. 

        正解: 
        \begin{align*}
            \mbox{原式} &= 7^{4n+1}\cdot \left(-\frac{1}{7}\right) ^{4n+2}\\
            &= [7\times(-\frac{1}{7})]^{4n+1} \cdot(-\frac{1}{7})\\
            &= (-1)^{4n+1} \cdot(-\frac{1}{7})\\
            &= \frac{1}{7}. 
        \end{align*}
    \else
        \\ \\ \\
    \fi
}

\item {
    已知$ m=8^9, n=9^8 $, 用含$m, n$的式子表示 $72^{72}$. 
    \ifshowSolution
        \fangsong\zihao{6}
        \\
        思路: 观察到$72=8\times 9$, 将原式中的72分解, 凑出$m, n$. 
        
        正解: 
        \begin{align*}
            \mbox{原式} &= (8\times 9)^{72}\\
            &= 8^{72}\times 9^{72}\\
            &= (8^9)^8\times (9^8)^9\\
            &= m^8 n^9. 
        \end{align*}
    \else
        \\ \\ \\
    \fi
}

\item {
    已知$x-y=k$, 求$(3x-3y)^3. $
    \ifshowSolution
        \fangsong\zihao{6}
        \\
        正解: 
        \begin{align*}
            \mbox{原式} &= [3(x-y)]^3\\
            &= 27(x-y)^3\\
            &= 27k^3. 
        \end{align*}
    \else
        \\ \\ \\
    \fi
}

\item {
    若$(a^nb^mb)^3 = a^9 b^{15}$, 求$2^{m+n}$. 
    \ifshowSolution
        \fangsong\zihao{6}
        \\
        思路: 先将左边化简, 再与右边比较, 解出$m, n$. 
        
        正解: 
        \begin{align*}
            (a^nb^{m+1})^3 &= a^9b^{15}\\
            a^{3n}b^{3m+3} &= a^9b^{15}
        \end{align*}
        $\therefore 3n=9, 3m+3=15$\\
        $\therefore n=3, m=4$
        \begin{align*}
            2^{m+n} &= 2^7\\
            &= 128. 
        \end{align*}
    \else
        \\ \\ \\
    \fi
}

    \item {
        化简: $(-a-b)^{2n}$ ($n$为整数). 
    }
    \\ \\ \\
    \item {
        化简: $(-a-b)^{2n+1}$ ($n$为整数). 
    }
    \\ \\ \\

    \item {
        (用科学计数法表示)已知 1 nm = 0. 000000001 m, 则 15 nm 等于多少 m?
        \ifshowSolution
        \fangsong\zihao{6}
        \\
        正解: 

        \textcircled{1} 写出换算关系
        \begin{align*}
            1 \rm{nm} &= 10^{-9} \rm{m}
        \end{align*}
        \textcircled{2} 两边同时乘15
        \begin{align*}
            15 \rm{nm} &= 15 \times 10^{-9} \rm{m}\\
            &= 1. 5\times 10^{-8} \rm{m}. 
        \end{align*}
        \fi
    }
    \\ \\ \\

    \item {
        (用科学计数法表示)肥皂泡表面厚度大约是 0. 0007 mm, 换算成以米为单位是多少?
    \ifshowSolution
    \fangsong\zihao{6}
    \\
    正解: 

    \textcircled{1} 写出换算关系
    \begin{align*}
        1 \rm{mm} &= 10^{-3} \rm{m}
    \end{align*}
    \textcircled{2} 两边同时乘0. 0007
    \begin{align*}
        0. 0007 \rm{mm} &= 0. 0007 \times 10^{-3} \rm{m}\\
        &= 7\times 10^{-7} \rm{m}. 
    \end{align*}
    \fi
    }
    \\ \\ \\

\item {
    (用科学计数法表示)已知 $0. 25 \upmu$m $ = 2. 5\times 10^{-7}$m, 那么 1 m 等于多少$\upmu$m?
    \ifshowSolution
        \fangsong\zihao{6}
        \\
        思路: 将题中给出的换算关系两边同时除以 $2. 5\times 10^{-7}$, 右边就出现了 1m. 

        正解: 
        \begin{align*}
            \frac{0. 25}{2. 5\times 10^{-7}} \rm{\upmu m} &= 1\rm{m}\\
            \frac{2. 5\times 0. 1}{2. 5\times \frac{1}{10^7}} \rm{\upmu m} &= 1\rm{m}\\
            0. 1\times 10^7 \rm{\upmu m} &= 1\rm{m}\\
            10^6 \rm{\upmu m} &= 1\rm{m}\\
            1\rm{m} &= 10^6 \rm{\upmu m}. 
        \end{align*}
    \else
        \\ \\ \\
    \fi
}

    \item {
        若多项式$ 9x^2 - mx+16$是一个完全平方式, 则 $m$的值是多少?
    }
    \\ \\ \\

\item {
    已知$a^2+b^2=8, a-b=3$, 求$ab$的值. 
    \ifshowSolution
        \fangsong\zihao{6}
        \\
        思路: 看到$a^2+b^2, a-b, ab$, 应该想到完全平方公式. 
    \else
        \\ \\ \\
    \fi
}

    \item {
        若$x^2+mx+9$是完全平方式, 求常数$m$的值. 
    }
    \\ \\ \\

    \item {
        若$x+y=2$, 求代数式$x^2-y^2+4y$的值. 
    }
    \\ \\ \\

\item {
    (注意: 除号使用分数形式) 已知$10^{-m}=a, 10^{-n}=b$($m, n$是整数), 求$10^{2m-3n}$的值(用含有$a, b$的代数式表示). 
    \\ \\ \\
}

\item {
    已知$2^x=3, 2^y=6, 2^z=12$, 判断下列有关$x, y, z$的数量关系式的对错. \\
    (1) $x+z=2y$\\
    (2) $x+y+3=2z$\\
    (3) $4x=z$\\
    (4) $x+1=y$
    \\ \\
}

\item {
    计算: $ (\frac{1}{2})^{-1} + \lvert 2-\pi \rvert $
    \ifshowSolution
    \fangsong\zihao{6}
    \\
    思路: 去绝对值符号, 运算到底. 

    正解: 
    \begin{align*}
        \mbox{原式} &= 2 + \pi - 2\\
        &= \pi. 
    \end{align*}
    \else
        \\ \\ \\
    \fi
}

\item {
    已知$(x+2)^{x+5}=1$, 求$x$. 
    \\ \\ \\
}

    \item {
        (把 $\frac{1}{27}$ 化为以3为底的幂) 若$3^{x-1}=\frac{1}{27}$, 求$x$. 
        \\ \\ \\
    }
    
\item {
    (注意: 除号使用分数形式) 已知$a^{2n}=3, a^{3m}=5$, 求$a^{6n-9m}$. 
    \ifshowSolution
    \fangsong\zihao{6}
    \\
    思路: 将$a^{6n-9m}$凑出$a^{2n}, a^{3m}$, 直接代入计算. 结果使用分数形式即可. 

    正解: 
    \begin{align*}
        a^{6n-9m} &= \frac{a^{6n}}{a^{9m}}\\
        &= \frac{(a^{2n})^3} {(a^{3m})^3}\\
        &= \frac{3^3} {5^3}\\
        &= \frac{27} {125}. 
    \end{align*}
    \else
        \\ \\ \\
    \fi
}

    \item {
        已知$3\cdot2^x + 2^{x+1}=40$, 求$x$. 
        \ifshowSolution
        \fangsong\zihao{6}
        \\
        思路: 将左边的2个$2^{x}$整理到一起. 
    
        正解: 
        \begin{align*}
            3\cdot2^x + 2^{x+1} &= 40\\
            3\cdot2^x + 2\cdot 2^{x} &= 40\\
            5\cdot2^x &= 40\\
            2^x &= 8\\
            \therefore x = 3. 
        \end{align*}
        \else
            \\ \\ \\
        \fi
    }
\end{comment}