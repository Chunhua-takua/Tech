\section{二元一次方程组}

\item {
    解方程组。
    \[\left\{
        \begin{array}{l}
            2x - y - z = 0 \\
            x + z = 5 \\
            3x + y - 2z = 1
        \end{array}
    \right.\]
    \ifshowSolution
        \fangsong\zihao{4}
        \\
        思路1

        解答: 
    \else
        \\ \\ \\ 
    \fi
}

\item {
    解方程组。
    \[\left\{
        \begin{array}{l}
            \frac{1}{x+y} + \frac{1}{y+z} = \frac{5}{6} \\
            \frac{1}{y+z} + \frac{1}{z+x} = \frac{7}{12} \\
            \frac{1}{z+x} + \frac{1}{x+y} = \frac{3}{4}
        \end{array}
    \right.\]
    \ifshowSolution
        \fangsong\zihao{4}
        \\
        思路1

        解答: 
    \else
        \\ \\ \\ 
    \fi
}

\item {
    % $\bigstar$
    在等式 $y=ax^2+bx+c$ 中,当$x=1$时,$y=-2$;当 $x=-1$ 时,$y=20$;当 $x=\frac32$ 与 $x=\frac13$ 时,$y$的值相等. 求 $a,b,c$ 的值。
    \ifshowSolution
        \fangsong\zihao{4}
        \\
        思路1

        解答: 
    \else
        \\ \\ \\ 
    \fi
}

\item {
    % $\bigstar$
    已知方程组
    $\begin{cases}
        x+y=3a \\ 
        y+z=5a \\ 
        x+z=4a 
    \end{cases}$
    的解使代数式 $x-2主+3z$ 的值是 -10.  求 $a$ 的值。
    \ifshowSolution
        \fangsong\zihao{4}
        \\
        思路1

        解答: 
    \else
        \\ \\ \\ 
    \fi
}

\item {
    % $\bigstar$
    在等式 $y=ax^2+bx+c$ 中,当$x=-1$时,$y=5$;当 $x=1$ 时,$y=1$;当 $x=2$时,$y=2$. 求 $a,b,c$ 的值。
    \ifshowSolution
        \fangsong\zihao{4}
        \\
        思路1

        解答: 
    \else
        \\ \\ \\ 
    \fi
}

\item {
    % $\bigstar$
    \[\left\{
        \begin{array}{l}
            3a + 2b = 5 \\
            2a - b = 1
        \end{array}
    \right.\]
    \ifshowSolution
        \fangsong\zihao{4}
        \\
        思路1

        解答: 
    \else
        \\ \\ 
    \fi
}

\item {
    % $\bigstar$
    \[\left\{
        \begin{array}{l}
            \frac{x+3}{2} + \frac{y+5}{3} = 7 \\
            \frac{x+3}{2} + \frac{y+5}{6} = 5
        \end{array}
    \right.\]
    \ifshowSolution
        \fangsong\zihao{4}
        \\
        思路1

        解答: 
    \else
        \\ \\ 
    \fi
}