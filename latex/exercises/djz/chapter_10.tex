\section{二元一次方程组}

\item {
    解方程组。
    \[\left\{
        \begin{array}{l}
            2x - y - z = 0 \\
            x + z = 5 \\
            3x + y - 2z = 1
        \end{array}
    \right.\]
    \ifshowSolution
        \fangsong\zihao{6}
        \\
        正解: 2;1;3.
    \else
        \\ \\ \\ \\ \\
    \fi
}

\item {
    解方程组。
    \[\left\{
        \begin{array}{l}
            \frac{1}{x+y} + \frac{1}{y+z} = \frac{5}{6} \\
            \frac{1}{y+z} + \frac{1}{z+x} = \frac{7}{12} \\
            \frac{1}{z+x} + \frac{1}{x+y} = \frac{3}{4}
        \end{array}
    \right.\]
    \ifshowSolution
        \fangsong\zihao{6}
        \\
        正解: $\frac32, \frac12, \frac52$.
    \else
        \\ \\ \\ \\ \\
    \fi
}

\item {
    解方程组。
    \[\left\{
        \begin{array}{l}
            \frac{3x-2y}{6} + \frac{2x+3y}{7} = 1 \\
            \frac{3x-2y}{6} - \frac{2x+3y}{7} = 5 
        \end{array}
    \right.\]
    \ifshowSolution
        \fangsong\zihao{6}
        \\
        正解: 2;-6.
    \else
        \\ \\ \\ \\ \\
    \fi
}

\item {
    $\bigstar$
    在等式 $y=ax^2+bx+c$ 中,当$x=1$时,$y=-2$;当 $x=-1$ 时,$y=20$;当 $x=\frac32$ 与 $x=\frac13$ 时,$y$的值相等. 求 $a,b,c$ 的值。
    \ifshowSolution
        \fangsong\zihao{6}
        \\
        正解: 
    \else
        \\ \\ \\ \\ \\
    \fi
}

\item {
    已知方程组
    $\begin{cases}
        x+y=3a \\ 
        y+z=5a \\ 
        x+z=4a 
    \end{cases}$
    的解使代数式 $x-2主+3z$ 的值是 -10.  求 $a$ 的值。
    \ifshowSolution
        \fangsong\zihao{6}
        \\
        正解: $-\frac45$.
    \else
        \\ \\ \\ \\ \\
    \fi
}

\item {
    在等式 $y=ax^2+bx+c$ 中,当$x=-1$时,$y=5$;当 $x=1$ 时,$y=1$;当 $x=2$时,$y=2$. 求 $a,b,c$ 的值。
    \ifshowSolution
        \fangsong\zihao{6}
        \\
        正解: 
    \else
        \\ \\ \\ \\ \\
    \fi
}

\item {
    \[\left\{
        \begin{array}{l}
            3a + 2b = 5 \\
            2a - b = 1
        \end{array}
    \right.\]
    \ifshowSolution
        \fangsong\zihao{6}
        \\
        正解: a=b=1.
    \else
        \\ \\ \\ \\ \\
    \fi
}

\item {
    \[\left\{
        \begin{array}{l}
            \frac{x+3}{2} + \frac{y+5}{3} = 7 \\
            \frac{x+3}{2} + \frac{y+5}{6} = 5
        \end{array}
    \right.\]
    \ifshowSolution
        \fangsong\zihao{6}
        \\
        正解: x=3, y=7.
    \else
        \\ \\ \\ \\ \\
    \fi
}

\item {
    若关于$x, y$的方程组
    $\begin{cases}
        a_1x + b_1y = c_1, \\ 
        a_2x + b_2y = c_2  
    \end{cases}$
    的解为
    $\begin{cases}
        x = 2, \\ 
        y = 1,  
    \end{cases}$
    则求关于 $x, y$ 的方程组
    $\begin{cases}
        2a_1x - b_1y = 3c_1, \\ 
        2a_2x - b_2y = 3c_2
    \end{cases}$
    的解.
    \ifshowSolution
        \fangsong\zihao{6}
        \\
        正解:
    \else
        \\ \\ \\ \\ \\
    \fi
}

\item {
    关于$x, y$的二元一次方程
    $(a+1)x + (a-2)y + 5 - 2a = 0$
    (其中a为常数且$a\neq -1, 2$).

    (1) 若
    $\begin{cases}
        x = 2, \\ 
        y = 1
    \end{cases}$
    是该方程的一个解,求 $a$ 的值.

    (2) 当a每取一个值时,都可得到一个方程,而这些方程有一个公共解,求出这个公共解.
    \ifshowSolution
        \fangsong\zihao{6}
        \\
        正解:
    \else
        \\ \\ \\ \\ \\
    \fi
}

\item {
    已知关于$x, y$的方程组
    $\begin{cases}
        x + y = a+4, \\ 
        x - 2y = 4a-2  
    \end{cases}$
    判断下列结论的对错:

    \textcircled{1} 当 $a=3$ 时,方程组的解也是 $x+y=2a+1$ 的解;

    \textcircled{2} 无论 $a$ 取何值,$x,y$的值都不可能互为相反数;

    \textcircled{3} $x,y$都为自然数的解有4对;

    \textcircled{4} 若 $2x+y=9$, 则 $a=1$.
    \ifshowSolution
        \fangsong\zihao{6}
        \\
        正解:
    \else
        \\ \\ \\ \\ \\ \\ \\ \\
    \fi
}

\item {
    已知$t$ 满足方程组
    $\begin{cases}
        x - 2y = -t, \\ 
        2x + t = -y  
    \end{cases}$
    则$x,y$之间满足的关系式是:
    \ifshowSolution
        \fangsong\zihao{6}
        \\
        正解:
    \else
        \\ \\ \\ \\ \\
    \fi
}

\item {
    对于未知数为$x, y$的二元一次方程组,如果方程组的解$x,y$满足 $|x-y| = 1$,我们就说方程组的解x和y具有``唯精唯一关系''.

    (1) 若方程组
    $\begin{cases}
        x + 2y = 7, \\ 
        x - y = 1  
    \end{cases}$
    的解x,y是否具有 ``唯精唯一关系''?说明理由.

    (2) 若方程组
    $\begin{cases}
        2x - y = 6, \\ 
        4x + y = 6m  
    \end{cases}$
    的解x,y具有 ``唯精唯一关系'',求m的值.
    
    (3) 若方程组
    $\begin{cases}
        2x - 3y + 4 = ab, \\ 
        3x + 2y + 6 = 8ab - 13b  
    \end{cases}$
    的解$x,y$具有 ``唯精唯一关系'',又$a,b$均为正整数,求$a,b$的值.
    \ifshowSolution
        \fangsong\zihao{6}
        \\
        正解:
    \else
        \\ \\ \\ \\ \\ \\ \\ \\ \\ \\ \\ \\ \\ \\
    \fi
}

\item {
    关于$x,y$ 的二元一次方程组
    $\begin{cases}
        x + 2y = -a+1, \\ 
        x - 3y = 4a + 6  
    \end{cases}$
    ($a$是常数),若不论$a$取什么实数,代数式$kx - 2y$($k$是常数)的值始终不变,求$k$.
    \ifshowSolution
        \fangsong\zihao{6}
        \\
        正解:
    \else
        \\ \\ \\ \\ \\
    \fi
}