\section{因子分解}
\subsection{整数与带余除法}

自然数 Natural numbers, 
$ \mathbb{N} = \{ 1, 2, 3, \cdots \} $

整数 Integers, 
$ \mathbb{Z} = \{ \cdots, -3, -2, -1, 0, 1, 2, 3, \cdots \} $

整数之间可以加、减、乘, 其结果仍是整数. $ \mathbb{Z} $ 对加、减、乘是封闭的. 

\textbf{定义1}(整除) 设 $a, b$ 均为整数, $ b \neq 0 $. 若存在整数 $ q $ 使得
\[ 
    a = bq,
\]
则称 b 整除 a, 记为 $ b \mid a $ ($b$ divides $a$). 

如果不存在整数 $q$ 使得 $a=bq$, 则称 $b$ 不整除 $a$, 记为 $b \nmid a$. 

整除性的性质:\\
(1)若 $ b \mid a $ 且 $ a \mid b$, 则 $a = \pm b$.\\
证明\\
首先 $a,b\neq 0$.
根据$ b \mid a $, 可知存在整数 $q_1$, 使得 $a=q_1 b$. 
根据$ a \mid b$, 可知存在整数 $q_2$, 使得 $b=q_2 a$. 
上两式相乘, 得 $q_1 q_2 = 1$, 于是\\
\[
    \left\{ 
        \begin{array}{lc}
            q_1 = 1\\
            q_2 = 1
        \end{array}
    \right.
\]
或
\[
    \left\{ 
        \begin{array}{lc}
            q_1 = -1\\
            q_2 = -1
        \end{array}
    \right.
\]
所以, $a = \pm b$.

(2)若 $ a \mid b $ 且 $ b \mid c $, 则 $a \mid c$.\\
证明\\
根据 $a\mid b$, 可知存在整数 $q_1$, 使得 $b=q_1 a$.\\
根据 $b\mid c$, 可知存在整数 $q_2$, 使得 $c=q_2 b$.\\
由上两式可得, $c=(q_1q_2)a$, 即存在整数 $q=q_1q_2$, 使得 $c=qa$, 即 $a \mid c$.

(3)$a\mid b$ 当且仅当 $ac\mid bc$. ($c\neq 0$)\\
证明\\
$\Rightarrow$: 存在整数 $q_1$,使得 $b=q_1a$.\\
$\because c\neq 0$, $\therefore bc=q_1ac$, 即 $ac\mid bc$.\\
$\Leftarrow$: 存在整数  $q_2$, 使得$bc=q_2ac$.\\
$\because c\neq 0, \therefore b=q_2a$, 即$a\mid b$.

(4)若$b\mid a_1$ 且$b\mid a_2$, 则 $b\mid a_1\pm a_2$.\\
证明\\
$\because b\mid a_1$, $\therefore$ 存在整数 $q_1$使得 $a_1 = q_1b$.\\
$\because b\mid a_2$, $\therefore$ 存在整数 $q_2$ 使得 $a_2 = q_2b$.\\
上两式相加,可得 $a_1+a_2=(q_1+q_2)b$,所以 $b\mid a_1+a_2$.\\
上两式相减,可得 $a_1-a_2=(q_1-q_1)b$, 所以 $b\mid a_1-a_2$.

1不是素数(prime number)也不是合数(composite number)。

以$\pi(x)$记不超过正数$x$的正素数个数,则$\pi(x)$与$xln x$近似(素数定理),即二者比值趋近于1.

\textbf{自然数的良序性}:自然数集的每个非空子集都含有最小数。

\textbf{引理1}(因子分解) 任一整数 $n(\neq 0, \pm 1)$ 可写为有限个素数之积。\\
数学归纳法。\\
\textcircled{1}(归纳起点) 当$n=2$时,引理显然成立。\\
\textcircled{2}(归纳法假设) 当$n>2$时,假设引理对所有小于$n$的自然数成立。\\
\textcircled{3} 现在证明引理对$n$成立。\\
如果$n$是素数,则引理对$n$成立。\\
如果$n$不是素数,则有$n=mq, (m,q\in \mathbb{N})$。显然 $m,q<n$,由假设可知$m,q$都可以写成有限个素数之积,
$m=q_1q_2\cdots q_s, q=q_{s+1}q_{s+2}\cdots q_{s+t}$
所以$n=q_1q_2\cdots q_s q_{s+1}q_{s+2}\cdots q_{s+t}$, 可以写为有限个素数之积。
证毕。$\hfill\qedsymbol$


