\section{因子分解}
\subsection{整数与带余除法}

自然数 Natural numbers, 
$ \mathbb{N} = \{ 1, 2, 3, \cdots \} $

整数 Integers, 
$ \mathbb{Z} = \{ \cdots, -3, -2, -1, 0, 1, 2, 3, \cdots \} $

整数之间可以加、减、乘, 其结果仍是整数. $ \mathbb{Z} $ 对加、减、乘是封闭的. 

\textbf{定义1}(整除) 设 $a, b$ 均为整数, $ b \neq 0 $. 若存在整数 $ q $ 使得
\[ 
    a = bq,
\]
则称 b 整除 a, 记为 $ b \mid a $ ($b$ divides $a$). 

如果不存在整数 $q$ 使得 $a=bq$, 则称 $b$ 不整除 $a$, 记为 $b \nmid a$. 

整除性的性质:\\
(1)若 $ b \mid a $ 且 $ a \mid b$, 则 $a = \pm b$.

证明\\
首先 $a,b\neq 0$.
根据$ b \mid a $, 可知存在整数 $q_1$, 使得 $a=q_1 b$. 
根据$ a \mid b$, 可知存在整数 $q_2$, 使得 $b=q_2 a$. 
上两式相乘, 得 $q_1 q_2 = 1$, 于是\\
\[
    \left\{ 
        \begin{array}{lc}
            q_1 = 1\\
            q_2 = 1
        \end{array}
    \right.
\]
或
\[
    \left\{ 
        \begin{array}{lc}
            q_1 = -1\\
            q_2 = -1
        \end{array}
    \right.
\]
所以, $a = \pm b$.