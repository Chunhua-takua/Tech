\documentclass[
    a4paper
    ]{ctexart}

\usepackage{makecell}
\usepackage{multirow}
\usepackage[UTF8, scheme=plain]{ctex}
\usepackage{hyperref}
\hypersetup{
    colorlinks=true,
    linktocpage=true,
    hidelinks
}
% \usepackage[hidelinks]{hyperref}

\usepackage[
    top = 37mm,
    left = 28mm,
    right = 26mm,
    bottom = 35mm
    ]{geometry}

\usepackage{titlesec}
\usepackage{lastpage}
\usepackage{fancyhdr}
\usepackage{color}
\definecolor{gray}{RGB}{100,100,100}

\usepackage{graphicx}
\usepackage{ulem}
\usepackage{CJKulem}

\usepackage{ctex}
\CTEXsetup[name={}, number={\chinese{section}}, format={\bfseries}]{section}

\date{}

% 行距
\renewcommand{\baselinestretch}
{1.5}

\begin{document}
    % 黑体小2号
    \title{\zihao{2}\heiti\textbf{刑事案件程序时间轴 }}
    % \setlength{\headsep}{10pt}
    \maketitle

    % 首页显示页眉页脚
    \thispagestyle{fancy}

    \fancyhf{}
    \fancyhead[L]{\includegraphics[scale=0.5]{D:/vt_logos/vt_logo_5.png}}


    % 正文 阿拉伯数字页码
    \setcounter{page}{1}
    \pagenumbering{arabic}
    \renewcommand{\headrulewidth}{0.5pt}

    \fancyhf{}
    \pagestyle{fancy}
    \fancyhead[L]{\includegraphics[scale=0.5]{D:/vt_logos/vt_logo_5.png}}

    \fancyfoot[C]{第 {\thepage} 页\quad 共 \pageref{LastPage} 页}

    \setlength{\headsep}{50pt}

    \begin{sloppy}
    {\zihao{4}
        \setlength{\parindent}{2em}

        \section{
            侦查阶段(公安阶段)
        }
        \begin{center}
            \begin{tabular}{|c|c|c|}
                \hline
                \textbf{事项} & \textbf{法条依据} & \textbf{期限} \\
                \hline
                传唤、拘传 & \multirow{2}*{一百一十九} & 12小时\\
                \cline{1-1}
                \cline{3-3}
                传唤、拘传(案情特别重大、复杂) & ~ & 24小时\\
                \hline

                \multirow{3}*{拘留} & \multirow{7}*{九十一} & 3天以内\\
                \cline{3-3}
                ~ & ~ & 特殊情况,4-7天\\
                \cline{3-3}
                ~ & ~ & 30天 \\
                \cline{1-1}
                \cline{3-3}

                提请批准逮捕 & ~ & 7天 \\
                \cline{1-1}
                \cline{3-3}
        
                \multirow{3}*{报捕、审查逮捕} & ~ & 3天以内\\
                \cline{3-3}
                ~ & ~ & 特殊情况,4-7天\\
                \cline{3-3}
                ~ & ~ & 30天 \\
                \hline

                逮捕后的侦查羁押 & \multirow{2}*{一百五十六} & 2个月  \\
                \cline{1-1}
                \cline{3-3}
                (案情复杂)逮捕后的侦查羁押 & ~ & 延长1个月  \\
                \hline
                (重大复杂)逮捕后的侦查羁押 & 一百五十八 & 延长2个月  \\
                \hline
                (重刑案件)逮捕后的侦查羁押 & 一百五十九 & 延长2个月  \\
                \hline
                \makecell[c]{(发现另有罪行一次)\\侦查羁押期限的重新计算} & 一百六十 & 7个月 \\
                \hline
            \end{tabular}
        \end{center}

        \vspace{1em}
        普通案件通常37天 + 2个月,最长合计可能15个月。

        \section{
            审查起诉(检察院阶段)
        }
        \begin{center}
            \begin{tabular}{|c|c|c|}
                \hline
                \textbf{事项} & \textbf{法条依据} & \textbf{期限} \\
                \hline
                审查起诉 & \multirow{2}*{一百七十二} & 1个月 \\
                \cline{1-1}
                \cline{3-3}
                (重大、复杂)审查起诉 & ~ & 延长15天\\
                \hline
                第一次补侦 & \multirow{4}*{一百七十五} & 1个月 \\
                \cline{1-1}
                \cline{3-3}
                审查起诉 & ~ & 1.5个月 \\
                \cline{1-1}
                \cline{3-3}
                第二次补侦 & ~ & 1个月 \\
                \cline{1-1}
                \cline{3-3}
                审查起诉 & ~ & 1.5个月 \\
                \hline
            \end{tabular}
        \end{center}

        \vspace{1em}
        普通案件不必经两退三延,时间在2个月左右。如有两退三延的情况,检察院阶段可能历经6个月或更久。

        \section{
            一审阶段(法院阶段)
        }
        \begin{center}
            \begin{tabular}{|c|c|c|}
                \hline
                \textbf{事项} & \textbf{法条依据} & \textbf{期限} \\
                \hline

                一般情况 & \multirow{8}*{二百零八}    & 3个月 \\
                
                
                % \hline
                \cline{1-1}
                \cline{3-3}
                重大复杂延长 & ~ & 3个月 \\
                \cline{1-1}
                \cline{3-3}
                第一次补充侦查 & ~ & 1个月 \\
                \cline{1-1}
                \cline{3-3}
                补侦后期限重新计算 & ~  & 6个月\\
                \cline{1-1}
                \cline{3-3}
                第二次补充侦查 & ~ & 1个月 \\
                \cline{1-1}
                \cline{3-3}
                补侦后期限重新计算 & ~ & 6个月 \\
                \cline{1-1}
                \cline{3-3}
                改变管辖重新计算 & ~ & 20个月\\
                \cline{1-1}
                \cline{3-3}
                特殊情况最高法院批准一次 & ~ & 3个月\\
                \hline
            \end{tabular}
        \end{center}

        \vspace{1em}
        普通案件通常3-6个月,最长合计可能43个月。

        \paragraph{
            第九十一条
        }
        【提请批捕和审查批捕的时限】公安机关对被拘留的人,认为需要逮捕的,应当在拘留后的三日以内,提请人民检察院审查批准。在特殊情况下,提请审查批准的时间可以延长一日至四日。

        对于流窜作案、多次作案、结伙作案的重大嫌疑分子,提请审查批准的时间可以延长至三十日。

        人民检察院应当自接到公安机关提请批准逮捕书后的七日以内,作出批准逮捕或者不批准逮捕的决定。人民检察院不批准逮捕的,公安机关应当在接到通知后立即释放,并且将执行情况及时通知人民检察院。对于需要继续侦查,并且符合取保候审、监视居住条件的,依法取保候审或者监视居住。


    }

\end{sloppy}
\end{document}
