\documentclass[
    a4paper
    ]{ctexart}

\usepackage[UTF8, scheme=plain]{ctex}
\usepackage[hidelinks]{hyperref}
\usepackage[
    twoside,
    top = 37mm,
    left = 28mm,
    right = 26mm,
    bottom = 35mm
    ]{geometry}

\usepackage{titlesec}
\usepackage{lastpage}
\usepackage{fancyhdr}
\usepackage{color}
\definecolor{gray}{RGB}{100,100,100}

\usepackage{graphicx}
\usepackage{ulem}
\usepackage{CJKulem}
\usepackage{enumitem}
\usepackage{makecell}
\usepackage{longtable}
\usepackage{supertabular}
\usepackage{ctex}

\date{}

% 行距
\renewcommand{\baselinestretch}
{1.75}

\newcommand{\tabincell}[2]{\begin{tabular}{@{}#1@{}}#2\end{tabular}}

\usepackage{ctex}
\CTEXsetup[name={}, number={\chinese{section}}]{section}
\CTEXsetup[name={(, )}, number={\chinese{subsection}}]{subsection}
\CTEXsetup[name={}, number={}, format={\heiti\zihao{-4}}]{paragraph}


\begin{document}
    % 黑体小2号
    \title{\zihao{-2}\heiti\textbf\texorpdfstring{张池武、黄小鸯假冒注册商标案\\ 检索报告}}
    \maketitle

    \vspace{25em}
    \centerline{\zihao{-4}\textbf{北京市万商天勤 (南京) 律师事务所}}
    \centerline{\zihao{-4}\textbf{\qquad 年 \qquad 月 \qquad 日}}

    % 首页显示页眉页脚
    \thispagestyle{fancy}

    \fancyhf{}
    \renewcommand{\headrulewidth}{0pt}
    \fancyhead[L]{\includegraphics[scale=1]{vt_logo.png}}
    % \fancyfoot[L]{\zihao{-5}\textcolor{gray}{律所地址:江苏省南京市建邺区江东中路347号国金中心一期31层}}

    \newpage

    \fancyhf{}
    \pagestyle{fancy}

    \fancyhead[L]{\includegraphics[scale=1]{vt_logo.png}}
    \fancyhead[R]{假冒注册商标案检索报告}
    \renewcommand{\headrulewidth}{0.5pt}
    \setlength{\headsep}{60pt}

    % 正文 阿拉伯数字页码
    \setcounter{page}{1}
    \pagenumbering{arabic}
    \fancyfoot[C]{第 {\thepage} 页\quad 共 \pageref{LastPage} 页}

    {\songti\zihao{-4}
        \setlength{\parindent}{2em}
        \section{\textbf{法律法规}}
            {
                \small
                \begin{longtable}{|c|c|c|c|}
                \hline
                序号 & 名称 & 效力等级 & 内容 \\
                \hline
                1 & 刑法 & 法律 & 
                \makecell[l]{
                    第二百一十四条 【销售假冒注册商标的商品罪】销售明知是假冒注册商标的\\
					商品,违法所得数额较大或者有其他严重情节的,处三年以下有期徒刑,并处\\
					或者单处罚金;违法所得数额巨大或者有其他特别严重情节的,处三年以上十\\
					年以下有期徒刑,并处罚金。
					} \\
                \hline

                2 & \makecell[l]{中华人\\民共和\\国商标\\法(\\2019\\)} & 法律 & 
                \makecell[l]{
                    第六十四条\quad 注册商标专用权人请求赔偿,被控侵权人以注册商标专用权人未\\
                    使用注册商标提出抗辩的,人民法院可以要求注册商标专用权人提供此前三年\\
                    内实际使用该注册商标的证据。注册商标专用权人不能证明此前三年内实际使\\
                    用过该注册商标,也不能证明因侵权行为受到其他损失的,被控侵权人不承担\\
                    赔偿责任。\\
					\qquad 销售不知道是侵犯注册商标专用权的商品,能证明该商品是自己合法取得\\
                    并说明提供者的,不承担赔偿责任。
                    } \\
                \hline

                3 & \makecell[l]{最高人\\民法院\\、最高\\人民检\\察院关\\于办理\\侵犯知\\识产权\\刑事案\\件具体\\应用法\\律若干\\问题的\\解释(\\三)(\\2020\\)} & 司法解释 & 
                \makecell[l]{
                    第一条\quad 具有下列情形之一的,可以认定为刑法第二百一十三条规定的“与其\\
                    注册商标相同的商标”:\\
                    \qquad (一)改变注册商标的字体、字母大小写或者文字横竖排列,与注册商标\\
                    之间基本无差别的;\\
                    \qquad (二)改变注册商标的文字、字母、数字等之间的间距,与注册商标之间\\
                    基本无差别的;\\
                    \qquad (三)改变注册商标颜色,不影响体现注册商标显著特征的;\\
                    \qquad (四)在注册商标上仅增加商品通用名称、型号等缺乏显著特征要素,不\\
                    影响体现注册商标显著特征的;\\
                    \qquad (五)与立体注册商标的三维标志及平面要素基本无差别的;\\
                    \qquad (六)其他与注册商标基本无差别、足以对公众产生误导的商标。\\
                    \\
                    第七条\quad 除特殊情况外,假冒注册商标的商品、非法制造的注册商标标识、侵\\
                    犯著作权的复制品、主要用于制造假冒注册商标的商品、注册商标标识或者侵\\
                    权复制品的材料和工具,应当依法予以没收和销毁。\\
                    \qquad 上述物品需要作为民事、行政案件的证据使用的,经权利人申请,可以在\\
                    民事、行政案件终结后或者采取取样、拍照等方式对证据固定后予以销毁。\\
                    \\
                    第八条\quad 具有下列情形之一的,可以酌情从重处罚,一般不适用缓刑:\\
                    \qquad (一)主要以侵犯知识产权为业的;\\
                    \qquad (二)因侵犯知识产权被行政处罚后再次侵犯知识产权构成犯罪的;\\
                    \qquad (三)在重大自然灾害、事故灾难、公共卫生事件期间,假冒抢险救灾、\\
                    防疫物资等商品的注册商标的;\\
                    \qquad (四)拒不交出违法所得的。\\
                    \\
                    第九条\quad 具有下列情形之一的,可以酌情从轻处罚:\\
                    \qquad (一)认罪认罚的;\\
                    \qquad (二)取得权利人谅解的;\\
                    \qquad (三)具有悔罪表现的;\\
                    \qquad (四)以不正当手段获取权利人的商业秘密后尚未披露、使用或者允许他\\
                    人使用的。\\
                    \\
                    第十条\quad 对于侵犯知识产权犯罪的,应当综合考虑犯罪违法所得数额、非法经\\
                    营数额、给权利人造成的损失数额、侵权假冒物品数量及社会危害性等情节,依法判处罚\\
                    金。\\
                    \qquad 罚金数额一般在违法所得数额的一倍以上五倍以下确定。违法所得数额无\\
                    法查清的,罚金数额一般按照非法经营数额的百分之五十以上一倍以下确定。\\
                    违法所得数额和非法经营数额均无法查清,判处三年以下有期徒刑、拘役、管\\
                    制或者单处罚金的,一般在三万元以上一百万元以下确定罚金数额;判处三年\\
                    以上有期徒刑的,一般在十五万元以上五百万元以下确定罚金数额。
                    } \\
                \hline

                    &&&\makecell[l]{
                        第二条\quad 销售明知是假冒注册商标的商品,销售金额在五万元以上的,属于刑\\
                        法第二百一十四条规定的“数额较大”,应当以销售假冒注册商标的商品罪判\\
                        处三年以下有期徒刑或者拘役,并处或者单处罚金。销售金额在二十五万元以\\
                        上的,属于刑法第二百一十四条规定的“数额巨大”,应当以销售假冒注册商\\
                        标的商品罪判处三年以上七年以下有期徒刑,并处罚金。\\
                        \\
                        第四条\quad 假冒他人专利,具有下列情形之一的,属于刑法第二百一十六条规定\\
                        的“情节严重”,应当以假冒专利罪判处三年以下有期徒刑或者拘役,并处或\\
                        者单处罚金:\\
                        \qquad (一)非法经营数额在二十万元以上或者违法所得数额在十万元以上的;\\
                        \qquad (二)给专利权人造成直接经济损失五十万元以上的;\\
                        \qquad (三)假冒两项以上他人专利,非法经营数额在十万元以上或者违法所得\\
                        数额在五万元以上的;\\
                        \qquad (四)其他情节严重的情形。\\
                    } \\
                3 & \makecell[l]{最高人\\民法院\\、最高\\人民检\\察院关\\于办理\\侵犯知\\识产权\\刑事案\\件具体\\应用法\\律若干\\问题的\\解释(\\2004\\)} & 司法解释 & \makecell[l]{
					第九条\quad 刑法第二百一十四条规定的“销售金额”,是指销售假冒注册商标的\\
                    商品后所得和应得的全部违法收入。具有下列情形之一的,应当认定为属于刑\\
                    法第二百一十四条规定的“明知”:\\
                    \qquad (一)知道自己销售的商品上的注册商标被涂改、调换或者覆盖的;\\
                    \qquad (二)因销售假冒注册商标的商品受到过行政处罚或者承担过民事责任、\\
                    又销售同一种假冒注册商标的商品的;(三)伪造、涂改商标注册人授权文件\\
                    或者知道该文件被伪造、涂改的;(四)其他知道或者应当知道是假冒注册商\\
                    标的商品的情形。\\
                    \\
					第十二条\quad 本解释所称“非法经营数额”,是指行为人在实施侵犯知识产权行\\
                    为过程中,制造、储存、运输、销售侵权产品的价值。已销售的侵权产品的价\\
                    值,按照实际销售的价格计算。制造、储存、运输和未销售的侵权产品的价值\\
                    ,按照标价或者已经查清的侵权产品的实际销售平均价格计算。侵权产品没有\\
                    标价或者无法查清其实际销售价格的,按照被侵权产品的市场中间价格计算。\\
					\qquad 多次实施侵犯知识产权行为,未经行政处理或者刑事处罚的,非法经营数\\
                    额、违法所得数额或者销售金额累计计算。\\
					\qquad 本解释第三条所规定的“件”,是指标有完整商标图样的一份标识。\\
                }\\
                \hline

                4 & \makecell[l]{关于办\\理侵犯\\知识产\\权刑事\\案件适\\用法律\\若干问\\题的意\\见(\\2011\\)} & 司法解释 & 
                \makecell[l]{
                    八、关于销售假冒注册商标的商品犯罪案件中尚未销售或者部分销售情形的定\\
                    罪量刑问题\\
                    销售明知是假冒注册商标的商品,具有下列情形之一的,依照刑法第二百一十\\
                    四条的规定,以销售假冒注册商标的商品罪(未遂)定罪处罚:\\
                    \qquad (一)假冒注册商标的商品尚未销售,货值金额在十五万元以上的;\\
                    \qquad (二)假冒注册商标的商品部分销售,已销售金额不满五万元,但与尚未\\
                    销售的假冒注册商标的商品的货值金额合计在十五万元以上的。\\
                    \qquad 假冒注册商标的商品尚未销售,货值金额分别达到十五万元以上不满二十\\
                    五万元、二十五万元以上的,分别依照刑法第二百一十四条规定的各法定刑幅\\
                    度定罪处罚。\\
                    \qquad 销售金额和未销售货值金额分别达到不同的法定刑幅度或者均达到同一法\\
                    定刑幅度的,在处罚较重的法定刑或者同一法定刑幅度内酌情从重处罚。
                    } \\
                    \hline
 
                5 & \makecell[l]{最高人\\民检察\\院、公\\安部关\\于公安\\机关管\\辖的刑\\事案件\\立案追\\诉标准\\的规定\\(二)\\(2010\\)} & 司法解释 & \makecell[l]{
                    (本篇法规被《关于印发《最高人民检察院 公安部关于公安机关管辖的刑事案\\
                    件立案追诉标准的规定(二)》的通知(2022)》(2022年4月6日发布;2022年\\
                    5月15日实施)废止)\\
                    \\
                    第七十条\quad 【销售假冒注册商标的商品案(刑法第二百一十四条)】销售明知\\
                    是假冒注册商标的商品,涉嫌下列情形之一的,应予立案追诉:\\
                    \qquad (一)销售金额在五万元以上的;\\
                    \qquad (二)尚未销售,货值金额在十五万元以上的;\\
                    \qquad (三)销售金额不满五万元,但已销售金额与尚未销售的货值金额合计在\\
                    十五万元以上的。
                } \\
                \hline
              \end{longtable}
            }

            \begin{sloppy}
            \section{\textbf{案例}}
            \subsection{被告人王某某、王某某1犯销售假冒注册商标的商品罪}
            \subsubsection*{(2017)辽0603刑初424号}
            
\paragraph{基本案情}

2015年11月,王某某在浙江省温州柳市镇长虹五金城内从被告人王某某1处订购了价值人民币11万元的假冒ABB品牌断路器,王某某1明知王某某订购假冒断路器系在工程项目上使用的情况下,仍为其组织货源,并于2015年11月至2016年1月间,从他人手中购买假冒ABB品牌断路器,再以人民币11万元价格销售给王某某。王某某收到上述假冒ABB品牌断路器后在丹东市振兴区进行组装,并将组装后剩余2万余元假冒ABB品牌断路器退给王某某1。王某某将组装完成配电箱用于河北未来石配电箱安装工程后,上述假冒ABB品牌断路器被扣押。经ABB(中国)有限公司认定,涉案断路器系假冒“ABB”注册商标的产品,所涉型号对应的正品市场销售价格共计人民币424981.7元。

\paragraph{指控与证明犯罪}

辽宁省丹东市振兴区人民检察院指控:2015年9月1日,被告人王某某在丹东市振兴区某建筑维修有限公司与山东某某某某集团有限公司代表董某某签订了一份总价值为人民币2289000元的配电箱安装工程施工合同,双方约定所用断路器需为ABB(中国)有限公司石家庄分公司生产。其中报价时王某某经询问ABB(中国)有限公司石家庄分公司销售经理后,将断路器价格以正品价格60\%进行报价。

对上述指控,公诉机关提供了相应的证据,认为被告人王某某,王某某1销明知是假冒注册商标的商品,其中王某某销售金额巨大、王某某1销售数据较大,其行为均触犯《中华人民共和国刑法》第二百一十四条之规定,犯罪事实清楚,证据确实、充分,应当以销售假冒注册商标的商品罪追究其刑事责任。鉴于被告人王某某1犯罪以后自动投案,如实供述自己的罪行,是自首,应当依照《中华人民共和国刑法》第六十七条第一款对其处罚;被告人王某某犯罪以后如实供述自己的罪行,是坦白,应当依照《中华人民共和国刑法》第六十七条第三款对其处罚。

\paragraph{辩护意见采纳情况}
被告人王某某辩解称其实际销售的金额不是424981.7元是20多万元。其辩护人提出:1、对被告人王某某量应当按照其实际销售金额25万元的数额量刑;2、被告人王某某指认了同案犯,并带侦查人员到温州去协助抓获王某某1,虽然没有当场抓获王某某1,但也应当认定其有立功行为。

被告人王某某1自愿认罪,无辩解。其辩护人提出:1、现有证据不足以认定ABB(中国)有限公司出具的“鉴定证明”中针对的商品就是被告人王某某1销售的商品;2、ABB(中国)有限公司非国家认可的有鉴定资质的鉴定机构,其出具的“鉴定证明”的性质仅是被害人陈述或证人证言,本案无其他客观证据证实犯罪事实,证据不充分、不确实;被告人王某某1犯罪数额相对较小,有自首情节,认罪态度较好,本次犯罪是初犯偶犯,请求对其单处罚金。

关于被告人王某某的辩护人提出王某某协助抓捕王某某1,虽然没有当场抓获王某某1,但也应当认定其有立功行为的辩护意见,没有法律依据,本院不予采纳。

关于被告人王某某1的辩护人提出现有证据不足以认定ABB(中国)有限公司出具的“鉴定证明”中针对的商品就是被告人王某某1销售的商品的辩护意见,经查,被告人王某某供述其只从被告人王某某1处购买了假冒的ABB品牌产品,且公安机关在德州亚太集团有限公司登记封存了王某某购买的假冒ABB品牌产品,被告人王某某对公安机关登记保存的假冒ABB品牌产品数量无异议,ABB(中国)有限公司出具的“鉴定证明”确定的数量与扣押数量一致,因此,辩护人的该辩护意见没有事实依据,本院不予采纳。

关于王某某1的辩护人提出ABB(中国)有限公司非国家认可的有鉴定资质的鉴定机构,其出具的“鉴定证明”的性质仅是被害人陈述或证人证言,本案无其他客观证据证实犯罪事实,证据不充分、不确实的辩护意见。本院认为,ABB(中国)有限公司出具的鉴定证明并非刑事诉讼证据证据中的鉴定意见,其内容应归类为被害单位陈述。本案中,除ABB(中国)有限公司出具的鉴定证明外,相关证人的证言及被告人王某某、王某某1的供述,均能证实两被告人销售的产品系假冒的ABB品牌产品,辩护人认为本案证据不充分、不确实的辩护意见没有事实依据,本院不予采纳。 

对王内阁、王某某1的辩护人其他辩护意见中的合理部分予以采纳。

\paragraph{判决与量刑}

本院认为,被告人王某某、王某某1销售明知是假冒注册商标的商品,其中王某某销售数额巨大,王某某1销售金额较大,上述两名被告人的行为均已构成销售假冒注册商标的商品罪,应予刑罚。公诉机关指控的事实及罪名成立,适用法律正确,应予支持。被告人王某某1犯罪以后自动投案并如实供述自己的罪行,有自首情节,应依法从轻处罚;被告人王某某犯罪以后如实供述自己的罪行,有坦白情节,可依法从轻处罚。

综上,根据被告人王某某、王某某1的犯罪事实、情节及社会危害程度,依照《中华人民共和国刑法》第二百一十四条、第六十七条第一款、第六十七条第三款、第七十二条第一款、第三款、第七十三条第二款、第三款、《最高人民法院、最高人民检察院关于办理侵犯知识产权刑事案件具体应用法律若干问题的解释》第二条、第九条第一款、第十二条第一款、《最高人民法院关于适用〈中华人民共和国刑事诉讼法〉的解释》第三百六十五条第二款之规定,判决如下:

一、被告人王某某犯销售假冒注册商标的商品罪,判处有期徒刑三年,缓刑五年,并处罚金人民币十万元(罚金已缴纳。缓刑考验期限从判决确定之日起计算)。

二、被告人王某某1犯销售假冒注册商标的商品罪,判处有期徒刑一年三个月,缓刑二年,并处罚金人民币十万元(罚金已缴纳。缓刑考验期限从判决确定之日起计算)。

三、在案扣押的涉案假冒ABB品牌的产品由扣押机关依法处理。



\subsection{卜凡朋销售假冒注册商标的商品罪一案}
\subsubsection*{(2019)皖1302刑初958号}
\paragraph{基本案情}
被告人卜凡朋,男,1985年1月10日出生于江苏省徐州市,汉族,大专文化,徐州斯睿尔机电科技有限公司实际负责人,住徐州市。因涉嫌犯销售假冒注册商标的商品罪于2018年11月13日被宿州市公安局刑事拘留,同年12月19日被宿州市人民检察院批准逮捕,当日由宿州市公安局执行逮捕。现羁押于宿州市看守所。

\paragraph{指控与证明犯罪}
宿州市埇桥区人民检察院指控:2017年宿州钱营孜低热值煤发电工程EPC总承包项目部(该项目部是由中国电力建设工程咨询有限公司和中国电力工程顾问集团华东电力设计院有限公司共同设立)与上海青润仪表科技有限公司签订热控仪表成套设备订货合同,包括宿州钱营孜低热值煤发电工程EPC总承包项目部向上海青润仪表科技有限公司订购“BARTEC”品牌(中文商标博太科)PSB-26型自调控伴热带5500米,金额82.50万元。2017年8月,上海青润仪表科技有限公司与徐州斯睿尔机电科技有限公司签订采购“BARTEC”品牌(中文名称博太科)PSB-26型自调控伴热带5500米购买合同,金额为47.85万元,并约定由出售方徐州斯睿尔机电科技有限公司直接发货至宿州钱营孜电厂。2018年1月12日,徐州斯睿尔机电科技有限公司实际负责人被告人卜凡朋将标志有“BARTEC”自调控伴热带送至宿州钱营孜低热值煤发电工程EPC总承包项目部。博太科防爆设备(上海)有限公司认定被告人卜凡朋销售给宿州钱营孜低热值煤发电工程EPC总承包项目部的5500米“BARTEC”品牌PSB-26型自调控伴热带电缆系假冒“BRATEC”商品。

针对上述指控,公诉机关当庭出示了被告人供述、证人证言、工业品买卖合同、发货清单、付款单据、商标注册证、现场照片及鉴定报告等相关证据予以证实,认为被告人卜凡朋的行为已构成销售假冒注册商标的商品罪,销售金额巨大,建议适用《中华人民共和国刑法》第二百一十四条的规定,对被告人卜凡朋在有期徒刑三年六个月至五年幅度内量刑,并处罚金。



\paragraph{辩护意见采纳情况}
被告人卜凡朋对公诉机关指控其犯销售假冒注册商标的商品罪无异议,但称其销售的电缆中只有部分是假冒的,不应认定为销售金额巨大。

辩护人的辩护意见是被告人卜凡朋销售的电缆已经使用,且其中含有正品,单纯依据鉴定报告不能确定销售金额,公诉机关指控被告人卜凡朋销售假冒注册商标的商品金额巨大,属事实不清,证据不足,应当认定被告人卜凡朋销售金额较大;另,被告人卜凡朋具有自首情节,依法可从轻或减轻处罚,建议对被告人卜凡朋在有期徒刑三年以下量刑,并适用缓刑。


\paragraph{判决与量刑}
本院认为:被告人卜凡朋销售明知是假冒注册商标的商品,销售金额人民币478500元,数额巨大,其行为已构成销售假冒注册商标的商品罪。公诉机关指控的罪名成立,本院予以支持。关于被告人卜凡朋及其辩护人所称公诉机关指控被告人卜凡朋销售金额巨大属事实不清,证据不足的辩解和辩护意见,经查,证人王某、吴某、李某1、孙某2、张某1、孙某1等人的证言、书证热控仪表成套设备订货合同、工业品买卖合同、发货清单、青润公司付款单据、商标注册证等相关书证及鉴定报告可相互印证,足以证明被告人卜凡朋销售到宿州钱营孜低热值煤发电工程项目的5500米博太科PSB-26电缆均系假冒注册商标的商品及销售金额为478500元的事实,对此亦有被告人卜凡朋的供述予以证实。被告人卜凡朋及其辩护人的上述辩解和辩护意见不成立,本院不予采纳。关于辩护人所称被告人卜凡朋具有自首情节的辩护意见,经查,宿州市公安局出具的案发及到案经过,虽能证实被告人卜凡朋系该局电话传唤到案,但被告人卜凡朋到案后未能如实供述自己的罪行,其行为不符合自首的构成要件。对辩护人的上述辩护意见,本院不予采纳。依照《中华人民共和国刑法》第二百一十四条、第六十一条之规定,判决如下:

被告人卜凡朋犯销售假冒注册商标的商品罪,判处有期徒刑四年,罚金人民币二十五万元。



\subsection{陈忠销售假冒注册商标的商品罪}
\subsubsection*{(2020)浙0381刑初313号}
\paragraph{基本案情}
被告人陈忠,男,1986年11月13日出生,汉族,大专文化程度,无职业,户籍所在地浙江省温州市龙湾区。因涉嫌销售假冒注册商标的商品罪,于2019年4月12日被平阳县公安局取保候审,于2020年2月20日被瑞安市人民检察院决定取保候审,于2020年3月25日被本院取保候审。

2018年7月以来,被告人陈忠从广东省深圳市华强北明通数码城郑伟海(另案处理)处购进假冒“”商标的手机、ipad电源适配器、数据线等,并在“函函小城”淘宝店铺上予以销售。期间,被告人陈忠将“”电源适配器、数据线各5个以120元的价格销售给金某(浙江平阳籍人)。2019年4月12日,公安人员在温州市龙湾区被告人陈忠家中将其查获,现场查扣尚未销售的假冒“”商标的电源适配器及数据线、“”商标贴纸等物品若干。期间,被告人陈忠销售假冒“”商标的商品金额共计21万余元,非法获利2万余元。 

另查明,“”商标的注册号为第6281379号,注册有效期限2010年4月28日至2020年4月27日,商标注册人为苹果公司,核定使用商品为第9类,含电适配器、电线、电缆等。


\paragraph{指控与证明犯罪}
瑞安市人民检察院以瑞检刑诉[2020]4496号起诉书指控被告人陈忠犯销售假冒注册商标的商品罪,于2020年3月23日向本院提起公诉。本院当日以简易程序立案,依法组成合议庭,于2020年3月31日公开开庭审理了本案。瑞安市人民检察院指派检察员杨奔出庭支持公诉,被告人陈忠到庭参加诉讼。现已审理终结。


\paragraph{辩护意见采纳情况}
审查起诉期间,被告人陈忠通过在线视频方式自愿具结认罪认罚,公诉机关提出“对被告人陈忠判处有期徒刑一年六个月,并处罚金,退赃后可适用缓刑”的量刑建议。
审理过程中,被告人退出赃款20000元。(要退还未交来)
本院认为,“”注册商标依法经我国商标局核准注册,且在有效期内,受法律保护。被告人陈忠销售明知是假冒注册商标的商品,销售金额数额较大,其行为已触犯刑律,构成销售假冒注册商标的商品罪。公诉机关指控的罪名成立。被告人陈忠归案后能如实供述自己的罪行,可以依法从轻处罚;自愿认罪认罚,可以依法从宽处理。根据被告人陈忠的犯罪情节和悔罪表现,没有再犯罪的危险,宣告缓刑对所居住社区没有重大不良影响,依法可宣告缓刑。公诉机关对被告人的量刑建议适当,本院予以采纳。


\paragraph{判决与量刑}
为严明国法,惩罚犯罪,保护知识产权,维护社会秩序,根据被告人犯罪的事实、性质、情节、社会危害性等,依照《中华人民共和国刑法》第二百一十四条、第六十四条、第六十七条第三款、第七十二条第一款、第三款,《最高人民法院、最高人民检察院关于办理侵犯知识产权刑事案件具体应用法律若干问题的解释》第二条第一款、《最高人民法院、最高人民检察院关于办理侵犯知识产权刑事案件具体应用法律若干问题的解释(二)》第四条,《中华人民共和国刑事诉讼法》第十五条之规定,判决如下:

一、被告人陈忠犯销售假冒注册商标的商品罪,判处有期徒刑一年六个月,缓刑二年,并处罚金人民币100000元。

(缓刑考验期限从判决确定之日起计算,罚金限判决生效之日起十日内缴纳。)

二、被告人陈忠退出的赃款20000元,予以没收,上缴国库。

三、扣押于平阳县公安局的假冒“”适配器及数据线、“”商标贴纸等物品,予以没收。


\subsection{欧某犯销售假冒注册商标的商品罪}
\subsubsection*{(2015)杭西知刑初字第2号}
\paragraph{基本案情}
被告单位杭州泰正科技有限公司,住所地杭州市西湖区文二路385、387号1幢411室。

诉讼代表人欧雪燕,系该公司法定代表人、总经理。

被告人欧某,原系杭州泰正科技有限公司法定代表人。因本案于2013年6月26日被刑事拘留,同年7月24日被取保候审,2014年7月24日被再次取保候审。

2013年1月至5月期间,杭州泰正科技有限公司在杭州市西湖区文二路385号1幢411室,销售假冒“SIEMENS”、“西门子”注册商标的工业自动化产品,销售金额共计人民币172583.3元。2013年6月25日公安机关查获该案,从该公司及仓库内查扣欲销售的前连接线、总线连接线、模块、通讯卡、电池模块、电源、导轨、延长线、MPI电缆、Profibus电缆等工控产品及销售合同、采购合同若干。上述工控产品经西门子(中国)有限公司鉴定,确认均为假冒“SIEMENS”、“西门子”注册商标的商品,货值金额共计人民币470020.73元。被告单位杭州泰正科技有限公司退出违法所得人民币20000元。

\paragraph{指控与证明犯罪}
杭州市西湖区人民检察院以西检公诉刑诉(2015)14号起诉书指控被告人欧某犯销售假冒注册商标的商品罪,于2015年1月5日向本院提起公诉,于2015年1月26日以西检公诉刑追诉(2015)1号追加起诉决定书指控被告单位杭州泰正科技有限公司犯销售假冒注册商标的商品罪向本院追加起诉。本院依法适用简易程序并组成合议庭,公开开庭审理了本案。杭州市西湖区人民检察院检察员朱思思、被告人欧某及其辩护人朱卫永到庭参加诉讼。现已审理终结。




\paragraph{判决与量刑}
本院认为,被告单位杭州泰正科技有限公司以营利为目的,销售明知是假冒注册商标的商品,销售金额数额较大,其行为已构成销售假冒注册商标的商品罪。被告人欧某系被告单位杭州泰正科技有限公司直接负责的主管人员,决策、实施了销售假冒注册商标的商品行为,其行为已构成销售假冒注册商标的商品罪。公诉机关的指控成立。被告单位杭州泰正科技有限公司、被告人欧某自愿认罪,且已退出违法所得,酌情予以从轻处罚。根据被告人欧某的犯罪情节和悔罪表现,其没有再犯罪的危险,且适用缓刑对所居住社区没有重大不良影响,故对其宣告缓刑。据此,依照《中华人民共和国刑法》第二百一十四条、第三十条、第二百二十条、第七十二条第一、三款、第七十三条第二、三款、第五十二条、第五十三条、第六十四条和《最高人民法院、最高人民检察院关于办理侵犯知识产权刑事案件具体应用法律若干问题的解释》第二条第一款、《最高人民法院、最高人民检察院关于办理侵犯知识产权刑事案件具体应用法律若干问题的解释(二)》第四条、《最高人民法院关于适用财产刑若干问题的规定》第二条第一款之规定,判决如下: 

一、被告单位杭州泰正科技有限公司犯销售假冒注册商标的商品罪,判处罚金人民币九万元(罚金限判决生效后十日内缴纳);

二、被告人欧某犯销售假冒注册商标的商品罪,判处有期徒刑一年,缓刑二年,并处罚金人民币九万元(缓刑考验期自判决确定之日起计算。罚金限判决生效后十日内缴纳);

三、暂存于本院的被告单位杭州泰正科技有限公司违法所得人民币二万元予以没收,上缴国库;扣押于杭州市公安局西湖区分局的假冒注册商标的商品予以没收。



\subsection{苏州市倍利控自动化科技有限公司、沈卢政犯销售假冒注册商标的商品罪}
\subsubsection*{(2019)苏0505刑初562号}

\paragraph{基本案情}
“SIEMENS”(第G637074号)注册商标的所有人是西门子股份公司,商标核定使用在第9类电子技术和电子仪器、电缆和导线等商品上,注册有效期自2015年3月31日至2025年3月31日。

被告人沈卢政为被告单位苏州市倍利控自动化科技有限公司的实际经营人。被告单位苏州市倍利控自动化科技有限公司从他处购进假冒“SIEMENS”注册商标的导轨、电缆、连接器、扩展模块、总线接头等产品后,在明知系假冒产品的情况下,于2017年7月至2018年3月期间,多次向无锡汇聪智能科技有限公司销售,销售金额共计人民币101360元。

2018年4月至5月,公安机关在被告人沈卢政居住地的车库及居住场所查获假冒“SIEMENS”商标的导轨、电缆、连接器、扩展模块、总线接头等商品若干,货值金额共计人民币63万余元。

\paragraph{指控与证明犯罪}
苏州市虎丘区人民检察院指控:被告单位苏州市倍利控自动化科技有限公司的实际经营人沈卢政从他处购进假冒“SIEMENS”、“西门子”注册商标的导轨、电缆、连接器、扩展模块等产品,后于2017年7月至2018年3月期间,多次向无锡汇聪智能科技有限公司销售,销售金额共计人民币101360元。公安机关在其车库及居住场所查获假冒“SIEMENS”商标的连接器、电缆线等产品若干,货值金额共计人民币63万余元。公诉机关认为,被告单位及被告人的行为构成销售假冒注册商标的商品罪,系犯罪未遂,可比照既遂犯从轻或减轻处罚,其犯罪以后自动投案,系自首,归案后能如实从述自己的罪行,可从轻处罚。被告单位及被告人自愿如实供述自己的罪行,承认指控的犯罪事实,愿意接受处罚,可从宽处理。为证实上述指控的事实及公诉意见,公诉人当庭讯问了被告单位及被告人,并宣读和出示了相关证据材料。

\paragraph{辩护意见采纳情况}
被告单位苏州市倍利控自动化科技有限公司、被告人沈卢政对公诉机关指控的犯罪事实均不持异议,并当庭自愿认罪。

被告人沈卢政的辩护人提出的辩护意见是:对于公诉机关指控的销售假冒注册商标的商品罪不持异议,同时对起诉书中所认定的被告人系自首、如实供述以及犯罪未遂、认罪认罚等相关情节,请法庭根据被告人的认罪态度以及实际情况,适用缓刑。对于辩护人的相应辩护意见,本院予以采纳。

被告单位苏州市倍利控自动化科技有限公司、被告人沈卢政犯罪以后自动投案,如实供述了自己的罪行。在苏州市虎丘区人民检察院审查起诉阶段,被告单位苏州市倍利控自动化科技有限公司、被告人沈卢政自愿认罪,并签署了认罪认罚具结书,其并在庭审中自愿认罪。在本案审理过程中,被告单位苏州市倍利控自动化科技有限公司、被告人沈卢政分别主动缴纳了罚金保证金各15万元。


\paragraph{判决与量刑}
本院认为,被告单位苏州市倍利控自动化科技有限公司向他人销售明知是假冒他人注册商标的商品,非法经营数额较大,其行为已构成销售假冒注册商标的商品罪,应当判处罚金。另外,对于公安机关查获的尚未销售的假冒注册商标的商品,属于犯罪未遂,其货值金额达到二十五万元以上,故应当依照刑法第二百一十四条规定的销售金额数额巨大的法定刑幅度定罪处罚,因该部分为犯罪未遂,故可以比照既遂犯减轻处罚。根据规定,销售金额和未销售货值金额分别达到不同的法定刑幅度的,在处罚较重的法定刑幅度内酌情从重处罚。被告人沈卢政作为公司的直接责任人员,其行为构成销售假冒注册商标的商品罪,根据上述分析,应当判处三年以上七年以下有期徒刑,并处罚金,因系犯罪未遂,故对其比照既遂犯予以减轻处罚。被告单位、被告人犯罪以后自动投案,如实供述了自己的罪行,系自首,依法可从轻处罚。被告单位、被告人自愿如实供述自己的罪行,承认指控的犯罪事实,愿意接受处罚,可以从宽处理。被告单位及被告人在本案审理过程中都能主动交纳罚金保证金并且认罪态度较好,对其可酌情从轻处罚。根据被告人沈卢政归案后的悔罪表现,对其适用缓刑没有再犯罪危险,故决定对其宣告缓刑。对于辩护人的相应辩护意见,本院予以采纳。公诉机关指控被告单位苏州市倍利控自动化科技有限公司、被告人沈卢政犯销售假冒注册商标的商品罪的事实清楚,证据确实、充分,指控的罪名及提请从轻处罚的理由成立,本院予以采纳。综上,依照《中华人民共和国刑法》第二百一十四条、第二百二十条、第六十四条、第六十七条第一款、第七十二条第一款、第三款、第七十三条第二款、第三款以及《最高人民法院、最高人民检察院关于办理侵犯知识产权刑事案件具体应用法律若干问题的解释》第二条第二款、第十二条第一款、第二款、《最高人民法院、最高人民检察院关于办理侵犯知识产权刑事案件具体应用法律若干问题的解释(二)》第四条,《中华人民共和国刑事诉讼法》第十五条之规定,判决如下:

一、被告单位苏州市倍利控自动化科技有限公司犯销售假冒注册商标的商品罪,判处罚金人民币十五万元。

(已缴纳的罚金保证金予以折抵罚金,上缴国库。)

二、被告人沈卢政犯销售假冒注册商标的商品罪,判处有期徒刑两年两个月,缓刑三年,并处罚金人民币十五万元。

(缓刑考验期限自判决确定之日起计算,已缴纳的罚金保证金予以折抵罚金,上缴国库。)

三、已被公安机关扣押的假冒注册商标的导轨、电缆、连接器、扩展模块、总线接头等予以没收。

\subsection{于某某销售假冒注册商标的商品罪}
\subsubsection*{(2020)鲁0105刑初645号}
\paragraph{基本案情}
被告人于某某,男,住河北省河间市,因涉嫌犯销售假冒注册商标的商品罪于2020年8月17日被取保候审。

\paragraph{指控与证明犯罪}
公诉机关指控:2010年12月25日,被告人于某某以山东某公司济南供应站名义与济南某有限公司签订工业品买卖合同,约定买卖某某牌电缆。2011年1月初,被告人于某某以1154592元的价格购进假冒“某某”注册商标的电缆,销售给济南某有限公司,销售金额共计1257175元。2020年8月17日,被告人于某某主动到公安机关投案。认为被告人于某某的行为构成销售假冒注册商标的商品罪,且具有自首的量刑情节,建议判处被告人三年以下有期徒刑,可适用缓刑,并处罚金。

\paragraph{辩护意见采纳情况}
本院认为,公诉机关指控被告人于某某犯销售假冒注册商标的商品罪的事实清楚,证据确实、充分,指控罪名成立,量刑建议适当,应予采纳。采纳辩护人提出的对被告人于某某判处缓刑的意见。被告人于某某认罪认罚,且具有自首的情节,对其可以从轻处罚。

\paragraph{判决与量刑}
依照《中华人民共和国刑法》第二百一十四条、第六十七条第一款、第七十二条第一款、第三款、《中华人民共和国刑事诉讼法》第十五条、《最高人民法院、最高人民检察院关于办理侵犯知识产权刑事案件具体应用法律若干问题的解释》第二条的规定,判决如下:

被告人于某某犯销售假冒注册商标的商品罪,判处有期徒刑三年,缓刑三年,并处罚金二十万元(缓刑考验期限,从判决确定之日起计算。罚金已缴纳)。
            \end{sloppy}
    }

\end{document}
