\documentclass[a4paper]{ctexart}

\usepackage{amsmath}
\usepackage{amssymb}
\usepackage[UTF8, scheme=plain]{ctex}
\usepackage{fancyhdr}
\usepackage{gensymb}
\usepackage[margin=1in]{geometry}
\usepackage{hyperref}
\usepackage{indentfirst} % 首行缩进
\usepackage{lastpage}
\usepackage{ragged2e}
\usepackage{tikz}
\usepackage{upgreek}
\usepackage{verbatim}
\usepackage{xassoccnt}

\usetikzlibrary{calc}

\date{} % not display time here

\fancyhf{}
\pagestyle{fancy} %fancyhdr宏包新增的页面风格

\begin{document}
    % whether show solutions 
    \newif\ifshowSolution
    % \showSolutiontrue
    \showSolutionfalse

    % 不显示自动编号 且 能自动生成目录
    \setcounter{secnumdepth}{0}
    % 行距
    \renewcommand{\baselinestretch}{1.5}

    \title{\heiti\zihao{2} 数学错题集}
    \maketitle
    \thispagestyle{empty} % 标题页不显示页码
    \centerline{开始时间:}
    \centerline{结束时间:}

    要求:
    
    1. 保证正确率;提高做题速度;

    2. 规范答题,书写有条理;保持卷面整洁,少涂改;

    3. 除号尽量使用分数形式;

    4. 能用公式的计算题用公式。

    \vfill
    \centerline{\date{\today}}

    \setlength{\parindent}{2em}
    \newpage

    % 目录 罗马数字页码
    \setcounter{page}{1}
    \pagenumbering{Roman}
    \cfoot{\thepage}

    % foot decorative lines
    % \renewcommand{\footrulewidth}{1pt}
    % \fancyhead[R]{\textbf{《鲁迅全集》笔记}}

    \hypersetup{colorlinks=true, linktocpage=true}
    \tableofcontents

    % 正文 阿拉伯数字页码
    \newpage
    \setcounter{page}{1}
    \pagenumbering{arabic}
    % 当前页 of 总页数
    \cfoot{Page \thepage\ of \pageref{LastPage}}

    \zihao{4}

    \begin{sloppy}
        \begin{enumerate}
            \section{幂的运算}

\item {
    若$a,b$是正整数,且满足 $2^a + 2^a + 2^a + 2^a = 2^b\cdot 2^b\cdot 2^b\cdot 2^b\cdot$,则 $a$与$b$ 的关系是?\\
    \ifshowSolution
    \fangsong\zihao{4}
    \\
    思路: 

    解答:
    \else
        \\ \\ \\
    \fi
}

\item {
    $\bigstar\bigstar\bigstar\bigstar$
    (用科学记数法)一个正方体集装箱的棱长为 $0. 8 \rm{m}$. \\
    % (1) 这个集装箱的体积是多少?(用科学记数法)\\
    (2) 若有一个小立方块的棱长为$2\times 10^{-3} $ m, 则需要多少个这样的小立方块才能将集装箱装满?
    \ifshowSolution
    \fangsong\zihao{4}
    \\
    思路: 问题(2)注意简便运算. 

    解答:
    \begin{align*}
        \frac{0. 8^3} {(2\times 10^{-3})^3} &= \frac{0. 8^3} {8\times 10^{-9}}\\
        &= \frac{0. 064} {\frac{1}{10^9}}\\
        &= 0. 064\times 10^9\\
        &= 6. 4\times 10^{-2}\times 10^9\\
        &= 6. 4\times 10^7\\
    \end{align*}
    \else
        \\ \\ \\
    \fi
}

\begin{comment}
\item {
    当 $ x=7, y=-\frac{1}{7}$ 时, 求 $x^{4n+1}\cdot y^{4n+2}$ ($n$为整数)的值. 
    \ifshowSolution
        \fangsong\zihao{4}
        \\
        思路: 观察到$xy=-1$, 让将$x$和$y$凑成一对, 相乘. 直接将$x, y$的值代入表达式中进行计算. 

        解答: 
        \begin{align*}
            \mbox{原式} &= 7^{4n+1}\cdot \left(-\frac{1}{7}\right) ^{4n+2}\\
            &= [7\times(-\frac{1}{7})]^{4n+1} \cdot(-\frac{1}{7})\\
            &= (-1)^{4n+1} \cdot(-\frac{1}{7})\\
            &= \frac{1}{7}. 
        \end{align*}
    \else
        \\ \\ \\
    \fi
}

\item {
    已知$ m=8^9, n=9^8 $, 用含$m, n$的式子表示 $72^{72}$. 
    \ifshowSolution
        \fangsong\zihao{4}
        \\
        思路: 观察到$72=8\times 9$, 将原式中的72分解, 凑出$m, n$. 
        
        解答: 
        \begin{align*}
            \mbox{原式} &= (8\times 9)^{72}\\
            &= 8^{72}\times 9^{72}\\
            &= (8^9)^8\times (9^8)^9\\
            &= m^8 n^9. 
        \end{align*}
    \else
        \\ \\ \\
    \fi
}

\item {
    已知$x-y=k$, 求$(3x-3y)^3. $
    \ifshowSolution
        \fangsong\zihao{4}
        \\
        解答: 
        \begin{align*}
            \mbox{原式} &= [3(x-y)]^3\\
            &= 27(x-y)^3\\
            &= 27k^3. 
        \end{align*}
    \else
        \\ \\ \\
    \fi
}

\item {
    若$(a^nb^mb)^3 = a^9 b^{15}$, 求$2^{m+n}$. 
    \ifshowSolution
        \fangsong\zihao{4}
        \\
        思路: 先将左边化简, 再与右边比较, 解出$m, n$. 
        
        解答: 
        \begin{align*}
            (a^nb^{m+1})^3 &= a^9b^{15}\\
            a^{3n}b^{3m+3} &= a^9b^{15}
        \end{align*}
        $\therefore 3n=9, 3m+3=15$\\
        $\therefore n=3, m=4$
        \begin{align*}
            2^{m+n} &= 2^7\\
            &= 128. 
        \end{align*}
    \else
        \\ \\ \\
    \fi
}

    \item {
        化简: $(-a-b)^{2n}$ ($n$为整数). 
    }
    \\ \\ \\
    \item {
        化简: $(-a-b)^{2n+1}$ ($n$为整数). 
    }
    \\ \\ \\

    \item {
        (用科学计数法表示)已知 1 nm = 0. 000000001 m, 则 15 nm 等于多少 m?
        \ifshowSolution
        \fangsong\zihao{4}
        \\
        解答: 

        \textcircled{1} 写出换算关系
        \begin{align*}
            1 \rm{nm} &= 10^{-9} \rm{m}
        \end{align*}
        \textcircled{2} 两边同时乘15
        \begin{align*}
            15 \rm{nm} &= 15 \times 10^{-9} \rm{m}\\
            &= 1. 5\times 10^{-8} \rm{m}. 
        \end{align*}
        \fi
    }
    \\ \\ \\

    \item {
        (用科学计数法表示)肥皂泡表面厚度大约是 0. 0007 mm, 换算成以米为单位是多少?
    \ifshowSolution
    \fangsong\zihao{4}
    \\
    解答: 

    \textcircled{1} 写出换算关系
    \begin{align*}
        1 \rm{mm} &= 10^{-3} \rm{m}
    \end{align*}
    \textcircled{2} 两边同时乘0. 0007
    \begin{align*}
        0. 0007 \rm{mm} &= 0. 0007 \times 10^{-3} \rm{m}\\
        &= 7\times 10^{-7} \rm{m}. 
    \end{align*}
    \fi
    }
    \\ \\ \\

\item {
    (用科学计数法表示)已知 $0. 25 \upmu$m $ = 2. 5\times 10^{-7}$m, 那么 1 m 等于多少$\upmu$m?
    \ifshowSolution
        \fangsong\zihao{4}
        \\
        思路: 将题中给出的换算关系两边同时除以 $2. 5\times 10^{-7}$, 右边就出现了 1m. 

        解答: 
        \begin{align*}
            \frac{0. 25}{2. 5\times 10^{-7}} \rm{\upmu m} &= 1\rm{m}\\
            \frac{2. 5\times 0. 1}{2. 5\times \frac{1}{10^7}} \rm{\upmu m} &= 1\rm{m}\\
            0. 1\times 10^7 \rm{\upmu m} &= 1\rm{m}\\
            10^6 \rm{\upmu m} &= 1\rm{m}\\
            1\rm{m} &= 10^6 \rm{\upmu m}. 
        \end{align*}
    \else
        \\ \\ \\
    \fi
}

    \item {
        若多项式$ 9x^2 - mx+16$是一个完全平方式, 则 $m$的值是多少?
    }
    \\ \\ \\

\item {
    已知$a^2+b^2=8, a-b=3$, 求$ab$的值. 
    \ifshowSolution
        \fangsong\zihao{4}
        \\
        思路: 看到$a^2+b^2, a-b, ab$, 应该想到完全平方公式. 
    \else
        \\ \\ \\
    \fi
}

    \item {
        若$x^2+mx+9$是完全平方式, 求常数$m$的值. 
    }
    \\ \\ \\

    \item {
        若$x+y=2$, 求代数式$x^2-y^2+4y$的值. 
    }
    \\ \\ \\

\item {
    (注意: 除号使用分数形式) 已知$10^{-m}=a, 10^{-n}=b$($m, n$是整数), 求$10^{2m-3n}$的值(用含有$a, b$的代数式表示). 
    \\ \\ \\
}

\item {
    已知$2^x=3, 2^y=6, 2^z=12$, 判断下列有关$x, y, z$的数量关系式的对错. \\
    (1) $x+z=2y$\\
    (2) $x+y+3=2z$\\
    (3) $4x=z$\\
    (4) $x+1=y$
    \\ \\
}

\item {
    计算: $ (\frac{1}{2})^{-1} + \lvert 2-\pi \rvert $
    \ifshowSolution
    \fangsong\zihao{4}
    \\
    思路: 去绝对值符号, 运算到底. 

    解答: 
    \begin{align*}
        \mbox{原式} &= 2 + \pi - 2\\
        &= \pi. 
    \end{align*}
    \else
        \\ \\ \\
    \fi
}

\item {
    已知$(x+2)^{x+5}=1$, 求$x$. 
    \\ \\ \\
}

    \item {
        (把 $\frac{1}{27}$ 化为以3为底的幂) 若$3^{x-1}=\frac{1}{27}$, 求$x$. 
        \\ \\ \\
    }
    
\item {
    (注意: 除号使用分数形式) 已知$a^{2n}=3, a^{3m}=5$, 求$a^{6n-9m}$. 
    \ifshowSolution
    \fangsong\zihao{4}
    \\
    思路: 将$a^{6n-9m}$凑出$a^{2n}, a^{3m}$, 直接代入计算. 结果使用分数形式即可. 

    解答: 
    \begin{align*}
        a^{6n-9m} &= \frac{a^{6n}}{a^{9m}}\\
        &= \frac{(a^{2n})^3} {(a^{3m})^3}\\
        &= \frac{3^3} {5^3}\\
        &= \frac{27} {125}. 
    \end{align*}
    \else
        \\ \\ \\
    \fi
}

    \item {
        已知$3\cdot2^x + 2^{x+1}=40$, 求$x$. 
        \ifshowSolution
        \fangsong\zihao{4}
        \\
        思路: 将左边的2个$2^{x}$整理到一起. 
    
        解答: 
        \begin{align*}
            3\cdot2^x + 2^{x+1} &= 40\\
            3\cdot2^x + 2\cdot 2^{x} &= 40\\
            5\cdot2^x &= 40\\
            2^x &= 8\\
            \therefore x = 3. 
        \end{align*}
        \else
            \\ \\ \\
        \fi
    }
\end{comment}
            \section{整式乘法}

\item {
    $\bigstar \bigstar$
    已知$y=\frac13 [(x - 1)^2 + (x - 3)^2 + (x - 2)^2]$, 当$x$等于多少时, $y$的值最小? $y$的最小值是多少? 
    \ifshowSolution
        \fangsong\zihao{6}
        \\
        正解: $y=x^2 - 4x + \frac{14}{3} = (x-2)^2+\frac23, y_{min}=\frac23$.
    \else
        \\ \\ \\ \\ 
    \fi
}

\item {
    $\bigstar$
    定义:$L(A)$ 是多项式A化简后的项数,例如多项式$A=x^2+2x-3$,则$L(A)=3$. 一个多项式$A$乘多项式$B$化简得到多项式C(即$C=A\times B$),如果$L(A)\leq L(C)\leq L(A)+1$, 则称$B$是$A$的``好多项式'',如果$L(A)=L(C)$, 则称$B$是$A$的``极好多项式''。若$A=x-3, B=x^2-ax+9$均是关项式$x$的多项式,且$B$是$A$的“极好多项式”,则$a$等于多少?
    \ifshowSolution
        \fangsong\zihao{6}
        \\
        % 思路: 
    \else
        \\ \\ \\ \\ 
    \fi
}


\begin{comment}


\item {
    已知$\frac1m + m = 3$,则$\frac{1}{m^2} + m^2$的值是多少?
    \ifshowSolution
        \fangsong\zihao{6}
        \\
        正解: 7.
    \else
        \\ \\ \\ \\ 
    \fi
}

\item {
    若 $\lvert x+y-4 \rvert + (xy-3)^2 = 0$, 计算$x^2 + y^2$. 
    \ifshowSolution
        \fangsong\zihao{6}
        \\
        正解: 10. 
    \else
        \\ \\ \\
    \fi
}

\item {
    $\bigstar\bigstar$
    求代数式 $-m^2 + 4m + 8$ 的最值, 并判断是最大值还是最小值. 
    \ifshowSolution
        \fangsong\zihao{6}
        \\
        正解: 12. 
    \else
        \\ \\ \\
    \fi
}

\item {
    $\bigstar$
    若多项式 $2x^2 + kx - 14$ 是由整式 $x-2$ 与另一个整式 $2x+m$相乘得到的, 求k的值. 
    \ifshowSolution
        \fangsong\zihao{6}
        \\
        正解:3.
    \else
        \\ \\ \\
    \fi
}

\item {
    $\bigstar \bigstar$
    观察下列各式, 写出第n个等式. 

    第\textcircled{1}个等式: $1\times 5 + 4 = 3^2;$ \\
    第\textcircled{2}个等式: $3\times 7 + 4 = 5^2;$ \\
    第\textcircled{3}个等式: $5\times 9 + 4 = 7^2; \\ \cdots $ \\
    \ifshowSolution
        \fangsong\zihao{6}
        \\
        思路: 先找到等式中第1列的数字的规律, 再找下一列数的规律, 最后写出公式. 第一列数是奇数1,3,5,可以用n表示为 $2n-1$. 第二列数比第一列大4,所以是$2n+3$. 依此类推. 

        正解: 
        $(2n-1)\times (2n+3) + 4 = (2n+1)^2$. 
    \else
        \\ \\ \\
    \fi
}

\item {
    $\bigstar$
    多项式 $a^2 + 4$ 加上一个单项式后, 可化为一个多项式的平方, 求这个单项式. 
    \ifshowSolution
    \fangsong\zihao{6}
    \\
    思路: 想到完全平方公式 $(x + y)^2 = x^2 + 2xy + y^2$. 

    正解: 
    $a^2$对应公式里的$x^2$, $4$对应公式里的$y^2$, 所以
    \[\left\{ 
        \begin{array}{lc}
            x^2 = a^2 \\
            y^2 = 4
        \end{array}
    \right. \]
    所以,
    \[\left\{ 
        \begin{array}{lc}
            x = \pm a \\
            y = \pm 2
        \end{array}
    \right. \]
    所以,$2xy = \pm 4a. $
    \else
        \\ \\ \\ \\ \\ \\
    \fi
}

\item {
    如果 $x^2 + (m-1)x + 16$是一个完全平方式展开后的结果,那么常数$m$的值是多少? 
    \ifshowSolution
        \fangsong\zihao{6}
        \\
        正解: -7或9. 
    \else
        \\ \\ \\
    \fi
}

\item {
    $\bigstar$
    用公式计算: $(a-2b+3c)\cdot (a+2b+3c)$. 
    \ifshowSolution
    \fangsong\zihao{6}
    \\
    思路: 先把原式整理为平方差公式的形式,再用公式. 

    正解:
    \begin{align*}
        \mbox{原式} &= (a+3c-2b)\cdot (a+3c+2b) \\
        &= (a+3c)^2 - 4b^2 \\
        &= a^2 + 6ac + 9c^2 - 4b^2. \\
    \end{align*}

    \else
        \\ \\ \\
    \fi
}

\item {
    $\bigstar$
    用若干A类正方形(边长a)、B类长方形(长a, 宽b)、C类正方形(边长b), 拼成一个边长为 $2a + 2b$ 的正方形, 需要B类正方形多少个? 
    \ifshowSolution
        \fangsong\zihao{6}
        \\
        思路: 画图. 
    \else
        \\ \\ \\ \\ \\ \\ \\ \\
    \fi
}

\item {
    $y=x^2 + 6x - 2$. 当$x$等于多少时, $y$的值最小? $y$的最小值是多少? 
    \ifshowSolution
        \fangsong\zihao{6}
        \\
        正解: -11. 
    \else
        \\ \\ \\ \\ 
    \fi
}

\item {
    $y=x^2 + 16x + 3$. 当$x$等于多少时, $y$的值最小? $y$的最小值是多少? 
    \ifshowSolution
        \fangsong\zihao{6}
        \\
        正解:-61. 
    \else
        \\ \\ \\ \\ 
    \fi
}

\item {
    $y=x^2 - 16x + 3$. 当$x$等于多少时, $y$的值最小? $y$的最小值是多少? 
    \ifshowSolution
        \fangsong\zihao{6}
        \\
        正解:-61. 
    \else
        \\ \\ \\ \\ 
    \fi
}

\item {
    先化简,再求值:$(2x+y)(2x-y) + (x-y)^2 - (10x^2y - 2xy^2)\div (2y)$. 其中$x=-4, y=\frac12$.
    \ifshowSolution
        \fangsong\zihao{6}
        \\
        正解:2.
    \else
        \\ \\ \\ \\ 
    \fi
}

\item {
    已知代数式$a^2 + (m+2)ab+ 16b^2$ 是一个完全平方式, 则有理数$m$ 的值是多少. 
    \ifshowSolution
        \fangsong\zihao{6}
        \\
        正解: $6$或$-10$. 
    \else
        \\ \\ \\ \\ 
    \fi
}

\item {
    已知多项式$x-a$与$x^2 - 2x + 1$的乘积的结果中不含 $x^2$ 项. 求常数$a$的值. 
    \ifshowSolution
        \fangsong\zihao{6}
        \\
        正解: $-2$.
    \else
        \\ \\ \\ \\ 
    \fi
}

\item {
    若$(3z^3 + M)(2z^2 - 1)$是一个五次多项式, 则下列说法中正确的是(\quad). 

    A. M是一个三次单项式

    B. M是一个三次多项式

    C. M的次数不高于三

    D. M不可能是一个常数
    \ifshowSolution
        \fangsong\zihao{6}
        \\
        % 思路: 直接把$x=2y+2$ 代入,可以消去字母. 
    \else
        \\ \\ \\ \\ 
    \fi
}
    
\item {
    若 $x=2y+2$, 则求 $x^2 - 4xy + 4y^2$ 的值. 
    \ifshowSolution
        \fangsong\zihao{6}
        \\
        思路: 直接把$x=2y+2$ 代入,可以消去字母. 
    \else
        \\ \\ \\ \\ 
    \fi
}
    
\item {
    $\bigstar$
    写出 $(a+b)^2, (a-b)^2, ab$ 三者之间的等量关系. 
    \ifshowSolution
        \fangsong\zihao{6}
        \\
        思路: 直接相减. 

        正解:
        \begin{align*}
            (a+b)^2 - (a-b)^2 &= a^2 + 2ab + b^2 - (a^2 - 2ab + b^2) \\
            &= 4ab. 
        \end{align*}
    \else
        \\ \\ \\
    \fi
}

\item {
    已知公式: $(x-1)(x^n + x^{n-1} + \cdots + x + 1) = x^{n+1} - 1$ (n为正整数). 
    
    利用上述公式, 求 $2^{100} + 2^{99} +\cdots + 2^2 + 2$ 的值. 
    \ifshowSolution
        \fangsong\zihao{6}
        \\
        思路: 使用公式前, 必须把原式的形式整理得和公式完全一致. 

        正解:记原式为S(这句话必须写,让别人知道S是什么). 
        \begin{align*}
            S + 1 &= 2^{100} + 2^{99} + \cdots + 2^2 + 2 + 1 \\
            &= (2-1)\times (2^{100} + 2^{99} + \cdots + 2^2 + 2 + 1) \\
            &= 2^{101} - 1 \\
            \therefore
            S &= 2^{101} - 2. 
        \end{align*}
    \else
        \\ \\ \\
    \fi
}

\item {
    $\bigstar \bigstar$
    用公式计算: $(x+2y-3z)^2$. 
    \\ \\ \\
}

\item {
    $\bigstar$
    (用2种方法)若把代数式$x^2-4x-5$化成$(x-m)^2+k$的形式, 其中$m, k$为常数. 求$m+k$的值. 
    \ifshowSolution
        \fangsong\zihao{6}
        \\
        思路1: 把$x^2-4x-5$转化为$(x-m)^2+k$的形式, 求出m和k. 

        正解: 
        \begin{align*}
            x^2-4x-5 &= x^2 - 2\times 2x + 4 - 9 \\
            &= (x-2)^2 - 9 \\
        \end{align*}
        所以, 
        \[\left\{ 
            \begin{array}{lc}
                m = 2\\
                k =-9
            \end{array}
        \right. \]
        $\therefore m+k=-7. $
        \\
        \begin{tikzpicture}
            \draw[dashed] (0, 0) -- (15, 0);
        \end{tikzpicture}
        \\
        思路2: 对应项系数,常数项相等。
        \begin{align*}
            (x-m)^2+k &= x^2 - 2m\cdot x + m^2 + k \\
        \end{align*}
        所以, 
        \[\left\{ 
            \begin{array}{lc}
                4 = 2m\\
                -5 = m^2 + k
            \end{array}
        \right. \]
        所以,
        \[\left\{ 
            \begin{array}{lc}
                m = 2\\
                k =-9
            \end{array}
        \right. \]
        $\therefore m+k=-7. $
    \else
        \\ \\ \\
    \fi
}

\item {
    $a=2^{44}, b=3^{33}, c=5^{22}$, 比较$a, b, c$的大小. 
    \ifshowSolution
    \fangsong\zihao{6}
    \\
    思路: 把指数化为一样, 比较底数大小. 

    正解: 
    \begin{align*}
        a &= (2^4)^{11} = 16^{11}\\
        b &= (3^3)^{11} = 27^{11}\\
        c &= (5^2)^{11} = 25^{11}\\
        & 16^{11} < 25^{11} < 27^{11}\\
        &\therefore a < c < b. 
    \end{align*}
    \else
        \\ \\ \\
    \fi
}

\item {
    已知$100^a=20, 1000^b=50$, 则$a+\frac{3}{2}b-\frac{3}{2}$的值是多少? 
    \ifshowSolution
    \fangsong\zihao{6}
    \\
    思路: 观察$100^a, 1000^b$ 发现 $a, b$都出现在指数上, 要求$a+\frac{3}{2}b-\frac{3}{2}$的值, 应该想到尝试把$a+\frac{3}{2}b-\frac{3}{2}$放在指数上. 

    正解: 
    \begin{align*}
        100^{a+\frac{3}{2}b-\frac{3}{2}} &= \frac{100^a\cdot 100^{\frac{3}{2}b}}{100^\frac{3}{2}}\\
        &= \frac{100^a\cdot 10^{2\cdot \frac{3}{2}b}}{10^{2\cdot\frac{3}{2}}}\\
        &= \frac{100^a\cdot 10^{3b}} {10^{3}}\\
        &= \frac{100^a\cdot 1000^{b}} {1000}\\
        &= \frac{20\times 50} {1000}\\
        &= 1\\
        &\therefore a+\frac{3}{2}b-\frac{3}{2} = 0. 
    \end{align*}
    \else
        \\ \\ \\
    \fi
}

\item {
    用多项式乘法法则推导完全平方公式. 
    \\ \\ \\
}

\item {
    计算 $(-x-y)^2$. 
    \ifshowSolution
    \fangsong\zihao{6}
    \\
    思路: 先将括号里的负号处理掉, 再用公式. 

    正解: 
    \begin{align*}
        \mbox{原式} &= (x+y)^2\\
        &= x^2 +2xy + y^2
    \end{align*}
    \else
        \\ \\ \\
    \fi
}

\item {
    判断下列各式的正误. 

    \textcircled{1}$(a-b)^2 = a^2 - b^2$\\
    \textcircled{2}$(-2a-3b)^2 = 4a^2 - 12ab +9b^2$\\
    \textcircled{3}$(\frac13 m + \frac12 n)^2 = \frac19 m^2 + \frac16 mn + \frac14 n^2$\\
    \textcircled{4}$(-y-3)^2 = y^2 + 6y + 9$
    \\ \\ \\
}

\item {
    用公式计算. 

    $(1) (-\frac34 x + \frac43 y)^2$ \\ \\

    $(2) (-7m-2n)^2$ \\ \\
    
    $(3) (a+b+c)^2$ \\ \\
    \\ \\
}

\item {
    计算: $(-xy^2)\cdot (x^2y - 6xy)$
    \\ \\ \\
}

\item {
    计算: $(a+3)(a-1) + a(a-2)$
    \\ \\ \\
}

\item {
    计算(用公式): $(x+2y-1)\cdot (x+2y+1)$
    \\ \\ \\
}

\item {
    $\bigstar \bigstar$
    已知多项式 $x-2a$ 与 $x^2+x-1$ 的乘积中不含 $x^2$ 项, 则常数$a$的值是多少? 
    \ifshowSolution
    \fangsong\zihao{6}
    \\
    思路: 不含 $x^2$ 项的意思即该项的“系数为0”. 展开后合并同类项. 

    正解:
    \begin{align*}
        (x-2a)x^2+x-1 &= x^3 + (1-2a)x^2-(1+2a)x + 2a\\
        1-2a &= 0 \\
        a &= \frac12. 
    \end{align*}

    \else
        \\ \\ \\
    \fi
}

\item {
    多项式 $4a^2+9$ 加上一个单项式后, 可化为一个多项式的平方, 求这个单项式. 
    \\ \\ \\
}

\item {
    已知 $a^2+a-1=0$, 求 $a^3 + 2a^2 + 2023$ 的值. 
    \\ \\ \\
}

\item {
    观察``杨辉三角'', 计算 $(a+b)^5$ 的结果中, 项 $a^3b^2$的系数, 
    \[
    \begin{matrix}
        & & & & 1 & & & \\
        & & & 1 & & 1 & & \\
        & & 1 & & 2 & & 1 & \\
        & 1 & & 3 & & 3 & & 1 \\
        1 & & 4 & & 6 & & 4 & & 1
    \end{matrix}
    \]
    \begin{align*}
        (a+b)^1 &= a+b \\
        (a+b)^2 &= a^2 + 2ab + b^2 \\
        (a+b)^3 &= a^3 + 3a^2b + 3ab^2 + b^3 \\
        (a+b)^4 &= a^4 + 4a^3b + 6a^2b^2 + 4ab^3 + b^4 \\
    \end{align*}
    \ifshowSolution
        \fangsong\zihao{6}
        \\
        % 思路: 画图. 
    \else
        \\ \\ \\
    \fi
}

\item {
    判断 $49^8 - 14^2\times 7^{12}$ 能否被9整除, 并说明理由. 
    \ifshowSolution
        \fangsong\zihao{6}
        \\
        % 思路: . 
    \else
        \\ \\ \\
    \fi
}

\item {
    求代数式 $y^2 +10y + 27$ 的最小值. 
    \ifshowSolution
        \fangsong\zihao{6}
        \\
        思路: 将原式整理成 $(y + m)^2 + k$ 的形式; 再利用 $a \geq 0$ 的性质,计算出原式的最小值. 
    \else
        \\ \\ \\
    \fi
}

\item {
    已知 $a^2 + a - 3 = 0$, 计算$(a^2 - 3)(a+1)$. 
    \ifshowSolution
        \fangsong\zihao{6}
        \\
        % 思路:
    \else
        \\ \\ \\
    \fi
}

\item {
    在计算 $(ax+1)(2x+b)$时,小泉同学看错了$b$的值, 计算结果为$2x^2 + 6x + 4$; 小张同学看错了$a$ 的值,计算结果为 $4x^2+12x+5$. 

    (1) 求 $a, b$. \\
    (2) 计算 $(ax+1)(2x+b)$ 的正确结果. 
    \ifshowSolution
        \fangsong\zihao{6}
        \\
        思路: 小泉同学算出的a是正确的, 小张同学算出的b是正确的. 
    \else
        \\ \\ \\ \\ 
    \fi
}
\end{comment}

            \section{图形的变换}

\item {
    如图,已知线段AB,求作等腰直角三角形,使其斜边等于线段AB,保留作图痕迹并证明.\\
    \begin{tikzpicture}
        % 带端点标注的线段
        \draw[|-|] (0,0) -- (4,0) 
            node[left] at (0,0) {A}
            node[right] at (4,0) {B};
    \end{tikzpicture}
    \ifshowSolution
        \fangsong\zihao{4}
        \\
        思路: 先满足等腰,再满足直角.
    \else
        \\ \\ \\ \\ \\
    \fi
}

\item {
    (尺规作图)如图,已知线段AB,作线段AB的垂直平分线.\\
    \begin{tikzpicture}
        % 带端点标注的线段
        \draw[|-|] (0,0) -- (4,4) 
            node[left] at (0,0) {A}
            node[right] at (4,4) {B};
    \end{tikzpicture}
    \ifshowSolution
        \fangsong\zihao{4}
        \\
        % 思路: 
    \else
        \\ \\ \\ \\
    \fi
}

\item {
    如图,$P$ 为 $\angle AOB$ 内一点,$OA$ 垂直平分线段 $PP_1$, $OB$垂直平分线段 $PP_2$,连接$P_1P_2$ 交$OA$ 于点 $N$,若$P_1P_2=15$,求 $\triangle PMN$ 的周长.\\
    \begin{tikzpicture}[scale=1.2]
    % 绘制角AOB
    \coordinate (O) at (0,0);
    \coordinate (A) at (4,4);
    \coordinate (B) at (4,0);
    \draw (O) -- (A);
    \draw (O) -- (B);
    
    % 标注点P
    \coordinate (P) at (3,2);
    \coordinate (P_1) at (2,3);
    \coordinate (P_2) at (3, -2);
    \draw (P) -- (P_1);
    \draw (P) -- (P_2);
    \draw (P_1) -- (P_2);

    \coordinate (M) at (2.17, 2.17);
    \coordinate (N) at (2.6, 0);
    \draw (M) -- (P);
    \draw (N) -- (P);

    
    % 标注所有点
    node[left] at (0,0) {A}
    \fill (O) circle  node[below left] {$O$};
    \fill (A) circle (0pt) node[below left] {$A$};
    \fill (B) circle (0pt) node[below left] {$B$};
    \fill (P) circle (1pt) node[below left] {$P$};
    \fill (P_1) circle (1pt) node[below left] {$P_1$};
    \fill (P_2) circle (1pt) node[below left] {$P_2$};
    \fill (M) circle (0pt) node[below left] {$M$};
    \fill (N) circle (0pt) node[below left] {$N$};
    \end{tikzpicture}
    \ifshowSolution
        \fangsong\zihao{4}
        \\
        % 思路: 
    \else
        \\ \\ \\ \\
    \fi
}
        \end{enumerate}
    \end{sloppy}
\end{document}
