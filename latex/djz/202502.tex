\section{202502}

% \subsection{解答}
\item{
    当 $ x=7, y=-\frac{1}{7}$ 时, 求 $x^{4n+1}\cdot y^{4n+2}$ ($n$为整数)的值.

    \fangsong\zihao{4}
    思路:观察到$xy=-1$,让将$x$和$y$凑成一对,相乘.直接将$x, y$的值代入表达式中进行计算.
    
    解答:
    \begin{align*}
        \mbox{原式} &= 7^{4n+1}\cdot \left(-\frac{1}{7}\right) ^{4n+2}\\
        &= [7\times(-\frac{1}{7})]^{4n+1} \cdot(-\frac{1}{7})\\
        &= (-1)^{4n+1} \cdot(-\frac{1}{7})\\
        &= \frac{1}{7}.
    \end{align*}
} 
% \\ \\
\item{
    已知$ m=8^9, n=9^8 $, 用含$m, n$的式子表示 $72^{72}$.

    \fangsong\zihao{4}
    思路:观察到$72=8\times 9$,将原式中的72分解,凑出$m, n$.
    
    解答:
    \begin{align*}
        \mbox{原式} &= (8\times 9)^{72}\\
        &= 8^{72}\times 9^{72}\\
        &= (8^9)^8\times (9^8)^9\\
        &= m^8 n^9.
    \end{align*}
} 
% \\ \\
\item{
    已知$x-y=k$, 求$(3x-3y)^3.$

    \fangsong\zihao{4}
    解答:
    \begin{align*}
        \mbox{原式} &= [3(x-y)]^3\\
        &= 27(x-y)^3\\
        &= 27k^3.
    \end{align*}
} 
% \\ \\
\item{
    若$(a^nb^mb)^3 = a^9 b^{15}$, 求$2^{m+n}$.

    \fangsong\zihao{4}
    思路:先将左边化简,再与右边比较,解出$m,n$.
    
    解答:
    \begin{align*}
        (a^nb^{m+1})^3 &= a^9b^{15}\\
        a^{3n}b^{3m+3} &= a^9b^{15}
    \end{align*}
    $\therefore 3n=9, 3m+3=15$\\
    $\therefore n=3, m=4$
    \begin{align*}
        2^{m+n} &= 2^7\\
        &= 128.
    \end{align*}
}
% \\ \\
\item{
    化简: $(-a-b)^{2n}$ ($n$为整数).
}
% \\ \\
\item{
    化简: $(-a-b)^{2n+1}$ ($n$为整数).
}
% \\ \\
\item{
    (用科学计数法表示)已知 1 nm = 0.000000001 m, 则 15 nm 等于多少 m?

    \fangsong\zihao{4}
    解答:

    \textcircled{1} 写出换算关系
    \begin{align*}
        1 \rm{nm} &= 10^{-9} \rm{m}
    \end{align*}
    \textcircled{2} 两边同时乘15
    \begin{align*}
        15 \rm{nm} &= 15 \times 10^{-9} \rm{m}\\
        &= 1.5\times 10^{-8} \rm{m}.
    \end{align*}
}
% \\ \\
\item{
    (用科学计数法表示)肥皂泡表面厚度大约是 0.0007 mm,换算成以米为单位是多少?

    \fangsong\zihao{4}
    解答:

    \textcircled{1} 写出换算关系
    \begin{align*}
        1 \rm{mm} &= 10^{-3} \rm{m}
    \end{align*}
    \textcircled{2} 两边同时乘0.0007
    \begin{align*}
        0.0007 \rm{mm} &= 0.0007 \times 10^{-3} \rm{m}\\
        &= 7\times 10^{-7} \rm{m}.
    \end{align*}
}
% \\ \\
\item{
    (用科学计数法表示)已知 $0.25 \upmu$m $ = 2.5\times 10^{-7}$m,那么 1 m 等于多少$\upmu$m?

    \fangsong\zihao{4}
    思路:将题中给出的换算关系两边同时除以 $2.5\times 10^{-7}$,右边就出现了 1m.

    解答:
    \begin{align*}
        \frac{0.25}{2.5\times 10^{-7}} \rm{\upmu m} &= 1\rm{m}\\
        10^6 \rm{\upmu m} &= 1\rm{m}\\
        1\rm{m} &= 10^6 \rm{\upmu m}.
    \end{align*}
}
% \\ \\
\item{
    若多项式$ 9x^2 - mx+16$是一个完全平方式,则 $m$的值是多少?
}
% \\ \\
\item{
    已知$a^2+b^2=8, a-b=3$,求$ab$的值.

    \fangsong\zihao{4}
    思路:看到$a^2+b^2, a-b, ab$, 应该想到完全平方公式.
}
% \\ \\
\item{
    若$x^2+mx+9$是完全平方式,求常数$m$的值.
}
% \\ \\
\item{
    若$x+y=2$,求代数式$x^2-y^2+4y$的值.
}
% \\ \\
\item{
    若把代数式$x^2-4x-5$化成$(x-m)^2+k$的形式,其中$m,k$为常数. 求$m+k$的值.

    \fangsong\zihao{4}
    思路:把$(x-m)^2+k$转化为$ax^2+bx+c$的形式,再比较各项系数,解出$m,k$.\\
    注意:系数的位置.

    解答:
    \begin{align*}
        (x-m)^2+k &= x^2-2mx+m^2+k
    \end{align*}
    与$x^2-4x-5$比较,对应项的系数相等,常数项相等,得到方程
    \[\left\{ 
        \begin{array}{lc}
            2m = 4\\
            m^2+k=-5
        \end{array}
    \right.\]
    解得
    \[\left\{ 
        \begin{array}{lc}
            m = 2\\
            k =-9
        \end{array}
    \right.\]
    $\therefore m+k=-7.$
}
% \\ \\
\item{
    计算:$(x+2y-3z)^2$.
}
% \\ \\
\item{
    已知$10^{-m}=a, 10^{-n}=b$($m, n$是整数),求$10^{2m-3n}$的值(用含有$a, b$的代数式表示).
}
% \\ \\
\item{
    已知$2^x=3, 2^y=6, 2^z=12$,判断下列有关$x, y, z$的数量关系式的对错.\\
    (1) $x+z=2y$\\
    (2) $x+y+3=2z$\\
    (3) $4x=z$\\
    (4) $x+1=y$
}
% \\ \\
\item{
    计算:$ (\frac{1}{3})^{-1} + \lvert 3-\pi \rvert $
    
    \fangsong\zihao{4}
    思路:去绝对值符号,运算到底.

    解答:
    \begin{align*}
        \mbox{原式} &= 3 + \pi - 3\\
        &= \pi.
    \end{align*}
}
% \\ \\
\item{
    已知$(x+2)^{x+5}=1$, 求$x$.
}
% \\ \\
\item{
    (用科学记数法)一个正方体集装箱的棱长为 $0.8 \rm{m}$.\\
    (1) 这个集装箱的体积是多少?(用科学记数法)\\
    (2) 若有一个小立方块的棱长为$2\times 10^{-3} $ m, 则需要多少个这样的小立方块才能将集装箱装满?

    \fangsong\zihao{4}
    思路:问题(2)注意简便运算.
}
% \\ \\ \\