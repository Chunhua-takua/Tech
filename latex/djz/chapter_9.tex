\section{图形的变换}

\item {
    如图,已知线段AB,求作等腰直角三角形,使其斜边等于线段AB,保留作图痕迹并证明.\\
    \begin{tikzpicture}
        % 带端点标注的线段
        \draw[|-|] (0,0) -- (4,0) 
            node[left] at (0,0) {A}
            node[right] at (4,0) {B};
    \end{tikzpicture}
    \ifshowSolution
        \fangsong\zihao{4}
        \\
        思路: 先满足等腰,再满足直角.
    \else
        \\ \\ \\ \\ \\
    \fi
}

\item {
    (尺规作图)如图,已知线段AB,作线段AB的垂直平分线.\\
    \begin{tikzpicture}
        % 带端点标注的线段
        \draw[|-|] (0,0) -- (4,4) 
            node[left] at (0,0) {A}
            node[right] at (4,4) {B};
    \end{tikzpicture}
    \ifshowSolution
        \fangsong\zihao{4}
        \\
        思路: 
    \else
        \\ \\ \\ \\
    \fi
}

\item {
    如图,$P$ 为 $\angle AOB$ 内一点,$OA$ 垂直平分线段 $PP_1$, $OB$垂直平分线段 $PP_2$,连接$P_1P_2$ 交$OA$ 于点 $N$,若$P_1P_2=15$,求 $\triangle PMN$ 的周长.\\
    \begin{tikzpicture}[scale=1.2]
    % 绘制角AOB
    \coordinate (O) at (0,0);
    \coordinate (A) at (4,4);
    \coordinate (B) at (4,0);
    \draw (O) -- (A);
    \draw (O) -- (B);
    
    % 标注点P
    \coordinate (P) at (3,2);
    \coordinate (P_1) at (2,3);
    \coordinate (P_2) at (3, -2);
    \draw (P) -- (P_1);
    \draw (P) -- (P_2);
    \draw (P_1) -- (P_2);

    \coordinate (M) at (2.17, 2.17);
    \coordinate (N) at (2.6, 0);
    \draw (M) -- (P);
    \draw (N) -- (P);

    
    % 标注所有点
    node[left] at (0,0) {A}
    \fill (O) circle  node[below left] {$O$};
    \fill (A) circle (0pt) node[below left] {$A$};
    \fill (B) circle (0pt) node[below left] {$B$};
    \fill (P) circle (1pt) node[below left] {$P$};
    \fill (P_1) circle (1pt) node[below left] {$P_1$};
    \fill (P_2) circle (1pt) node[below left] {$P_2$};
    \fill (M) circle (0pt) node[below left] {$M$};
    \fill (N) circle (0pt) node[below left] {$N$};
    \end{tikzpicture}
    \ifshowSolution
        \fangsong\zihao{4}
        \\
        思路: 
    \else
        \\ \\ \\ \\
    \fi
}