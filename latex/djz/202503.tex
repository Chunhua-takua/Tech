\section{202503}

\item{
    $a=2^{44}, b=3^{33}, c=5^{22}$, 比较$a, b, c$的大小.
    \iffalse
    \fangsong\zihao{4}
    思路: 把指数化为一样,比较底数大小.

    解答: 
    \begin{align*}
        a &= (2^4)^{11} = 16^{11}\\
        b &= (3^3)^{11} = 27^{11}\\
        c &= (5^2)^{11} = 25^{11}\\
        & 16^{11} < 25^{11} < 27^{11}\\
        &\therefore a < c < b.
    \end{align*}
    \fi
}
\\ \\ \\
\item{
    已知$100^a=20, 1000^b=50$, 则$a+\frac{3}{2}b-\frac{3}{2}$的值是多少?
     \iffalse
     \fangsong\zihao{4}
     思路: 观察$100^a, 1000^b$ 发现 $a,b$都出现在指数上, 要求$a+\frac{3}{2}b-\frac{3}{2}$的值, 应该想到尝试把$a+\frac{3}{2}b-\frac{3}{2}$放在指数上.
 
     解答: 
     \begin{align*}
         100^{a+\frac{3}{2}b-\frac{3}{2}} &= \frac{100^a\cdot 100^{\frac{3}{2}b}}{100^\frac{3}{2}}\\
         &= \frac{100^a\cdot 10^{2\cdot \frac{3}{2}b}}{10^{2\cdot\frac{3}{2}}}\\
         &= \frac{100^a\cdot 10^{3b}} {10^{3}}\\
         &= \frac{100^a\cdot 1000^{b}} {1000}\\
         &= \frac{20\times 50} {1000}\\
         &= 1\\
         &\therefore a+\frac{3}{2}b-\frac{3}{2} = 0.
     \end{align*}
     \fi
}
\\ \\ \\
\item{
    若$3^{x-1}=\frac{1}{27}$, 求$x$.
}
\\ \\ \\
\item{
    已知$a^{2n}=3, a^{3m}=5$, 求$a^{6n-9m}$.
    \iffalse
    \fangsong\zihao{4}
    思路: 将$a^{6n-9m}$凑出$a^{2n}, a^{3m}$, 直接代入计算. 结果使用分数形式即可.

    解答: 
    \begin{align*}
        a^{6n-9m} &= \frac{a^{6n}}{a^{9m}}\\
        &= \frac{(a^{2n})^3} {(a^{3m})^3}\\
        &= \frac{3^3} {5^3}\\
        &= \frac{27} {125}.
    \end{align*}
    \fi
}
\\ \\ \\
\item{
    已知$3\cdot2^x + 2^{x+1}=40$, 求$x$.
        \iffalse
        \fangsong\zihao{4}
        思路: 将左边的2个$2^{x}$整理到一起.
    
        解答: 
        \begin{align*}
            3\cdot2^x + 2^{x+1} &= 40\\
            3\cdot2^x + 2\cdot 2^{x} &= 40\\
            5\cdot2^x &= 40\\
            2^x &= 8\\
            \therefore x = 3.
        \end{align*}
        \fi
}
\\ \\ \\
\item{
    用多项式乘法法则推导完全平方公式.
}
\\ \\ \\
\item{
    计算 $(-x-y)^2$.
}
\\ \\ \\
\item{
    判断下列各式的正误.

    $(a-b)^2 = a^2 - b^2$\\
    $(-2a-3b)^2 = 4a^2 - 12ab +9b^2$\\
    $(\frac13 m + \frac12 n)^2 = \frac19 m^2 + \frac16 mn + \frac14 n^2$\\
    $(-y-3)^2 = y^2 + 6y + 9$
}
\\ \\ \\
\item{
    用完全平方公式计算.

    $(1) (-\frac34 x + \frac43 y)^2$ \\ \\
    $(2) (-7m-2n)^2$ \\ \\
    $(3) (a+b+c)^2$ \\ \\
}
\\ \\ \\