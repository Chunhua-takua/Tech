\section{202503}

\item{
    $a=2^{44}, b=3^{33}, c=5^{22}$, 比较$a, b, c$的大小.
    \ifshowSolution
    \fangsong\zihao{4}
    \\
    思路: 把指数化为一样,比较底数大小.

    解答: 
    \begin{align*}
        a &= (2^4)^{11} = 16^{11}\\
        b &= (3^3)^{11} = 27^{11}\\
        c &= (5^2)^{11} = 25^{11}\\
        & 16^{11} < 25^{11} < 27^{11}\\
        &\therefore a < c < b.
    \end{align*}
    \fi
    \unless\ifshowSolution
    \\ \\ \\
    \fi
}

\item{
    已知$100^a=20, 1000^b=50$, 则$a+\frac{3}{2}b-\frac{3}{2}$的值是多少?
    \ifshowSolution
    \fangsong\zihao{4}
    \\
    思路: 观察$100^a, 1000^b$ 发现 $a,b$都出现在指数上, 要求$a+\frac{3}{2}b-\frac{3}{2}$的值, 应该想到尝试把$a+\frac{3}{2}b-\frac{3}{2}$放在指数上.

    解答: 
    \begin{align*}
        100^{a+\frac{3}{2}b-\frac{3}{2}} &= \frac{100^a\cdot 100^{\frac{3}{2}b}}{100^\frac{3}{2}}\\
        &= \frac{100^a\cdot 10^{2\cdot \frac{3}{2}b}}{10^{2\cdot\frac{3}{2}}}\\
        &= \frac{100^a\cdot 10^{3b}} {10^{3}}\\
        &= \frac{100^a\cdot 1000^{b}} {1000}\\
        &= \frac{20\times 50} {1000}\\
        &= 1\\
        &\therefore a+\frac{3}{2}b-\frac{3}{2} = 0.
    \end{align*}
    \fi
    \unless\ifshowSolution
    \\ \\ \\
    \fi
}

\item{
    (把 $\frac{1}{27}$ 化为以3为底的幂) 若$3^{x-1}=\frac{1}{27}$, 求$x$.
}
\\ \\ \\

\item{
    (注意: 除号使用分数形式) 已知$a^{2n}=3, a^{3m}=5$, 求$a^{6n-9m}$.
    \ifshowSolution
    \fangsong\zihao{4}
    \\
    思路: 将$a^{6n-9m}$凑出$a^{2n}, a^{3m}$, 直接代入计算. 结果使用分数形式即可.

    解答: 
    \begin{align*}
        a^{6n-9m} &= \frac{a^{6n}}{a^{9m}}\\
        &= \frac{(a^{2n})^3} {(a^{3m})^3}\\
        &= \frac{3^3} {5^3}\\
        &= \frac{27} {125}.
    \end{align*}
    \fi
    \unless\ifshowSolution
    \\ \\ \\
    \fi
}

\item{
    已知$3\cdot2^x + 2^{x+1}=40$, 求$x$.
    \ifshowSolution
    \fangsong\zihao{4}
    \\
    思路: 将左边的2个$2^{x}$整理到一起.

    解答: 
    \begin{align*}
        3\cdot2^x + 2^{x+1} &= 40\\
        3\cdot2^x + 2\cdot 2^{x} &= 40\\
        5\cdot2^x &= 40\\
        2^x &= 8\\
        \therefore x = 3.
    \end{align*}
    \fi
    \unless\ifshowSolution
    \\ \\ \\
    \fi
}

\item{
    用多项式乘法法则推导完全平方公式.
}
\\ \\ \\

\item{
    计算 $(-x-y)^2$.
    \ifshowSolution
    \fangsong\zihao{4}
    \\
    思路: 先将括号里的负号处理掉, 再用公式.

    解答: 
    \begin{align*}
        \mbox{原式} &= (x+y)^2\\
        &= x^2 + y^2
    \end{align*}
    \fi
    \unless\ifshowSolution
    \\ \\ \\
    \fi
}

\item{
    判断下列各式的正误.

    $(a-b)^2 = a^2 - b^2$\\
    $(-2a-3b)^2 = 4a^2 - 12ab +9b^2$\\
    $(\frac13 m + \frac12 n)^2 = \frac19 m^2 + \frac16 mn + \frac14 n^2$\\
    $(-y-3)^2 = y^2 + 6y + 9$
}
\\ \\ \\

\item{
    用完全平方公式计算.

    $(1) (-\frac34 x + \frac43 y)^2$ \\ \\

    $(2) (-7m-2n)^2$ \\ \\
    
    $(3) (a+b+c)^2$ \\ \\
}
\\ \\

\item{
    多项式 $a^2 + 4$ 加上一个单项式后, 可化为一个多项式的平方, 求这个单项式.(列出所有结果)
    \ifshowSolution
    \fangsong\zihao{4}
    \\
    思路: 想到完全平方公式 $(x\pm y)^2 = x^2 \pm 2xy + y^2$.

    解答: 

    (1) $a^2$对应公式里的$x^2$, $4$对应公式里的$y^2$, 所以$x= \pm a, y= \pm 2$, 所以$2xy = \pm 4a$. 即
    \begin{align*}
        (a+2)^2 = a^2 + 4a + 4 \\
        (a-2)^2 = a^2 - 4a + 4
    \end{align*}
    (2) $4$对应公式里的$x^2$, $a^2$对应公式里的$2xy$, 所以$x= \pm 2, y=\pm \frac{a^2}{4}$, 所以$y^2 = \frac{a^4}{16}$. 即
    \begin{align*}
        (2 + \frac{a^2}{4})^2 = 4 + a^2 + \frac{a^4}{16}
    \end{align*}
    \fi
    \unless\ifshowSolution
    \\ \\ \\
    \fi
}

\item{
    计算: $(-xy^2)\cdot (x^2y - 6xy)$
}
\\ \\ \\

\item{
    计算: $(a+3)(a-1) + a(a-2)$
}
\\ \\ \\

\item{
    计算(用公式): $(x+2y-1)\cdot (x+2y+1)$
}
\\ \\ \\

\item{
    观察下列各式, 写出第n个等式.

    第\textcircled{1}个等式: $1\times 5 + 4 = 3^2;$ \\
    第\textcircled{2}个等式: $3\times 7 + 4 = 5^2;$ \\
    第\textcircled{3}个等式: $5\times 9 + 4 = 7^2; \\ \cdots $ \\
    \ifshowSolution
    \fangsong\zihao{4}
    \\
    思路: 先找到等式中某个位置 (比如第1个) 的数字的规律, 再找下一个位置数的规律, 最后写出公式.
    \fi
    \unless\ifshowSolution
    \\ \\ \\
    \fi
}

\item{
    已知多项式 $x-2a$ 与 $x^2+x-1$ 的乘积中不含 $x^2$ 项, 则常数$a$的值是多少?
    \ifshowSolution
    \fangsong\zihao{4}
    \\
    思路: 不含 $x^2$ 项的意思即该项的“系数为0”.展开后合并同类项.
    \fi
    \unless\ifshowSolution
    \\ \\ \\
    \fi
}

\item{
    多项式 $4a^2+9$ 加上一个单项式后, 可化为一个多项式的平方, 求这个单项式.
}
\\ \\ \\

\item{
    已知 $a^2+a-1=0$, 求 $a^3 + 2a^2 + 2023$ 的值.
}
\\ \\ \\

\item{
    用公式计算: $(a-2b+3c)\cdot (a+2b+3c)$.
}
\\ \\ \\

\item{
    已知公式: $(x-1)(x^n + x^{n-1} + \cdots + x + 1) = x^{n+1} - 1$ (n为正整数).
    
    利用上述公式, 求 $2^{100} + 2^{99} +\cdots + 2^2 + 2$ 的值.
    \ifshowSolution
    \fangsong\zihao{4}
    \\
    思路: 使用公式前, 必须把原式的形式整理得和公式完全一致.
    \fi
    \unless\ifshowSolution
    \\ \\ \\
    \fi
}