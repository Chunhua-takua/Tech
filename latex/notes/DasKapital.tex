\documentclass[a4paper]{article}
\usepackage[UTF8, scheme=plain]{ctex}
% \usepackage[hidelinks]{hyperref}
\usepackage{hyperref}
\hypersetup{
    colorlinks=true,
    linktocpage=true
}
\usepackage[margin=1in]{geometry}
\usepackage{ragged2e}
\usepackage{fancyhdr}
\usepackage{lastpage}
\usepackage{xassoccnt}

% 不显示自动编号 且 能自动生成目录
\setcounter{secnumdepth}
{0}

% 黑体2号
\title{\heiti\zihao{2} 资本论}
\author{TKA}
\date{\today}

\renewcommand\contentsname{目录}

% 行距
\renewcommand{\baselinestretch}
{1.75}

\fancyhf{}
\pagestyle{fancy} %fancyhdr宏包新增的页面风格

\begin{document}
    \maketitle

    \tableofcontents

    \part{第二篇 货币转化为资本}

    \section{第四章  货币转化为资本}
    \subsection{劳动力的买和卖}

    可见,货币所有者要把货币转化为资本,就必须在商品市场上找到自由的工人。这里所说的自由,具有双重意义:一方面,工人是自由人,能够把自己的劳动力当作自己的商品来支配,另一方面,他没有别的商品可以出卖,自由得一无所有,没有任何实现自己的劳动力所必需的东西。





\end{document}