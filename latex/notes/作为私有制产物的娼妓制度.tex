\documentclass[a4paper]{article}

\usepackage[UTF8, scheme=plain]{ctex}
\usepackage{hyperref}
\hypersetup{
    colorlinks=true,
    linktocpage=true
}

\usepackage[
    top = 37mm,
    left = 28mm,
    right = 26mm,
    bottom = 35mm
    ]{geometry}

\usepackage{titlesec}
\usepackage{lastpage}
\usepackage{fancyhdr}
\usepackage{color}
\definecolor{gray}{RGB}{100,100,100}

\usepackage{graphicx}
\usepackage{ulem}
\usepackage{CJKulem}

\date{}
\author{辛若水}
\title{\zihao{-2}\heiti\textbf{作为私有制产物的娼妓制度}}

% 行距
\renewcommand{\baselinestretch}
{1.75}

\begin{document}
    \maketitle

(一)所谓娼妓制度

娼妓制度是私有制的产物,这是马克思主义的基本观点。但是,许多人在考察娼妓制度的时候,却有意无意地忽略这一点。我们讲娼妓制度是私有制的产物,是有三层内涵的,即娼妓制度因为私有制的产生而产生,因为私有制的消灭而消灭,因为私有制的恢复而恢复。所以,我们实在可以把娼妓制度,作为判定一个社会是公有制,还是私有制的重要标志。也就是说,一个社会只要存在着娼妓制度,无论它是非法的、还是合法的,是隐蔽的、还是公开的,都印证了一点即私有制本身的存在。从某种意义上讲,娼妓制度实在把私有制发挥到了极致。因为在娼妓制度,不只依附于私有制,而且人本身亦成为了可以通过金钱进行交换的性商品。亦即娼妓制度,把人本身变成了性商品;当然,在这里,人本身亦不可避免地沦为了性爱动物。可以说,在娼妓制度这里,人本身真正地失落了。但是,我们也应该认识到,必须激烈批判的是娼妓制度,而不是作为弱者的娼妓。其实,人们往往是以道德义愤的态度,来对待娼妓本身的;仿佛不如此,就不足以印证自己所拥有的深沉的道德感。但是,我们看到,在这里道德义愤与赏玩的心态是统一在一起的。实际上,这赏玩的心态早就否定了深沉的道德感。可以这样说,如果我们不能够终结娼妓制度,那就没有理由把道德的污水泼向娼妓。我们可以说,在道德本身同样有恃强凌弱、欺软怕硬的一面。道德本身,在面对娼妓制度的时候,是无可奈何的;它除了鼓吹一些崇高的道德教条之外,并不能动摇作为娼妓制度根源的私有制本身。如果道德本身只会对着弱者喑呜不已,那它也就这点出息了。当然,这样的道德,并不是真诚的道德,相反,却是伪道德。其实,也有一些人想从自然人性的角度来解释娼妓制度。所以,在这里便有一个问题,即娼妓制度是否合乎自然人性?我们知道,娼妓制度是根源于私有制的;如果私有制是合乎自然人性的,那娼妓制度合乎自然人性,似乎也就顺理成章了。其实,有很多哲学在宣扬“人性是自私的”,“人不为己,天诛地灭”,不过,通过这些宣扬,我们就可以认识到一点,即这些哲学实在是用来论证私有制的。人性本身就是自私的,那私有制的合理性就显而易见了。其实,在这里,就是通过自私的人性来引导出私有制本身。但是,我们早就论证过,一个人人自私的社会是无法维系的。其实,即便人性有自私的一面,也不能够讲人性的全部都是自私的,更不能够由之引导出私有制本身。实际上,我们并不能够通过人性来解释私有制;因为私有制实在是历史的产物,虽然在这里,人性也起到了一些作用,但却是比较渺小的,更何况,人性本身同样是为历史所塑造的。亦即,我的观点很明确,私有制度并不能够用人性自私来解释。同样地,我们也不能够用自然人性来解释娼妓制度。亦即娼妓制度,不仅不合乎自然人性,而且以纵欲的方式践踏了自然人性。我说过,娼妓制度把私有制度发挥到了极致。在这里,不只人的身体成为了性商品,而且人的精神同样成为了性商品。亦即在这里,人之身体与精神,皆不为人本身所有。当然,我们也可以用人本身的异化来解释这一点。但是,因为在这里拥有太多的色情色彩,所以很难为人们自觉。可以说,在娼妓制度的背景之下,色情不只具有了意识形态的性质,而且成为意识形态本身。其实,娼妓制度,就是在金钱的基础之上,来建构色情的社会。也就是说,色情的社会同样是由金钱来推动的。如果没有了金钱,色情社会亦将不复存在。那么,我们能不能用欲望的乌托邦来解释色情社会呢?其实,不能够这样讲的。欲望的乌托邦,至少还有审美的性质;但色情社会,就直接通向罪恶了。也就是说,不要把娼妓制度的根源往自然人性那里归结。在我们,不仅要否定娼妓制度,而且要终结娼妓制度;当然,这矛头也不可避免地指向了私有制本身。其实,在私有制成为现实本身的时候,我们只能表达一种迂阔的理想。但是,惟一可以自豪的是,这种迂阔的理想竟然是曾经的现实。

(二)因私有制的产生而产生

我们讲,娼妓制度因私有制的产生而产生,那是不是说有了私有制,便同时产生了娼妓制度呢?虽然在学理上可以这讲,但是却很难诉诸实证。如果讲得谨慎一点,我们可以说娼妓制度是私有制发展到一定阶段的产物。在这里,我们应该承认一点,即娼妓制度有着宗教的起源。也就是说,娼妓制度曾经具有过宗教意义的神圣;但是,很显然这种神圣不合乎自然人性本身。当然,对于娼妓制度与宗教的关系,我们就不再深究了。其实,从神圣的意义上来理解娼妓制度,总不免给人一种滑稽之感,甚至有某种荒诞的色彩;但是,这毕竟是存在本身,所以也没有必要以现代的观点苛责。我们可以这样说,娼妓制度诞生的标志,即是它本身成为国家政治的一部分。所以,中国的娼妓史,亦必然要从管仲写起。为什么呢?因为是管仲第一个把娼妓制度纳入国家政治。虽然他是从经济的角度出发,以游女夜合之资充当国用的,但是,很明显,这本身就是国家政治。我们可以看到,国家政治在对待娼妓制度的时候,既没有考虑自然人性,也没有伦理正义的尺度,而完全是从功利的角度出发。也可以说,在管仲这里,所谓的娼妓制度,不过是富国的手段。我们知道,在娼妓制度本身,不只摧残自然人性,而且最大程度地容纳了罪恶;那它为什么能够为国家政治所容忍,甚至被纳入国家政治本身呢?我想,最深刻的根源,就在于这里的国家政治是以私有制为根基的。我们可以看到,自从娼妓制度被纳入国家政治之后,它就再也不可能独立于国家政治之外了。一方面,娼妓制度是服务于特权阶级的,而特权阶级即是国家政治的掌握者;另一方面,娼妓制度不可避免地影响国家政治,甚至娼妓制度可以成为统治人心、牢笼士人的手段,龚自珍在《京师乐籍论》中就表达了这样的观点。我们可以看到,娼妓业的繁荣会成为太平盛世的点缀。许多故国之思以及对太平盛世的追忆,往往通过对风流艳迹的怀恋表达出来,譬如《桃花扇》中的《哀江南》就是这个样子。似乎人们可以通过娼妓业的繁荣与凋落,觇见一个王朝的盛衰。那么,我们能不能说,娼妓制度最终败坏了国家政治呢?其实,这讲,确实在一定程度上道出了真理本身。我们可以举南朝的例子。有的学者在研究南朝社会的时候,直接称之为“色情社会”。不是说“商女不知亡国恨,隔江犹唱后庭花”吗?其实,不知亡国恨的,何尝是商女?恰恰是那些沉迷于酒色的特权阶级。在这里,我同样要强调一点,是娼妓制度败坏的国家政治,而不是娼妓本身。其实,作为弱者的娼妓,在历史的进程中,不仅留下了美丽的倩影、发展了各门艺术,而且成就了独立的人格,甚至成为了民族气节的代表。可以说,陈寅恪先生的《柳如是别传》即很好地印证了这一点。我们在这里,看一个问题,即娼妓制度的产生,是否让自由的男女关系成为可能?可以说,研究娼妓史,我们必须慎重地对待这个问题。我们知道,在古代社会,由于礼法秩序以及伦理道德的重压,自由的男女关系几乎成为不可能。真正的爱情往往与礼法、伦理道德处于激烈冲突的状态,而且它本身往往为礼法、伦理道德所粉碎,古往今来许多的爱情悲剧都印证了这一点。当然,所谓的婚姻,不过是以家族利益的利益为基础,在这里爱情并不占多么重要的地位。也可以说,在婚姻中,男女关系并不处于自由的状态,相反,双方都背负着伦理道德的重负。那么,男女之间,能不能够拥有一种自由的关系呢?我们看到,在伦理社会的汪洋大海里,却浮现出了青楼文化的海市蜃楼。那么,在青楼文化的背景之下,自由的男女关系是否可能呢?不可否认,在这里,男女双方都可以获得更大的自由,但是,这种自由显然是虚幻的。也可以说,这种自由是以金钱为支撑的,所以也很容易为金钱所粉碎。亦即在青楼文化的背景之下,即便自由的男女关系成为可能,它也只是呈现在海市蜃楼之中,并不具有现实的基础。可以说,在自由的男女关系这里,有着人类理想的表达;但是,它的实现不应该在海市蜃楼中。  

(三)因私有制的消灭而消灭

可以说,娼妓制度因私有制的消灭而消灭,已经为历史本身所印证了。我们知道,娼妓制度是私有制发展到一定阶段的产物,亦即娼妓制度依附于私有制。所以,一旦私有制为公有制所取代,那娼妓制度也就走到了尽头。可以说,在娼妓制度层面,国家政治的最高水平就是消灭娼妓制度本身。当然,在中国,消灭娼妓制度的,就是毛泽东。从管仲把娼妓制度纳入国家政治,到毛泽东利用国家政治消灭娼妓制度本身,这实在是一个辩证的历程;当然,这个辩证的历程,也就是中国娼妓史的全部。我们知道,在管仲把娼妓制度纳入国家政治的时候,既不曾考虑自然人性,也没有伦理正义的尺度,而只是本着功利的目的。可以说,作为国家政治的娼妓制度,不但不曾尊重自然人性,反而扭曲了自然人性,不但不曾表达伦理的正义,反而践踏了伦理的正义;所以,要尊重自然人性,恢复伦理的正义,必然地要求消灭娼妓制度。可以说,消灭娼妓制度,必须有公有制的基础;亦即在私有制的基础上是不可能消灭娼妓制度的。即便从宅心仁厚处讲,私有制亦是天然地渴求娼妓制度的;而娼妓制度亦乐得于私有制狼狈为奸。我们知道,在历史上,不止一次地发生过废娼运动,但是,为什么独独共产主义中国取得了成功呢?当然,这废娼运动,一方面本之于自然人性,另一方面更有伦理正义的表达,但是,却都以失败告终。我想,这最深刻的根源就在于它缺乏公有制的基础。也就是说,历史上的废娼运动,从来不曾改变作为娼妓制度基础的私有制,而只是本着善心,大讲什么教育感化。娼妓制度违反自然人性,践踏伦理正义,这早就为有识之士所洞察。但是,不去触动私有制,那废娼运动也只能半途而废、无果而终。其实,共产主义中国在废娼的时候,就从根本上触动了私有制,并建立了公有制;甚至消灭娼妓制度,已经成为了公有制的重要标志。我们可以看到,在废娼的时候,一方面对妓院保持了政治的高压,另一方面也根绝了娼妓制度所赖以存在的经济基础;更为重要的是,它不仅消灭了娼妓制度,而且完成了对妓女的改造。当然,现在很多人对改造人性、改造人本身,大摇其头;仿佛改造本身实在践踏了人之尊严似的。但是,在我却是认同对人性痼疾以及黑暗因子的改造的。对于为娼妓制度所蹂躏的妓女,我们当然持同情的态度;但是,我们还应该看到另外一点,即娼妓制度实在腐蚀了人的心魂。也就是说,在这里不只自然人性被扭曲了,就是精神本身亦处于畸形的状态。我们可以说,消灭娼妓制度的最深层次的内涵,就是对被扭曲的人性的改造。也就是说,这不是单纯地对妓女的改造,更有对隐蔽着的、更为广大的群体,亦即嫖客的改造。妓女承担了娼妓制度的全部黑暗,而制造这黑暗的恰恰是隐蔽着的、更为广大的群体。如果从终极性来讲,消灭娼妓制度,不只是让娼妓制度成为历史,同时也要让娼妓文化成为历史。但是,很明显,这同样是一种理想的表达。如果娼妓文化真正的成为历史,那我也没有必要在这里做哲学的批判了。当然,现在不少人有这样的观点,即通过国家政治来消灭娼妓制度,只能奏一时之效,而不能够行之久远。其实,这样的观点,是最没有原则的;而且它必然导致对娼妓制度的容忍乃至认同。我们没有什么理由指责通过国家政治消灭娼妓制度本身,因为这里不只有崇高理想的表达,而且尊重了自然人性。当然,如果放弃了崇高的理想,走向了真实的背叛,那讲什么样的奇谈怪论,都不足为奇、不足为怪了。当然,我们看待娼妓史,也不会仅仅从国家政治的角度出发,虽然娼妓制度并不能够独立于国家政治之外。通过娼妓史,我们往往可以洞见一个社会最为真实的情形。可以说,在这里,既有诗酒风流的文雅,也有世俗欲望的粗俗;既有如花的美丽,又有相伴而生的罪恶;既有自由的男女关系,又有伦理道德的束缚;既有生命的光华,又有悲苦的命运;既有离合之情,又有兴亡之恨。也可以说,生活的万花筒,就在青楼文化这里展开了。当然,这些一方面已经成为历史,另一方面却依然在现实中延续。 

(四)因私有制的恢复而恢复

娼妓制度会因为私有制的恢复而恢复,这一点已经为现实本身所印证。在这里,我并不想讲太多,因为这必然涉及到对私有制的批判。但是,现在做这个工作,显然是吃力不讨好的事情。我们在这里,要讨论的问题是,如何对待娼妓制度本身。当然,在现在的意识形态下,娼妓制度本身显然是非法的存在。而这一点,实在可以看做毛泽东利用国家政治消灭娼妓制度,留下的最重要亦是最宝贵的遗产。所以,我们不仅要尊重这份遗产,更要尊重保留在这份遗产中的崇高理想。当然,现在有的社会学家、性学家,从尊重妇女以及性自由、性解放的角度出发,主张实现卖淫合法化。如果这个提议变成现实,那娼妓制度就由非法的存在转变为合法的存在。我们在这里要看的问题是,尊重妇女以及性自由、性解放,能不能为卖淫的合法化提供理由?试问卖淫合法化,就是尊重妇女吗?可以说,这里的荒谬是显而易见的。如果以这种方式尊重妇女,那实在是对女性尊严的践踏。但是,我们知道,荒谬的学理往往是以荒谬的现实为支撑的。这里问题的复杂就在于娼妓制度虽然是非法的存在,但是,它毕竟存在。所以,在这里,就有一个如何尊重所谓失足妇女的问题。亦即在这里存在着一个悖论,一方面要打击卖淫,另一方面又要尊重失足妇女。坦率地讲,这个问题很难解决。我们看到,在打击卖淫的过程中,出现在人们视线中的永远都是作为弱者的失足妇女,而另一方面幕后的黑手、以及涉及卖淫的资本,却总能逃到比较安全的角落里。其实,所以出现这样的悖论,根源是很明确的,那就是私有制本身。在不触动根源的情况下,是很难走出所谓的悖论的。以尊重妇女的名义,去实现卖淫的合法化;如此荒谬的逻辑,不知为什么会产生在所谓学者的头脑中。也许,这也只有在人欲那里得到解释了。我们再看一下,能否用所谓的性解放、性自由来解释卖淫。其实,所谓性解放,就是让性本身冲破伦理道德的束缚;而所谓性自由呢,显然就是性解放的终极。可以说,在性自由这里,一方面贞洁的观念破产了,另一方面伦理道德失落了。从某种意义上讲,性自由就是以性本身为旨归的。当然,这种性本身的快乐,并不单纯是动物意义的快乐,它同样具有着精神的、灵魂的意义。其实,我们必须以谨慎的态度来对待性解放、性自由。在我,是赞同把性本身规范在伦理道德之内的;所以,我一直强调性自由的限度。我们可以看到,性解放、性自由似乎论证了卖淫的合理性。因为在卖淫这里,性本身确实冲破了伦理道德的束缚,似乎达到了性自由的境界。但是,很明显,这不过是臆想或者理论的虚造。其实,在卖淫这里,性只是手段,而不是终极;也可以说,它是以身体获取金钱。从某种意义上讲,对身体的出卖即是对灵魂的出卖。也可以说,卖淫并不能够在性自由、性解放这里得到解释。很明显,在性自由、性解放这里,是以性本身为终极的,并且这种性还具有精神的、灵魂的意义。当然,在一些学者还有一个法宝,那就是用自然人性来解释卖淫本身。在他们看来,卖淫是合乎自然人性的;既然合乎自然人性,那又有什么理由不合法化呢?可以说,在这里,自然人性已被引向了纵欲的道路。其实,对欲望的放纵并不能够在自然人性这里得到解释。可以说,通过自然人性去论证卖淫的合理性,所走的是感性路线,所以,也必然地以纵欲为归宿。而一个社会的维系,恰恰是需要理性精神的;当然,在理性的精神,就要求把性本身规范在伦理道德之内了。高尚的理由,可以为卑鄙的行为辩护,这在历史上、现实中并不少见。其实,无论尊重妇女也好,性解放、性自由也罢,都无法论证卖淫的合理性。相反,只有消灭卖淫本身,才有对自然人性的尊重以及伦理正义的表达。许多时候,我们必须穿越理性思辩所设计的圈套,才能够到达真理本身。当思辩智慧服务于现实智慧的时候,它本身就变得不再可靠。在娼妓制度面前,我们没有理由讲“存在的就是合理的”,相反,我们必须强调它的不合理性。也就是说,娼妓制度会为自身的不合理性所埋葬。 

(五)如何终结娼妓制度

如何终结娼妓制度?如果从学理来讲,那就是终结私有制。因为娼妓制度是私有制的产物;所以只要终结了私有制,那娼妓制度就会随之消亡。这可以说是釜底抽薪的法子。但是,现在人们似乎习惯了抱薪救火、扬汤止沸,所以,许多大而无当的话也就不要讲了。我们知道,能够消灭娼妓制度的只有国家政治,而且必须是以公有制为基础的国家政治。被消灭了的娼妓制度,所以死灰复燃,其间的机微是不难洞见的。我们只能说,终结娼妓制度,在目前的情况下,依然只是一种理想的表达。当然,在这里还有另外一个思路,即我们能不能把终结娼妓制度,寄希望于娼妓制度的自然消亡。我讲过,娼妓制度会为自身的不合理性所埋葬。从辩证分析的角度,自然可以这样讲;但是,说娼妓制度会自然消亡,那也只是欺人之谈。我们知道,一样事物的存在,既有合理性,也有不合理性。可以说,在“存在的都是合理的”之外,还有另外一点,即不合理的依然存在。那么,我们应该如何对待不合理的存在呢?答案很简单,那就是消灭它。我们知道,能够彻底扫荡娼妓制度的,也只有作为国家政治的革命本身。而在告别革命的时代,被扫荡了的东西,又死灰复燃,似乎也可以得到合理的解释。但是,终结娼妓制度本身,这个理想不应该被修正。当然,所以如此,还是因为我们的出发点,即尊重妇女,尊重自然人性,表达伦理的正义。可以说,终结娼妓制度,才是对妇女的最大尊重。有社会学家、性学家打着尊重妇女的旗号,竟要实现卖淫的合法化;这其间的荒谬,我们已经揭露了。我们知道,娼妓制度是非常不尊重自然人性的;可以说,在这里自然人性被引向了放纵的道路,而这也难免以自然人性的毁灭为归宿。当然,妄图用自然人性来解释娼妓制度,那就更加得荒谬了。我们知道,娼妓制度可以最大程度的容纳罪恶本身;所以,只有消灭娼妓制度,才会有伦理的正义表达。也可以说,只有消灭了娼妓制度,我们才能够把性本身规范在伦理道德之内。我曾经讲过,我们应该以伦理社会的建构为旨归。在伦理社会里,伦理道德决不是对人本身的束缚;相反,它是人之自由的保证。让性本身冲破伦理道德的束缚,并不能够实现所谓的性解放、性自由。性解放、性自由都是有限度的;否则它本身就是没有保障的。我讲过,不能够用性解放、性自由来解释娼妓制度本身。也就是说,在娼妓制度这里,并没有性解放、性自由;相反,在这里性本身,为金钱所奴役,亦即性只是手段,而金钱才是目的。我们实在可以把娼妓制度本身,比喻为恶之花的。即便恶之花绽放得再妖艳,这里也没有伦理正义的表达。也可以说,结束娼妓制度即是最大的伦理正义。但是,很显然,在现在的情形下,人们对这一点,已经不怎么感兴趣了。如果娼妓制度已经成为历史,那就没有研究的必要了。而现在,这种研究还值得做,那实在是因为存在本身。当然,无论研究什么都需要理解之同情;而这理解之同情,实在服务于入室操戈的目的。亦即研究娼妓制度,即是为了终结娼妓制度。即便无法在现实中终结它,也要在学理上终结它。当然,在学理上终结娼妓制度本身,即是理想的表达。从某种意义上讲,娼妓制度的存在即印证了一点,这个世界本身是由金钱来推动的。可以说,在这里金钱已经把人本身变成了性商品;不仅人的身体成为了性商品,就是精神、情感都成为了性商品。亦即娼妓制度必然会把人本身变成性爱动物。所以,消灭娼妓制度,即是为了人本身的解放,亦即把人本身从性爱动物的宿命中解放出来。我们知道,娼妓制度对人们意识的影响是非常大的。当然,这种影响,不单意味着造就了青楼文化或者娼妓文化,而且意味着造就了泛青楼文化或者泛娼妓文化。当然,泛青楼文化的存在,就印证了娼妓制度的影响范围之广。泛青楼文化的图景是非常可怕的,因为这意味着色情成为了意识形态本身。其实,终结娼妓制度,就是为了避免色情成为意识形态本身。一个有理性的社会,绝不会以色情成为意识形态本身;相反,真正的意识形态必然是打击色情的。

\end{document}