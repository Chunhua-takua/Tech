\documentclass[a4paper]{article}
\usepackage[UTF8, scheme=plain]{ctex}

\usepackage{hyperref}
\hypersetup{
    colorlinks=true,
    linktocpage=true
}
\usepackage[margin=1in]{geometry}
\usepackage{ragged2e}
\usepackage{fancyhdr}
\usepackage{lastpage}
\usepackage{xassoccnt}

% 首行缩进
\usepackage{indentfirst}

% 不显示自动编号 且 能自动生成目录
\setcounter{secnumdepth}
{0}

% 黑体2号
\title{\heiti\zihao{2} 1844年经济学哲学手稿}
\author{Karl Marx}
\date{\today}

% \renewcommand\contentsname{目录}

% 行距
\renewcommand{\baselinestretch}
{1.75}

\fancyhf{}
\pagestyle{fancy} %fancyhdr宏包新增的页面风格

\begin{document}
    % 标题页不显示页码
    \maketitle
    \thispagestyle{empty}

    \newpage

    % 目录 罗马数字页码
    \setcounter{page}{1}
    \pagenumbering{Roman}
    \cfoot{\thepage}

    % foot decorative lines
    \renewcommand{\footrulewidth}{1pt}
    \fancyhead[R]{\textbf{1844年经济学哲学手稿}}
    \tableofcontents

    \newpage

    % 正文 阿拉伯数字页码

    \setcounter{page}{1}
    \pagenumbering{arabic}
    % 当前页 of 总页数
    \cfoot{Page \thepage\ of \pageref{LastPage}}

    \zihao{4}

    \begin{sloppy}
        \part{第一手稿}
        \section{
            工资
        }
        \setlength{\parindent}{2em}
        工资决定于资本家和工人之间的敌对的斗争。胜利必定属于资本家。资本家没有工人能比工人没有资本家活得长久。资本家的联合是很通常而卓有成效的,工人的联合则遭到禁止并会给他们招来恶果。此外,土地所有者和资本家可以把产业收益加进自己的收入,而工人除了劳动所得既无地租,也无资本利息。所以,工人之间的竞争是很激烈的。从而,资本、地产和劳动三者的分离,只有对工人来说才是必然的、本质的、有害的分离。资本和地产无须停留于这种分离,而工人的劳动则不能摆脱这种分离。
        
        最低的和唯一必要的工资额就是工人在劳动期间的生活费用,再加上使工人能够养家活口并使工人种族不致死绝的费用。按照斯密的意见,通常的工资就是同``普通人''即畜类的生活水平相适应的最低工资。

        工人和资本家同样在苦恼时,工人是为他的生存而苦恼,资本家则是为他的死钱财的赢利而苦恼。

        如果社会财富处于衰落状态,那末工人所受的痛苦最大。因为,即使在社会的幸福状态中工人阶级也不可能取得像所有者阶级所取得的那么多好处,“没有一个阶级像工人阶级那样因社会财富的衰落而遭受深重的苦难”。

        现在且拿财富正在增进的社会来看。这是对工人唯一有利的状态。这里资本家之间展开竞争。对工人的需求超过了工人的供给。
        
        但是,第一,工资的提高引起工人的过度劳动。他们越想多挣几个钱,他们就越不得牺牲自己的时间,并且完全放弃一切自由来替贪婪者从事奴隶劳动。这就缩短了工人的寿命。工人寿命的缩短对整个工人阶级是一个有利状况,因为这样就必然会不断产生对劳动的新需求,这个阶级始终不得不牺牲自己的一部分,以避免同归于尽。

        在福利增长的社会中,只有最富有的人才能靠货币利息生活。其余的人都不得不用自己的资本经营某种行业,或者把自己的资本投入商业。这样一来,资本家之间的竞争就会加剧,资本家的积聚就会增强,大资本家使小资本家陷于破产,一部分先前的资本家就沦为工人阶级,而工人阶级则由于这种增加,部分地又要经受工资降低之苦,同时更加依赖于少数大资本家。资本家由于人数减少,他们为争夺工人而进行的竞争几乎不再存在;而工人由于人数增加,彼此间的竞争变得越来越激烈、反常和带有强制性。正像一部分中等资本家必然沦为工人等级一样。

        由此可见,即使在对工人最有利的社会状态中,工人的结局也必然是:过度劳动和早死,沦为机器,沦为资本家的奴隶(资本的积累作为某种有危险的东西而与他相对立),发生新的竞争以及一部分工人饿死或行乞。
    \end{sloppy}
\end{document}
