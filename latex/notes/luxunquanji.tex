\documentclass[a4paper]{article}
\usepackage[UTF8, scheme=plain]{ctex}

\usepackage{hyperref}
\hypersetup{
    colorlinks=true,
    linktocpage=true
}
\usepackage[margin=1in]{geometry}
\usepackage{ragged2e}
\usepackage{fancyhdr}
\usepackage{lastpage}
\usepackage{xassoccnt}

% 首行缩进
\usepackage{indentfirst}

% 不显示自动编号 且 能自动生成目录
\setcounter{secnumdepth}
{0}

% 黑体2号
\title{\heiti\zihao{2} 《鲁迅全集》笔记}
\author{TKA}
\date{\today}

% \renewcommand\contentsname{目录}

% 行距
\renewcommand{\baselinestretch}
{1.75}

\fancyhf{}
\pagestyle{fancy} %fancyhdr宏包新增的页面风格

\begin{document}
    % 标题页不显示页码
    \maketitle
    \thispagestyle{empty}

    \newpage

    % 目录 罗马数字页码
    \setcounter{page}{1}
    \pagenumbering{Roman}
    \cfoot{\thepage}

    % foot decorative lines
    \renewcommand{\footrulewidth}{1pt}
    \fancyhead[R]{\textbf{《鲁迅全集》笔记}}
    \tableofcontents

    \newpage

    % 正文 阿拉伯数字页码

    \setcounter{page}{1}
    \pagenumbering{arabic}
    % 当前页 of 总页数
    \cfoot{Page \thepage\ of \pageref{LastPage}}

    \zihao{4}

    \begin{sloppy}
		\part{呐喊}
		\section{
			自序		
		} {
            \setlength{\parindent}{2em}
			凡有一人的主张,得了赞美,是促其前进的,得了反对,是促其奋斗的,独有叫喊于生人中,而生人并无反应,既非赞同,也无反对,如置身毫无边际的荒原,无可措手的了,这是怎样的悲哀呵,我于是以我所感到者为寂寞。
		}
		
	
        \part{而已集}
        \section{
            “公理”之所在
        } {
            \setlength{\parindent}{2em}
            我的话已经说完,去年说的,今年还适用,恐怕明年也还适用。但我诚恳地希望他不至于适用到十年二十年之后。
        }

        \section{
            答有恒先生
        }
        我也在救助我自己,还是老法子:一是麻痹,二是忘却。

        \section{
            读所谓“大内档案”
        }

        军阀是不看杂志的,就靠叭儿狗嗅,候补叭儿狗吠。

        \section{
            革命时代的文学
        }
        但为什么人类成了人,猴子终于是猴子呢?这就因为猴子不肯变化——它爱用四只脚走路。也许曾有一个猴子站起来,试用两脚走路的罢,但许多猴子就说:“我们底祖先一向是爬的,不许你站!”咬死了。

        有些民族因为叫苦无用,连苦也不叫了。

        所有的文学,歌呀,诗呀,大抵是给上等人看的:他们吃饭了,睡在躺椅上,捧着看。

        或者讲上等人怎样有趣和快乐,下等人怎样可笑。
        
        下等人没奈何,也只好替他们一同欢喜欢喜。

        \section{
            黄花节的杂感
        }
        久受压制的人们,被压制时只能忍苦,幸而解放了便只知道作乐,悲壮剧是不能久留在记忆里的。

        然而革命成功的时候,革命家死掉了,却能每年给生存的大家以热闹,甚而至于欢欣鼓舞。惟独革命家,无论他生或死,都能给大家以幸福。

        \section{
            略谈香港
        }
        其实是种族革命,要将土地从异族的手里取得,归还旧主人。

        \section{
            谈“激烈”
        }
        因为奴才都叹气,虽无大害,主人看了究竟不舒服。必须要如罗素所称赞的杭州的轿夫一样,常是笑嘻嘻。

        \section{
            小杂感
        }
        约翰穆勒说:专制使人们变成冷嘲。

        而他竟不知道共和使人们变成沉默。

        \vspace{1em}
        曾经阔气的要复古,正在阔气的要保持现状,未曾阔气的要革新。

        大抵如是。大抵!

        \vspace{1em}
        人类的悲欢并不相通,我只觉得他们吵闹。

        \vspace{1em}
        叭儿狗往往比它的主人更严厉。

        \vspace{1em}
        凡为当局所“诛”者皆有“罪”。

        \vspace{1em}
        法三章者,话一句耳。

        \part{二心集}
        \section{
            “好政府主义”
        }
        有如被压榨得痛了,就要叫喊,原不必在想出更好的主义之前,就定要咬住牙关。

        他在药方上所开的却不是药名,而是“好药料”三个大字。

        \section{
            “丧家的”“资本家的乏走狗”
        }
        凡走狗,虽或为一个资本家所豢养,其实是属于所有的资本家的,所以它遇见所有的阔人都驯良,遇见所有的穷人都狂吠。不知道谁是它的主子,正是它遇见所有阔人都驯良的原因,也就是属于所有的资本家的证据。即使无人豢养,饿的精瘦,变成野狗了,但还是遇见所有的阔人都驯良,遇见所有的穷人都狂吠的,不过这时它就愈不明白谁是主子了。

        \section{
            “硬译”与“文学的阶级性”
        }
        至于无产者应该“辛辛苦苦”爬有产阶级去的“正当”的方法,则是中国有钱的老太爷高兴时候,教导穷工人的古训,在实际上,现今正在“辛辛苦苦诚诚实实”想爬上一级去的“无产者”也还多。

        \section{
            “友邦惊诧”论
        }
        “友邦人士,莫名惊诧,长此以往,国将不国”了!

        \section{
            “智识劳动者”万岁
        }
        “劳动者”这句话成了“罪人”的代名词,已经足足四年了。压迫罢,谁也不响;杀戮罢,谁也不响;文学上一提起这句话,就有许多“文人学士”和“正人君子”来笑骂,接送又有许多他们的徒子徒孙来笑骂。劳动者呀劳动者,真要永世不得翻身了。

        \section{
            沉滓的泛起
        }
        要趁“国难声中”或“和平声中”将利益更多的榨到自己的手里的。

        \section{
            对于左翼作家联盟的意见
        }
        倘若不和实际的社会斗争接触,单关在玻璃窗内做文章,研究问题,那是无论怎样的激烈,“左”,都是容易办到的;然而一碰到实际,便即刻要撞碎了。

        现在为劳动大人众革命,将来革命成功,劳动阶级一定从丰报酬,特别优待,请他坐特等车,吃特等饭,或者劳动者捧着牛油面包来献他,说:“我们的诗人,请用吧!”这也是不正确的;因为实际上决不会有这种事,恐怕那时比现在还要苦,不但没有牛油面包,连黑面包都没有也说不定,俄国革命后一二年的情形便是例子。

        “反动派且已经有联合战线了,而我们还没有团结起来!”只因为他们的目的相同,所以行动就一致,在我们 看来就好像联合战线。

        \section{
            非革命的急进革命论者
        }
        他现为批评家而说话的时候,就随便捞到一种东西以驳诘相反的东西。要驳互助说时用争存说,驳争存说时用互助说;反对和平论时用阶级争斗说,反对斗争时就主张人类之爱。论敌是唯心论者呢,他的立场是唯物论,待到和唯物论者相辩难,他却又化为唯心论者了。

        \section{
            关于翻译的通信
        }
        不可与言而与之言,失言。

        这正如俄国革命以后,欧美的富家奴去看了一看,回来就摇头皱脸,做出文章,慨叹着工农还在怎样吃苦,怎样忍饥,说得满纸凄凄惨惨。仿佛惟有他却是极希望一个筋斗,工农就都住王宫,吃大菜,躺安乐椅子享福的人。谁料还是苦,所以俄国不行了,革命不好了,阿呀阿呀了,可恶之极了。

        \section{
            黑暗中国的文艺界的现状
        }
        他以为文艺原不是无产阶级的东西,无产者倘要创作或鉴赏文艺,先应该辛苦地积钱,爬上资产阶级去,而不应该大家浑身褴褛,到这花园中来吵嚷。

        \section{
            上海文艺之一瞥
        }
        去嫖的时候,可以叫十个二十个的年青姑娘聚集在一处,样子很有些像《红楼梦》,于是他就觉得自己好像贾宝玉;自己是才子,那么婊子当然是佳人,于是才子佳人的书就产生了。

        佳人并非因为“爱才若渴”而做婊子的,佳人只为的是钱。

        其实革命是并非教人死而是教人活的。

        要人帮忙时候用克鲁巴金的互助论,要和人争闹的时候就用达尔文的生存竞争说。

        \section{
            唐朝的钉梢
        }
        即使骂,也就大有希望,因为一骂便可有言语来往,所以也就是“扳谈”的开头。

        \section{
            现代电影与有产阶级
        }
        在实际上,电影是以大多数的小市民和无产阶级为看客的。而他们,小市民和无产阶级,早已渐渐地觉察出有产阶级的诡计来了。就是,已经注意于“支配阶级制作了宣布那服从于己的观念形态的影片,而以此来做掠取无产者的衣袋的手段”这事实的真相了。

        拙劣的煽动,却招致反对的结果。

        露骨的宣传是停止了。最所希望的,是使电影的看客看不见“阶级”这观念。至少,是坐在银幕之前的数小时中,使他们忘却了一切社会底对立。
        这样子,就产生了小市民的影片。

        过屠门与大嚼,虽不得肉,亦且快意。

        \section{
            序言
        }
        在坏了下去的旧社会里,倘有人怀一点不同的意见,有一点携贰的心思,是一定要吃其苦的。而攻击陷害得最凶的,则是这人的同阶级的人物。

        \section{
            宣传与做戏
        }
        全国的人们十之九不识字,然而总得请几位博士,使他对西洋人去讲中国的精神文明。

        \section{
            中国无产阶级革命文学和前驱的血
        }
        我们的劳苦大众历来只被最剧烈的压迫和榨取,连识字教育的布施也得不到,惟有默默地身受着宰割和灭亡。繁难的象形字,又使他们不能有自修的机会。

    \end{sloppy}
\end{document}
