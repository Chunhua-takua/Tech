\documentclass[a4paper]{article}
\usepackage[UTF8, scheme=plain]{ctex}
% \usepackage[hidelinks]{hyperref}
\usepackage{hyperref}
\hypersetup{
    colorlinks=true,
    linktocpage=true
}
\usepackage[margin=1in]{geometry}
\usepackage{ragged2e}
\usepackage{fancyhdr}
\usepackage{lastpage}
\usepackage{xassoccnt}

% 不显示自动编号 且 能自动生成目录
\setcounter{secnumdepth}
{0}

% 黑体2号
\title{\heiti\zihao{2} 《鲁迅全集》笔记}
\author{TKA}
\date{\today}

% \renewcommand\contentsname{目录}

% 行距
\renewcommand{\baselinestretch}
{1.75}

\fancyhf{}
\pagestyle{fancy} %fancyhdr宏包新增的页面风格

\begin{document}
    % 标题页不显示页码
    \maketitle
    \thispagestyle{empty}

    \newpage

    % 目录 罗马数字页码
    \setcounter{page}{1}
    \pagenumbering{Roman}
    \cfoot{\thepage}

    % foot decorative lines
    \renewcommand{\footrulewidth}{1pt}
    \fancyhead[R]{\textbf{《鲁迅全集》笔记}}
    \tableofcontents

    \newpage

    % 正文 阿拉伯数字页码

    \setcounter{page}{1}
    \pagenumbering{arabic}
    % 当前页 of 总页数
    \cfoot{Page \thepage\ of \pageref{LastPage}}

    \zihao{4}
    \part{而已集}

    \section{“公理”之所在}
    我的话已经说完,去年说的,今年还适用,恐怕明年也还适用。但我诚恳地希望他不至于适用到十年二十年之后。

    \section{答有恒先生}
    我也在救助我自己,还是老法子:一是麻痹,二是忘却。

    \section{读所谓“大内档案”}

    军阀是不看杂志的,就靠叭儿狗嗅,候补叭儿狗吠。

    \section{革命时代的文学}
但为什么人类成了人,猴子终于是猴子呢?这就因为猴子不肯变化——它爱用四只脚走路。也许曾有一个猴子站起来,试用两脚走路的罢,但许多猴子就说:“我们底祖先一向是爬的,不许你站!”咬死了。

有些民族因为叫苦无用,连苦也不叫了。

所有的文学,歌呀,诗呀,大抵是给上等人看的:他们吃饭了,睡在躺椅上,捧着看。

或者讲上等人怎样有趣和快乐,下等人怎样可笑。
下等人没奈何,也只好替他们一同欢喜欢喜。

\section{黄花节的杂感}

久受压制的人们,被压制时只能忍苦,幸而解放了便只知道作乐,悲壮剧是不能久留在记忆里的。

然而革命成功的时候,革命家死掉了,却能每年给生存的大家以热闹,甚而至于欢欣鼓舞。惟独革命家,无论他生或死,都能给大家以幸福。

\section{略谈香港}

其实是种族革命,要将土地从异族的手里取得,归还旧主人。

\section{谈“激烈”}

因为奴才都叹气,虽无大害,主人看了究竟不舒服。必须要如罗素所称赞的杭州的轿夫一样,常是笑嘻嘻。

\section{小杂感}

曾经阔气的要复古,正在阔气的要保持现状,未曾阔气的要革新。
大抵如是。大抵!

人类的悲欢并不相通,我只觉得他们吵闹。

叭儿狗往往比它的主人更严厉。

凡为当局所“诛”者皆有“罪”。

法三章者,话一句耳。


\end{document}