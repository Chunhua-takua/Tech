\documentclass[a4paper]{article}
\usepackage[UTF8, scheme=plain]{ctex}
% \usepackage[hidelinks]{hyperref}
\usepackage{hyperref}
\hypersetup{
    colorlinks=true,
    linktocpage=true
}
\usepackage[margin=1in]{geometry}
\usepackage{ragged2e}
\usepackage{fancyhdr}
\usepackage{lastpage}
\usepackage{xassoccnt}

% 不显示自动编号 且 能自动生成目录
\setcounter{secnumdepth}
{0}

% 黑体2号
\title{\heiti\zihao{2} 德国的革命和反革命}
\author{TKA}
\date{\today}

\renewcommand\contentsname{目录}

% 行距
\renewcommand{\baselinestretch}
{1.75}

\fancyhf{}
\pagestyle{fancy} %fancyhdr宏包新增的页面风格

\begin{document}
    \maketitle

    \tableofcontents

    \part{革命前夕的德国}


    把革命的发生归咎于少数煽动者的恶意的那种迷信时代,是早已过去了。现在每个人都知道,任何地方发生革命震动,总是有一种社会要求为其背景,而腐朽的制度阻碍这种要求得到满足。这种要求也许还未被人强烈地普遍地感觉到,因此还不能立即得到胜利;但是,如果企图用暴力来压制这种要求,那只能使它愈来愈强烈,直到最后把它的枷锁打碎。所以,如果我们被打败了,我们就只有再从头干起。值得庆幸的是,在运动的第一幕闭幕之后和第二幕开幕之前,有一个大约很短的休息使我们有时间来做一件很紧要的工作:研究决定这次革命必然爆发而又必然失败的原因。这些原因不应该从几个领袖的偶然的动机、优点、缺点、错误或变节中寻找,而应该从每个经历了震动的国家的总的社会状况和生活条件中寻找。1848年2月和3月突然爆发的运动,不是少数几个人活动的结果,而是人民的要求和需要的自发的不可遏止的表现,每个国家的各个阶级对这种要求和需要的认识程度虽然各不相同,但都已清楚地感觉到,——这已经是一件公认的事实。但当你问到反革命成功的原因时,你却到处听到一种现成的回答:因为某甲或某乙“出卖”了人民。从具体情况来看,这种回答也许正确,也许错误,但在任何情况下,它都不能解释半点东西,甚至不能说明,``人民''怎么会让别人出卖自己。而且,如果一个政党的全部本钱就只是知道某某人不可靠这一件事,它的前途就太可悲了。




\end{document}